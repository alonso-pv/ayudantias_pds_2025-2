\documentclass[12pt]{exam}
\usepackage{../preamble}

% disable para esconder
\usepackage[enable]{easy-todo}
\usepackage{bm}
\usepackage{amssymb,amsmath}

% Comentar para esconder soluciones
\printanswers

%%%%%%%%%%%%%%%%%%%%%%%%%%%%%%%%
%% -- Inicio del Documento -- %%
%%%%%%%%%%%%%%%%%%%%%%%%%%%%%%%%
\begin{document}
\begin{center}
    {\Large
        Ayudantía 7 - Procesamiento Digital de Señales
    }
\end{center}

\begin{questions}
\question
Sea la secuencia $x[n]=\{1\;1\;0\;3\}$
\begin{parts}
    \part Calcule la TFD $X[k]$ utilizando la matriz $\bm{W_4}$.
    \begin{solution}
        Recordemos que la TFD se define por:
        \begin{equation}
            X[k] = \sum_{n=0}^{N-1}x[n]e^{-j\frac{2\pi}{N}kn}
        \end{equation}

        Y definiendo $W_N=e^{-j\frac{2\pi}{N}}$:
        \begin{equation}
            X[k] = \sum_{n=0}^{N-1}x[n]W_N^{kn}
        \end{equation}
        Esta suma se puede reescribir como un producto entre una matriz $\bm{W_N}$
        y un vector $\bm x$. De forma general:
        \begin{gather}
            \bm{X}=\bm{W_Nx}\\
            \begin{bmatrix}
                X[0]\\X[1]\\X[2]\\\vdots\\X[N-1]
            \end{bmatrix}
            =\begin{bmatrix}
                1 & 1 & 1 & \dots & 1\\
                1 & W_N^1 & W_N^2 & \dots & W_N^{N-1}\\  
                1 & W_N^2 & W_N^4 & \dots & W_N^{2(N-1)}\\  
                \vdots & \vdots & \vdots & \ddots & \vdots\\  
                1 & W_N^{N-1} & W_N^{2(N-1)} & \dots & W_N^{(N-1)(N-1)}\\  
            \end{bmatrix}
            \begin{bmatrix}
                x[0]\\  x[1]\\  x[2]\\ \vdots\\x[N-1]
            \end{bmatrix}
        \end{gather}
        Para la matriz $\bm{W_N}$: 
        \begin{itemize}
            \item Tiene dimensiones $N\times N$.
            \item La primera fila y columna son solo unos.
            \item Para la fila $k$, los exponentes de $W_N$ aumentan de $k-1$ en $k-1$. Ej: La segunda fila los exponentes
        van de uno en uno, para la tercera fila van de dos en dos, etc...
        \item Es simétrica ($\bm{W_N=W_N^\text{T}}$)
        \end{itemize} 
        En el caso de la secuencia $x[n]=\{1\; 1\; 0\; 3\}$ ($N=4$) la transformada es:
        \begin{align}
            X[k]=\sum_{n=0}^{3}x[n]W_{4}^{kn}
        \end{align}
        y en forma matricial:
        \begin{gather}
            \bm{X}=\bm{W_4x}\\
            \begin{bmatrix}
                X[0]\\ X[1]\\ X[2]\\ X[3] 
            \end{bmatrix}
            =\begin{bmatrix}
                1 & 1 & 1 & 1\\
                1 & W_4^1 & W_4^2 & W_4^3 \\
                1 & W_4^2 & W_4^4 & W_4^6 \\
                1 & W_4^3 & W_4^6 & W_4^9 \\
            \end{bmatrix}
            \begin{bmatrix}
                x[0]\\  x[1]\\  x[2]\\  x[3]
            \end{bmatrix}
        \end{gather}
        Como $W_N^n=W_N^{n+N}$, entonces $W_4^4=W_4^0$, $W_4^6=W_4^2$ y $W_4^9=W_4^1$ se escribe:
         \begin{align}
            \begin{bmatrix}
                X[0]\\ X[1]\\ X[2]\\ X[3] 
            \end{bmatrix}
            =\begin{bmatrix}
                1 & 1 & 1 & 1\\
                1 & W_4^1 & W_4^2 & W_4^3 \\
                1 & W_4^2 & W_4^0 & W_4^2 \\
                1 & W_4^3 & W_4^2 & W_4^1 \\
            \end{bmatrix}
            \begin{bmatrix}
                x[0]\\  x[1]\\  x[2]\\  x[3]
            \end{bmatrix}
        \end{align}
        Notando que $W_4^0=1$, $W_4^1=e^{-j\frac{2\pi}{4}}=-j$, $W_4^2=-1$ y $W_4^3=j$, y reemplazando
        por los valores de $x[n]$:
        \begin{align}
            \begin{bmatrix}
                X[0]\\ X[1]\\ X[2]\\ X[3] 
            \end{bmatrix}
            =\begin{bmatrix}
                1 & 1 & 1 & 1\\
                1 & -j & -1 & j \\
                1 & -1 & 1 & -1 \\
                1 & j & -1 & -j \\
            \end{bmatrix}
            \begin{bmatrix}
                1\\  1\\  0\\  3
            \end{bmatrix}
            =\begin{bmatrix}
                1+1+0+3\\
                1-j+0+3j\\
                1-1+0-3\\
                1+j+0-3j
            \end{bmatrix}
            =
            \begin{bmatrix}
                5\\1+2j\\-3\\1-2j
            \end{bmatrix}
        \end{align}
        Es decir: $X[k]=\{5\quad {1+2j}\quad-3\quad1-2j\}$
    \end{solution}

    \part Determine la la TFD inversa $z[n]$ de $W^{2k}_4X[k]$. Grafique $z[n]$
    y $x[n]$ para su comparación.
    \begin{solution}
        Para el cálculo de la transformada inversa de
        $W_4^{2k}X[k]=e^{-j\frac{2\pi}{4}2k}X[k]$ debemos tener en cuenta la propiedad
        de desplazamiento circular. Sea una señal $x[n]$ y $X[k]$ su TFD de $N$ puntos:
        \begin{align}
            \boxed{x[\langle n-m \rangle_N]
            \stackrel{TFD}{\longleftrightarrow}
            W_N^{mk}X[k]=e^{-j\frac{2\pi}{N}mk}X[k]}
        \end{align} Con $\langle*\rangle_N$ la operación módulo $N$. Notar que la
        exponencial debe estar en forma $e^{-j\frac{2\pi}{N}mk}$.
        
        Al aplicar la propiedad en nuestro caso ($N=4$, $m=2$):
        \begin{equation}
            W_4^{2k}X[k]\xrightarrow[]{TFDI}z[n]=x[\langle n-2 \rangle_4]
        \end{equation}

        Para encontrar los valores de la señal para $n=0,\,1,\,2,\,3$ se retarda
        la señal en $m=2$ rotando los valores que se salen del rango $0\leq n\leq 3=N-1$:
        \begin{align}
            x[n]&=\{1\;1\;0\;3\}\\
            &\downarrow\nonumber\\
            z[n]=x[\langle n-2\rangle_4]&=\{0\;3\;1\;1\}
        \end{align}

        Al graficar $x[n]$, su extensión periódica y $z[n]$ vemos como la TFD no ``ve''
        la señal $x[n]$ sino su extensión periódica al aplicar la propiedad de desplazamiento:
        
        \includegraphics[width=14cm]{imagenes/7.1b.png}
        
    \end{solution}

    \part Realice la convolución circular $x[n]\circledast h[n]$, con $h[n]=\{1\;0\;1\;0\}$
    en el plano temporal.
    \begin{solution}
        La convolución circular se define por:
        \begin{align}
            y[n]=x[n]\circledast h[n]&=\sum_{m=0}^{N-1}h[n]x[\langle n-m\rangle_N]\\
            &=\sum_{m=0}^{3}h[n]x[\langle n-m\rangle_4]
        \end{align}

        Lo que se puede expresar en forma matricial:
        \begin{align}
            \begin{bmatrix}
                y[0]\\ y[1]\\ y[2]\\ y[3]
            \end{bmatrix}
            &=\begin{bmatrix}
                x[0] & x[3] & x[2] & x[1] \\ 
                x[1] & x[0] & x[3] & x[2] \\ 
                x[2] & x[1] & x[0] & x[3] \\ 
                x[3] & x[2] & x[1] & x[0]
            \end{bmatrix}
            \begin{bmatrix}
                h[0]\\ h[1]\\ h[2]\\ h[3]
            \end{bmatrix}\\
            &=\begin{bmatrix}
               1 & 3 & 0 & 1 \\ 
               1 & 1 & 3 & 0 \\ 
               0 & 1 & 1 & 3 \\ 
               3 & 0 & 1 & 1 
            \end{bmatrix}
            \begin{bmatrix}
                1 \\ 0 \\ 1 \\ 0
            \end{bmatrix}
            =\begin{bmatrix}
               1 \\ 1+3 \\ 1 \\ 3+1 
            \end{bmatrix}
            =\begin{bmatrix}
                1 \\ 4 \\ 1 \\ 4
            \end{bmatrix}
        \end{align}

        Por lo tanto: $y[n]= \{1\;4\;1\;4\}$

    \end{solution}

    \part Repita (c) usando las TFD $X[k]$ y $H[k]$.
    \begin{solution}
        Del ejercicio 1(a): $X[k]=\{5\quad {1+2j}\quad-3\quad1-2j\}$. $H[k]$ se 
        puede obtener de forma similar con la matriz $\bm{W_4}$ y un vector $\bm{h}$:
        \begin{gather}
            \bm{H}=\bm{W_4h}\\
            \begin{align}
            \begin{bmatrix}
                H[0]\\ H[1]\\ H[2]\\ H[3] 
            \end{bmatrix}
            &=\begin{bmatrix}
                1 & 1 & 1 & 1\\
                1 & -j & -1 & j \\
                1 & -1 & 1 & -1 \\
                1 & j & -1 & -j \\
            \end{bmatrix}
            \begin{bmatrix}
                h[0]\\  h[1]\\  h[2]\\  h[3]
            \end{bmatrix}\\
            &=\begin{bmatrix}
                1 & 1 & 1 & 1\\
                1 & -j & -1 & j \\
                1 & -1 & 1 & -1 \\
                1 & j & -1 & -j \\
            \end{bmatrix}
            \begin{bmatrix}
                1\\  0\\  1\\  0
            \end{bmatrix}
            =\begin{bmatrix}
                1+1\\1-1\\1+1\\1-1
            \end{bmatrix}
            =\begin{bmatrix}
                2\\0\\2\\0
            \end{bmatrix}
        \end{align}
        \end{gather}
        Es decir: $H[k]=\{2\;0\;2\;0\}$.

    Luego aplicamos la propiedad que relaciona la convolución cíclica con la TFD:
    \begin{align}
        \boxed{y[n]=x[n]\circledast h[n]\stackrel{TFD}{\longleftrightarrow}Y[k]=X[k]H[k]}
    \end{align}
    Es decir, multiplicando las transformadas y aplicando la TFDI a $Y[k]$ obtenemos $y[n]$.
    La transformada $Y[k]$ es:
    \begin{align}
       X[k]&=\{5\quad 1+2j\quad -3\quad 1-2j\}\nonumber\\
       H[k]&=\{2\quad 0\quad 2\quad 0\}\notag\\
       Y[k]=X[k]H[k]&=\{10\quad0\quad -6\quad0\}
    \end{align}
    
    Aplicamos la transformada inversa. Matricialmente:
    \begin{align}
        \bm{y}=\bm{W_4^{-1}Y}
    \end{align}

    Recordando que en general $\bm{W_N^{-1}}={1 \over N}\bm{W_N^*}$, tenemos:
    \begin{gather}
        \begin{align}
        \bm{y}&=\bm{W_4^{-1}Y}\\
        &=\frac{1}{4}\bm{W_4^*Y}
        \end{align}\\
        \begin{align}
        \begin{bmatrix}
            y[0]\\ y[1]\\ y[2]\\ y[3]
        \end{bmatrix}
        &=\frac{1}{4}
        \begin{bmatrix}
                1 & 1 & 1 & 1\\
                1 & -j & -1 & j \\
                1 & -1 & 1 & -1 \\
                1 & j & -1 & -j \\
        \end{bmatrix}^*
        \begin{bmatrix}
            Y[0]\\ Y[1]\\ Y[2]\\ Y[3]
        \end{bmatrix}\\
        &=\frac{1}{4}
        \begin{bmatrix}
                1 & 1 & 1 & 1\\
                1 & j & -1 & -j \\
                1 & -1 & 1 & -1 \\
                1 & -j & -1 & j \\
        \end{bmatrix}
        \begin{bmatrix}
            10\\ 0\\ -6\\ 0
        \end{bmatrix}\\
        &=\frac{1}{4}\begin{bmatrix}
            10-6\\10+6\\10-6\\10+6
        \end{bmatrix}
        =\frac{1}{4}\begin{bmatrix}
            4\\16\\4\\16
        \end{bmatrix}
        =\begin{bmatrix}
           1\\4\\1\\4 
        \end{bmatrix}
        \end{align}
    \end{gather}
    Es decir: $y[n]=\{1\;4\;1\;4\}$. Coincide con la convolución en el tiempo.
    \end{solution}
\end{parts}

\question Muestre brevemente como se realizaría el calculo de la TFD de $x[n]=\{0\;1\;2\;3\}$
usando el algoritmo de Transformada Rápida de Fourier visto en clases.
\begin{solution}
    Un algoritmo de Transformada Rápida de Fourier divide la señal en el tiempo en muestras
    pares e impares ($\bm{x_p}$ y $\bm{x_i}$), aplica la TFD (es recursivo!) a cada una y las vuelve a unir
    de acuerdo al siguiente esquema:

    \begin{align}
        x[n]\rightarrow
        \stackrel{FFT}{
        \boxed{
        \bm{x}=\begin{bmatrix}
            x[0]\\
            x[1]\\
            x[2]\\
            x[3]\\
            x[4]\\
            \vdots\\
            x[N-1]
        \end{bmatrix}
        \begin{matrix}
            \nearrow\\ \\ \\
            \searrow
        \end{matrix}
        \begin{matrix}
            \bm{x_p}=\begin{bmatrix}
                x[0]\\x[2]\\x[4]\\\vdots
            \end{bmatrix}\stackrel{FFT}{\longrightarrow}\bm{X_p}\\
            \\
            \bm{x_i}=\begin{bmatrix}
                x[1]\\x[3]\\x[6]\\\vdots
            \end{bmatrix}\stackrel{FFT}{\longrightarrow}\bm{X_i}
        \end{matrix}
        \begin{matrix}
            \searrow\\ \\ \\
            \nearrow
        \end{matrix}
        \begin{matrix}
            \bm{X_p+D_NX_i = X_T}\longrightarrow\\
            \bm{X_p-D_NX_i = X_B}\longrightarrow
        \end{matrix}
        \begin{bmatrix}
            \bm{X_T}\\
            \bm{X_B}
        \end{bmatrix}=\bm{X}
        }
        }\rightarrow X[k]
    \end{align}
    Con la matriz $\bm{D_N}$ dada por:
    \begin{equation} 
        \bm{D_N}=\begin{bmatrix}
            1 & 0& 0 & \dots & 0\\            
            0 & W_N^1& 0 & \dots &  0\\            
            0 & 0& W_N^2 & \dots & 0\\            
            \vdots & \vdots& \vdots & \ddots& \vdots\\            
            0 & 0& 0 & \dots& W_N^{\frac{N}{2}-1}\\            
        \end{bmatrix}
    \end{equation}
    Al aplicar recursivamente el algoritmo las secuencias $\bm{x_p}$ y $\bm{x_i}$
    se vuelven a dividir, cada una en dos sub-secuencias, hasta llegar a una
    secuencia de largo 1. Al aplicar la TFD a una secuencia de largo 1, se obtiene la misma
    secuencia, y se comienza a construir la TFD.

    En el caso de la señal $x[n]=\{0\;1\;2\;3\}$ ($N=4$), al aplicar recursivamente
    el algoritmo, y notando que $\bm{D_2}=[1]$ y $\bm{D_4}=\begin{bmatrix}
        1 & 0\\ 0 & W_4
    \end{bmatrix}=\begin{bmatrix}
        1 & 0\\ 0 & -j
    \end{bmatrix}$:

    \begin{gather}
    \begin{bmatrix}
        x[0]\\x[1]\\x[2]\\x[3]
    \end{bmatrix}
    =\begin{bmatrix}
        0\\1\\2\\3
    \end{bmatrix}
    \begin{matrix}
        \nearrow\\ \\ \\
        \searrow
    \end{matrix}
    \begin{matrix}
        \begin{bmatrix}
            0\\ 2
        \end{bmatrix}
        \begin{matrix}
            \nearrow\\ 
            \searrow
        \end{matrix}
        \\ \\ \\ \\
        \begin{bmatrix}
            1\\ 3
        \end{bmatrix}
        \begin{matrix}
            \nearrow\\ 
            \searrow
        \end{matrix}
    \end{matrix}
    \begin{matrix}
        \\
        \begin{bmatrix}
            0
        \end{bmatrix}\stackrel{TFD}{\longrightarrow}
        \begin{bmatrix}
            0
        \end{bmatrix}\\\\
        \begin{bmatrix}
            2
        \end{bmatrix}\stackrel{TFD}{\longrightarrow}
        \begin{bmatrix}
            2
        \end{bmatrix}\\\\\\
        \begin{bmatrix}
            1
        \end{bmatrix}\stackrel{TFD}{\longrightarrow}
        \begin{bmatrix}
            1
        \end{bmatrix}\\\\
        \begin{bmatrix}
            3
        \end{bmatrix}\stackrel{TFD}{\longrightarrow}
        \begin{bmatrix}
            3
        \end{bmatrix}\\\\
    \end{matrix}
    \begin{matrix}
        \begin{matrix}
            \searrow\\\\
            \nearrow
        \end{matrix}
        \begin{bmatrix}
            0 + (1)2\\
            0 - (1)2\\
        \end{bmatrix}
        =\begin{bmatrix}
            2\\
            -2
        \end{bmatrix}
        \searrow
        \\ \\ \\\\
        \begin{matrix}
            \searrow\\\\
            \nearrow
        \end{matrix}
        \begin{bmatrix}
            1 + (1)3\\
            1 - (1)3\\
        \end{bmatrix}
        =\begin{bmatrix}
            4\\
            -2
        \end{bmatrix}
        \nearrow
    \end{matrix}
    \begin{matrix}
        \bm{X_T}=\begin{bmatrix}
            2\\ -2
        \end{bmatrix}
        +\begin{bmatrix}
            1 & 0\\
            0 & -j
        \end{bmatrix}
        \begin{bmatrix}
            4\\-2
        \end{bmatrix}\\\\
        \bm{X_B}=\begin{bmatrix}
            2\\ -2
        \end{bmatrix}
        -\begin{bmatrix}
            1 & 0\\
            0 & -j
        \end{bmatrix}
        \begin{bmatrix}
            4\\-2
        \end{bmatrix}
    \end{matrix}\\
    \begin{matrix}
    \bm{X_T}=\begin{bmatrix}
        2\\-2
    \end{bmatrix}
    +
    \begin{bmatrix}
        4\\2j
    \end{bmatrix}
    =\begin{bmatrix}
        6\\-2+2j
    \end{bmatrix}\searrow
    \\\\
    \bm{X_B}=\begin{bmatrix}
        2\\-2
    \end{bmatrix}
    -
    \begin{bmatrix}
       4\\2j 
    \end{bmatrix}
    =\begin{bmatrix}
        -2\\-2-2j
    \end{bmatrix}\nearrow
    \end{matrix}
    \begin{bmatrix}
        \bm{X_T}\\\bm{X_B}
    \end{bmatrix}=
    \begin{bmatrix}
        6\\-2+2j\\-2\\-2-2j
    \end{bmatrix}
    =\begin{bmatrix}
        X[0]\\X[1]\\X[2]\\X[3]
    \end{bmatrix}
    \end{gather}

\end{solution}

\question Implemente en Matlab el algoritmo de Transformada Rápida de Fourier en
Matlab. Puede asumir que el vector de entrada es de $N=2^{m}$ una potencia de dos.

 \begin{parts}
    \part Compruebe la validez calculando la TFD $x[n]$ de la pregunta 1.
    \begin{solution}
        En la pag. 439 del libro se encuentra una implementación del algoritmo recién
        visto. Sin embargo la implementación del libro tiene dos errores. Arreglando los errores
        se obtiene:
        \begin{minted}[fontsize=\footnotesize]{matlab}
function Xdft = fftrecur(x)
% Recursive computation of the DFT using divide & conquer
% N should be a power of 2
N = length(x);
if N == 1
    Xdft = x;
else
    m = N/2;
    Xp = fftrecur(x(1:2:N));
    Xi = fftrecur(x(2:2:N));
    W = exp(-2j*pi/N).^((0:m-1)'); % <- Aquí había un error
    temp = W.*Xi;
    Xdft = [ Xp+temp ; Xp-temp ]; % <- Aquí había otro
end
end
        \end{minted}

        Y se comprueba que la implementación funciona.
        \begin{minted}[fontsize=\footnotesize]{matlab}
x = [1 1 0 3];
X = fftrecur(x)

X =

   5.0000 + 0.0000i
   1.0000 + 2.0000i
  -3.0000 + 0.0000i
   1.0000 - 2.0000i
        \end{minted}
    \end{solution}
    \part Considere la siguiente implementación directa (por definición) de la TFD:
\begin{minted}[fontsize=\footnotesize]{matlab}
function Xdft=dftdirect(x)
% Direct computation of the DFT
N=length(x); Q=2*pi/N;
for k=1:N
    S=0;
    for n=1:N
        W(k,n)=exp(-1j*Q*(k-1)*(n-1));
        S=S+W(k,n)*x(n);
    end
    Xdft(k)=S;
end
end
\end{minted}

    Usando las funciones \texttt{tic} y \texttt{toc} de Matlab, compare
    el rendimiento de su implementación y la de \texttt{dftdirect}. Para la generación del vector de entrada de prueba puede utilizar la función \texttt{rand}.
    \begin{solution}
        Al comparar el tiempo que demora en ejecutarse un algoritmo de TFD por definición, y otro
        de Transformada Rápida de Fourier hay una diferencia notoria:

        \includegraphics[width=14cm]{imagenes/7.3b.png}
        \begin{minted}[fontsize=\footnotesize]{matlab}
%% 3 (b)

K = 10; % N=2^1 hasta 2^10
reps = 50; % Cada prueba se repite 10 veces y se promedia

k_vec = 1:K;
tiempos_direct = zeros(1,K);
tiempos_fft = zeros(1,K);

for k = k_vec
    for r = 1:reps
        x = rand(1,2^k);

        tic;
        dftdirect(x);
        tiempos_direct(k) = tiempos_direct(k) + toc;

        tic;
        fftrecur(x);
        tiempos_fft(k) = tiempos_fft(k) + toc;
    end
end

figure
subplot(2,1,1)
plot(k_vec, tiempos_direct./reps, 'LineWidth', 2); hold on
plot(k_vec, tiempos_fft./reps, 'LineWidth', 2)
legend("dft direct", "fftrecur")
xlabel("K")
ylabel("Tiempo (segundos)")
title("Tiempo de ejecucion de TFD, N=2^K")
subplot(2,1,2)
semilogy(k_vec, tiempos_direct./reps, 'LineWidth', 2); hold on
semilogy(k_vec, tiempos_fft./reps, 'LineWidth', 2)
legend("dft direct", "fftrecur")
xlabel("K")
ylabel("Tiempo (segundos) (escala logaritmica)")
title("Tiempo de ejecucion de TFD, N=2^K (escala logaritmica)")
        \end{minted}
    \end{solution}
 \end{parts}

\end{questions}
%\tableofcontents
%\listoffigures
%\listoftables
%\listoftodos

%%%%%%%%%%%%%%%%%%%%%%%%%%%%%%%%
%% -- Fin del Documento -- %%
%%%%%%%%%%%%%%%%%%%%%%%%%%%%%%%%
\end{document}