\documentclass[12pt]{exam}
\usepackage{preamble}

% disable para esconder
\usepackage[disable]{easy-todo}

% Comentar para esconder soluciones
\printanswers

%%%%%%%%%%%%%%%%%%%%%%%%%%%%%%%%
%% -- Inicio del Documento -- %%
%%%%%%%%%%%%%%%%%%%%%%%%%%%%%%%%
\begin{document}
\begin{center}
    {\Large
    Ayudantía 1 - Procesamiento Digital de Señales
    }
\end{center}
%\listoftodos

\begin{questions}

\question
Para las siguientes señales, identifique:

\begin{itemize}
    \item Si son continuas, tiempo discreto o digitales.
    \item Tipo de soporte.
    \item Si son periódicas.
\end{itemize}

\begin{parts}
    \part $x_1(t) = \sin(3t)$
    \begin{solution}
        La señal depende del parámetro $t$ y toma valores continuos.
        Por lo tanto, la señal es continua. Además $\sin(3t)$ es periódica. Es de soporte infinito.
    \end{solution}

    \part $x_2[n] = 0.4^n u[n]$
    \begin{solution}
        La señal está definida para $n\in\mathbb{Z}$, no puede ser continua. A medida que $n>0$ crece,
        el valor de la señal se vuelve arbitrariamente pequeño, por lo que no puede ser representada
        en cuantos (requeriría bits infinitos), no puede ser digital; la señal es tiempo discreto.
        
        La señal $x_2[n]$ es mayor a cero para todo $n$ positivo, y $x_2[n]$ es cero para $n$ negativo.
        Es de soporte positivo.
        
        La señal no es periódica.
    \end{solution}

    \part $x_1(t)$, si se muestrea con $T=0.1$
    \begin{solution}
        Al muestrear se obtiene:
        \begin{equation}
        x_3[n]= x_1(nT)|_{T=0.1} = \sin(0.3n),\quad n\in\mathbb{Z}
        \end{equation}
        La señal es ahora de tiempo discreto.

        La señal mantiene su soporte.

        A primera vista puede parecer periódica. Recordemos que esto se cumple si existe un $m$ tal que para cualquier $n$:
        \begin{equation}
            x_3[n] = x_3[n+m]
        \end{equation}
        Al reemplazar por la expresión obtenida:
        \begin{equation}
            \sin(0.3n) = \sin(0.3n+0.3m)
        \end{equation}
        $\sin$ es $2\pi$ periódica, por lo que debe existir $m$ tal que:
        \begin{equation}
            0.3m = 2\pi k,\quad k\in \mathbb{N}
        \end{equation}

        El lado izquierdo de la ecuación es racional, mientras que $\pi$ es irracional, es decir no existen
        $m$ y $k$ que satisfagan la ecuación, por lo tanto, la señal no es periódica.
    \end{solution}

\end{parts}

\question
Se tiene un sistema tiempo discreto cuya respuesta
a impulso es $h[n] =(0.5)^n u[n]$. Usando convolución discreta, calcule la salida del
sistema $y[n]$ si se somete a una entrada $x[n]=u[n] - u[n-4]$.
\begin{solution}
    Conociendo la respuesta a impulso y entrada al sistema, la respuesta viene dada por la expresión:
    \[
        y[n] = x[n]*h[n]
    \]
    Desarrollando mediante la definición de convolución:
    \begin{align}
        y[n] &= \sum_{k=-\infty}^{\infty}x[k]h[n-k]   \\
            &= \sum_{k=-\infty}^{\infty}(u[k]-u[k-4])h[n-k]
    \end{align}
    Notemos que $u[k]-u[k-4]$ es uno para $0\leq k \leq 3$, y cero en cualquier otro caso. Podemos simplificar la serie a una suma finita.
    \begin{align}
        y[n]&= \sum_{k=0}^{3}h[n-k]\\
        &= \sum_{k=0}^{3}(0.5)^{n-k} u[n-k] \\
        &=(0.5)^nu[n] +(0.5)^{n-1}u[n-1]    \\  \nonumber
        &\qquad +(0.5)^{n-2}u[n-2] +(0.5)^{n-3}u[n-3]
    \end{align}
\end{solution}


\question
Calcular la transformada Z, región de convergencia (RC) e identificar polos de las señales a continuación. Bosquejar RC y polos.
\todo{bosquejos}
\begin{parts}
    \part $(-0.8)^n u[n]$, utilizando definición de la transformada Z.
    \begin{solution}
        Al ser de soporte positivo, podemos utilizar la transformada Z unilateral:
        \begin{align}
            \mathcal{Z}\{(-0.8)^nu[n]\} &= \mathcal{Z^+}\{(-0.8)^n\}\\
            &=\sum_{n=0}^{\infty}(-0.8)^n z^{-n}   \\
            &=\sum_{n=0}^{\infty}(-0.8z^{-1})^n
        \end{align}
        Aplicando la identidad $\sum_{n=0}^{\infty}a^n=\frac{1}{1-a}$, obtenemos:
        \begin{align}
            \sum_{n=0}^{\infty}(-0.8z^{-1})^n& = \frac{1}{1+0.8z^{-1}}
        \end{align}
        Para la región de convergencia, $|-0.8z^{-1}|=|0.8z^{-1}|<1$. Es decir $|z|>0.8$. La RC es el exterior de
        una circunferencia de radio $0.8$.

        Existe un único polo cuando $1+0.8z^{-1}=0$. Es decir cuando $z=-0.8$.
        
        Se bosqueja RC y el polo:
        
        \includegraphics[width=10cm]{imagenes/1.3a.png}
    \end{solution}

    \part $\sin(\frac{\pi}{4}n)u[n] + (0.5)^n u[n-3]$, utilizando todos sus conocimientos.
    \begin{solution}
        Para calcular la transformada Z en este caso aplicamos linealidad.
        Primero calculamos $\mathcal{Z}\{\sin(\frac{\pi}{4}n)u[n]\}$.
        Recordemos que en general:
        \begin{equation}
            \mathcal{Z}\{\sin(\omega n)u[n]\}=\frac{z\sin(\omega)}{z^2-2\cos(\omega) z+1}
        \end{equation}
        Con RC $|z|>1$.

        En nuestro caso $\omega = \frac{\pi}{4}$, y $\sin\frac{\pi}{4}=\cos\frac{\pi}{4}=\frac{\sqrt{2}}{2}$.
        Por lo que:
        \begin{equation}
            \mathcal{Z}\{\sin(\frac{\pi}{4}n)u[n]\}=\frac{z\frac{\sqrt{2}}{2}}{z^2-\sqrt{2} z+1}
        \end{equation}

        

        Ahora buscamos $\mathcal{Z}\{(0.5)^n u[n-3]\}$. Sabemos que:
        \begin{equation}
            \mathcal{Z}\{a^n u[n]\} = \frac{1}{1-az^{-1}}
        \end{equation}
        Con RC $|z|>|a|$.
        Podemos aplicar la propiedad de desplazamiento en el tiempo para obtener la transformada Z.
        Sin embargo, para esto debemos preocuparnos que todas las partes de la expresión tengan el desplazamiento:
        \begin{align}
            \mathcal{Z}\{(0.5)^n u[n-3]\} &= \mathcal{Z}\{(0.5)^3(0.5)^{n-3} u[n-3]\}\\
            &= (0.5)^3\mathcal{Z}\{(0.5)^{n-3} u[n-3]\}\\
            &=(0.5)^3z^{-3}\mathcal{Z}\{(0.5)^nu[n]\}\\
            &=\frac{1}{8z^3}\cdot\frac{1}{1-0.5z^{-1}}\\
            &=\frac{1}{8}\cdot\frac{1}{z^2(z-0.5)}
        \end{align}

        Finalmente, sumando ambas partes tenemos la transformada Z completa:
        \begin{equation}
            \frac{z\frac{\sqrt{2}}{2}}{z^2-\sqrt{2} z+1} + \frac{1}{8}\cdot\frac{1}{z^2(z-0.5)}
        \end{equation}
        Con RC $|z|>0.5$
        Para la región de convergencia y polos, consideramos ambas partes. La RC final será la intersección de ambas RC.
        Ambas son el exterior de una circunferencia. La RC será el exterior de la circunferencia más grande: $|z|>1$.

        Los polos vienen dados por: 
        \begin{align}
            z^2-\sqrt{2} z+1=0 &\Rightarrow p_{1,2}=\frac{\sqrt{2}}{2}\pm\frac{\sqrt{2}}{2}j\\
            z-0.5 = 0 &\Rightarrow p_3=0.5  \\
            z^2 &\Rightarrow p_{4,5}= 0
        \end{align}

        Al bosquejar RC y polos queda:

        \includegraphics[width=10cm]{imagenes/1.3b.png}

    \end{solution}
\end{parts}

\question
Un sistema cumple:
\begin{equation}
    y[n]-0.6y[n-1]=x[n],\quad y[-1]=1
\end{equation}
\begin{parts}
    \part Obtener $H(z)$.
    \begin{solution}
        Dado que tenemos una ecuación de diferencias con condición inicial, utilizamos transformada Z unilateral
        para encontrar función de transferencia y respuesta del sistema. La condición inicial
        aparece al aplicar la propiedad de desplazamiento en el tiempo.

        \begin{align}
            &y[n]-0.6y[n-1]=x[n] \\
            &Y(z)-0.6z^{-1}Y(z)-0.6y[-1]=X(z)\\
            \Rightarrow &Y(z)=\underbrace{\frac{z}{z-0.6}}_{H(z)}X(z)+\underbrace{\frac{0.6z}{z-0.6}y[-1]}_{\text{Respuesta natural}}\\
        \end{align}
    \end{solution}

    \part Calcular la respuesta $y[n]$ para entrada escalón usando transformada Z.
    \begin{solution}
        Para calcular la respuesta para entrada escalón, sustituimos la entrada por
        su transformada Z, condición inicial y aplicamos transformada Z.
        \begin{align}
            Y(z)&=\frac{z}{z-0.6}\cdot\frac{z}{z-1}+\frac{0.6z}{z-0.6}(1) \\
                &= z\left[\frac{z}{(z-0.6)(z-1)}\right]+\frac{0.6z}{z-0.6}\\
                &=z\left[\frac{c_1}{z-0.6}+\frac{c_2}{z-1}\right]+\frac{0.6z}{z-0.6}
        \end{align}
        \begin{align}
            &\boxed{c_1 = \left.\frac{z}{z-1}\right|_{z=0.6}=-1.5}\\
            &\boxed{c_2 = \left.\frac{z}{z-0.6}\right|_{z=1}=2.5}
        \end{align}
        \begin{align}
            Y(z) &= -0.9\frac{z}{z-0.6} + 2.5 \frac{z}{z-1}\\
            y[n] &= -0.9(0.6)^nu[n]+2.5u[n], \quad n\geq 0
        \end{align}
    \end{solution}
\end{parts}

\question
Dada la función de transferencia:
\[
    H(z) =\frac{z}{z-0.6}
\]

\begin{parts}
    \part Encontrar respuesta a impulso del sistema.
    \part Analizar causalidad y estabilidad
    \begin{solution}
        Tenemos la función de transferencia, y se reconoce que corresponde a un escalon con un a exponencial.
        Sin embargo no tenemos la RC. La respuesta impulso, causalidad y estabilidad dependerán de la RC.

        \textbf{Caso (i)} RC: $|z|> 0.6$
        Aplicando transformada inversa:
        \begin{equation}
            h(n) = 0.6^nu[n]
        \end{equation}
        El sistema es estable (circunferencia unitaria dentro de la RC) y causal.

        \textbf{Caso (ii)} RC: $|z|< 0.6$
        Aplicando transformada inversa:
        \begin{equation}
            h(n) = -0.6^nu[-n-1]
        \end{equation}
        El sistema es inestable (circunferencia unitaria fuera de la RC) y anticausal.
    \end{solution}
    
    \part ¿Que ocurriría si el polo estuviera en $z=1.2$?
    \begin{solution}
        Suponemos nueva F. de T.:
        \begin{equation}
           H(z) =\frac{z}{z-1.2}
        \end{equation}
        \textbf{Caso (iii)} RC: $|z|> 1.2$
        Aplicando transformada inversa:
        \begin{equation}
            h(n) = 1.2^nu[n]
        \end{equation}
        El sistema es inestable (circunferencia unitaria fuera de la RC) y causal.

        \textbf{Caso (iv)} RC: $|z|< 1.2$
        Aplicando transformada inversa:
        \begin{equation}
            h(n) = -1.2^nu[-n-1]
        \end{equation}
        El sistema es estable (circunferencia unitaria dentro de la RC) y anticausal.

        \textbf{Observación:} Al mover el polo hacia 1.2, la estabilidad cambia, no así la causalidad.
    \end{solution}
\end{parts}

\question
Se tiene un sistema tiempo invariante descrito por las siguientes relaciones:
\begin{align}
    T[k+1] &= a_0T[k]+x[k]^2 \nonumber \\ 
    T_m[k] &=T[k-d]     \nonumber
\end{align}
Donde $x[k]$ es la entrada, y $T_m[k]$ es la salida.
\begin{parts}
    \part ¿Es este un sistema lineal? Si no es así, obtenga un modelo linealizado para una entrada constante 1.
    \begin{solution}
        El sistema no es lineal, debido al término $x[k]^2$.
        Con entrada constante $x_e=1$ y asumiendo estado estacionario, tenemos:
        \begin{equation}
            T[k+1]=T[k]=T_e
        \end{equation}
        Luego:
        \begin{align}
            T[k+1] &= a_0T[k]+x[k]^2 \\
            T_e &= a_0T_e+x_e^2\\
            T_e &= a_0T_e+1\\
            \Rightarrow T_e&=\frac{1}{1-a_0}
        \end{align}
        $T_m$ es simplemente $T$ tras un retardo, su valor en estado estacionario es el mismo.
        \begin{equation}
            T_{me}=\frac{1}{1-a_0}
        \end{equation}
        La variables utilizando el punto de operación quedan decritas como:
        \begin{align}
            x[k] &= x_e + \Delta x[k]\\
            T[k] &= T_e + \Delta T[k]\\
            T_m[k] &= T_{me} + \Delta T_m[k]
        \end{align}

        Con (la primera ecuación se linealiza, la segunda ecuación se toma tal cual pues ya es lineal):
        \begin{align}
            \Delta T[k+1] = c_1\Delta T[k] + c_2\Delta x[k]\\
            \Delta T_m[k] = \Delta T[k-d] 
        \end{align}
        Buscamos los coeficientes $c_1$ y $c_2$. Consideramos la función a partir de la primera
        ecuación de estado:
        \begin{align}
            f(T,x) &= a_0T+x^2\\
            c_1 &= \left.\frac{\partial}{\partial T}f(T,x)\right|_{(T_e,x_e)}=a_0\\
            c_2 &= \left.\frac{\partial}{\partial x}f(T,x)\right|_{(T_e,x_e)}=2x_e =2
        \end{align}
        Por lo que:
        \begin{align}
            \Delta T[k+1] = a_0\Delta T[k] + 2\Delta x[k]\\
            \Delta T_m[k] = \Delta T[k-d] 
        \end{align}
    \end{solution}

    \part Encuentre la función de transferencia del sistema.
    \begin{solution}
        La F. de T. del sistema linealizado se puede expresar como:
        \begin{equation}
            H(z) = \frac{\Delta T_m(z)}{\Delta X(z)} = \frac{\Delta T_m(z)}{\Delta T(z)}\cdot \frac{\Delta T(z)}{\Delta X(z)}
        \end{equation}
        Luego utilizamos transformada Z en el sistema linealizado para encontrar las expresiones que buscamos:
        \begin{align}
            \Delta T[k+1] &= a_0\Delta T[k] + 2\Delta x[k]\\
            z\Delta T(z) &= a_0\Delta T(z) + 2\Delta X(z)\\
            \Rightarrow \frac{\Delta T(z)}{\Delta X(z)} &= \frac{2}{z-a_0}
        \end{align}
            
        \begin{align}
            \Delta T_m[k] &= \Delta T[k-d]\\
            \Delta T_m(z) &= z^{-d}\Delta T(z)\\
            \Rightarrow\frac{\Delta T_m(z)}{\Delta T(z)}&=z^{-d}
        \end{align}

        Finalmente:
        \begin{equation}
            H(z) = \frac{2}{z-a_0}\cdot z^{-d}=\frac{2}{z^d(z-a_0)}
        \end{equation}
    \end{solution}

    \part Encuentre la salida ante una entrada $1+u[k]$ y $d=2$.
    \begin{solution}
        Para la respuesta utilizamos el sistema linealizado condierando el punto de operación. Tenemos para la entrada:
        \begin{align}
            x[k] = x_e+\Delta x[k] &= 1 + u[k]\\
                1+\Delta x[k] &= 1 + u[k]\\
                \Rightarrow \Delta x[k] &= u[k]
        \end{align}
        En cuanto a la salida:
        \begin{align}
            T_m[k] &= T_{me} + \Delta T_m[k]\\
            &= \frac{1}{1-a_0} + \Delta T_m[k]
        \end{align}
        Obtenemos $\Delta T_m[k]$, mediante la función de transferencia con $d=2$.
        \begin{align}
            \Delta T_m(z) &= H(z)\Delta X(z)\\
            &= \frac{2}{z^2(z-a_0)}\cdot \frac{z}{z-1}\\
            &= \frac{2}{z(z-a_0)(z-1)}\\
            &= \frac{c_1}{z} + \frac{c_2}{z-a_0} + \frac{c_3}{z-1}
        \end{align}
        \begin{align}
            &\boxed{c_1 = \left.\frac{2}{(z-a_0)(z-1)}\right|_0 = \frac{2}{a_0}}\\
            &\boxed{c_2 = \left.\frac{2}{z(z-1)}\right|_{a_0} = \frac{2}{a_0^2-a_0}}\\
            &\boxed{c_3 = \left.\frac{2}{z(z-a_0)}\right|_{1} = \frac{2}{1-a_0}}
        \end{align}
        Finalmente, tenemos que:
        \begin{align}
            \Delta T_m(z) &= \frac{2}{a_0}\frac{1}{z} + \frac{2}{a_0^2-a_0}\frac{1}{z-a_0} + \frac{2}{1-a_0}\frac{1}{z-1}\\
            &= z^{-1}\frac{2}{a_0} + z^{-1}\frac{2}{a_0^2-a_0}\frac{z}{z-a_0} + z^{-1}\frac{2}{1-a_0}\frac{z}{z-1}\\
            \Delta T_m[k] &= \frac{2}{a_0}\delta[k-1] + \frac{2}{a_0^2-a_0}a_0^{k-1}u[k-1] + \frac{2}{1-a_0}u[k-1]
        \end{align}
        Y la respuesta completa sería:
        \begin{align}
            T_m[k] &= T_{me} + \Delta T_m[k]\\
                &=\frac{1}{1-a_0} + \frac{2}{a_0}\delta[k-1] + \frac{2}{a_0^2-a_0}a_0^{k-1}u[k-1] + \frac{2}{1-a_0}u[k-1]
        \end{align}
    \end{solution}
\end{parts}

\end{questions}

%%%%%%%%%%%%%%%%%%%%%%%%%%%%%%%%
%% -- Fin del Documento -- %%
%%%%%%%%%%%%%%%%%%%%%%%%%%%%%%%%
\end{document}