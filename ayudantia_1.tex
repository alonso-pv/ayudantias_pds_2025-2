\documentclass[12pt]{exam}
\usepackage{preamble}

% disable para esconder
\usepackage[enable]{easy-todo}

% Comentar para esconder soluciones
\printanswers

%%%%%%%%%%%%%%%%%%%%%%%%%%%%%%%%
%% -- Inicio del Documento -- %%
%%%%%%%%%%%%%%%%%%%%%%%%%%%%%%%%
\begin{document}
\begin{center}
    {\Large
    Ayudantía 1 - Procesamiento Digital de Señales
    }
\end{center}
\listoftodos

\begin{questions}

\question
Para las siguientes señales, identifique:

\begin{itemize}
    \item Si son continuas, tiempo discreto o digitales.
    \item Tipo de soporte.
    \item Si son periódicas.
\end{itemize}

\begin{parts}
    \part $x_1(t) = \sin(3t)$
    \begin{solution}
        La señal depende del parámetro $t$ y toma valores continuos.
        Por lo tanto, la señal es continua. Además $\sin(3t)$ es periódica. Es de soporte infinito.
    \end{solution}

    \part $x_2[n] = 0.4^n u[n]$
    \begin{solution}
        La señal está definida para $n\in\mathbb{Z}$, no puede ser continua. A medida que $n>0$ crece,
        el valor de la señal se vuelve arbitrariamente pequeño, por lo que no puede ser representada
        en cuantos (requeriría bits infinitos), no puede ser digital; la señal es tiempo discreto.
        
        La señal $x_2[n]$ es mayor a cero para todo $n$ positivo, y $x_2[n]$ es cero para $n$ negativo.
        Es de soporte positivo.
        
        La señal no es periódica.
    \end{solution}

    \part $x_1(t)$, si se muestrea con $T=0.1$
    \begin{solution}
        Al muestrear se obtiene:
        \begin{equation}
        x_3[n]= x_1(nT)|_{T=0.1} = \sin(0.3n),\quad n\in\mathbb{Z}
        \end{equation}
        La señal es ahora de tiempo discreto.

        La señal mantiene su soporte.

        A primera vista puede parecer periódica. Recordemos que esto se cumple si existe un $m$ tal que para cualquier $n$:
        \[
            x_3[n] = x_3[n+m]
        \]
        Al reemplazar por la expresión obtenida:
        \[
            \sin(0.3n) = \sin(0.3n+0.3m)
        \]
        $\sin$ es $2\pi$ periódica, por lo que debe existir $m$ tal que:
        \begin{align}
            0.3m &= 2\pi k,\quad k\in \mathbb{N}\\
            \Rightarrow \frac{0.3m}{2k}&=\pi
        \end{align}

        El lado izquierdo de la ecuación es racional, mientras que $\pi$ es irracional, es decir no existen
        $m$ y $k$ que satisfagan la ecuación, por lo tanto, la señal no es periódica.
    \end{solution}

\end{parts}

\question
Se tiene un sistema tiempo discreto cuya respuesta
a impulso es $h[n] =(0.5)^n u[n]$. Usando convolución discreta, calcule la salida del
sistema $y[n]$ si se somete a una entrada $x[n]=u[n] - u[n-4]$.
\begin{solution}
    Conociendo la respuesta a impulso y entrada al sistema, la respuesta viene dada por la expresión:
    \[
        y[n] = x[n]*h[n]
    \]
    Desarrollando mediante la definición de convolución:
    \begin{align}
        y[n] &= \sum_{k=-\infty}^{\infty}x[k]h[n-k]   \\
            &= \sum_{k=-\infty}^{\infty}(u[k]-u[k-4])h[n-k]
    \end{align}
    Notemos que $u[k]-u[k-4]$ es uno para $0\leq k \leq 3$, y cero en cualquier otro caso. Podemos simplificar la serie a una suma finita.
    \begin{align}
        y[n]&= \sum_{k=0}^{3}h[n-k]\\
        &= \sum_{k=0}^{3}(0.5)^{n-k} u[n-k] \\
        &=(0.5)^nu[n] +(0.5)^{n-1}u[n-1]    \\  \nonumber
        &\qquad +(0.5)^{n-2}u[n-2] +(0.5)^{n-3}u[n-3]
    \end{align}
\end{solution}


\question
Calcular la transformada Z, región de convergencia (RC) e identificar polos de las señales a continuación. Bosquejar RC y polos.
\todo{bosquejos}
\begin{parts}
    \part $(-0.8)^n u[n]$, utilizando definición de la transformada Z.
    \begin{solution}
        Al ser de soporte positivo, podemos utilizar la transformada Z unilateral:
        \begin{align}
            \mathcal{Z}\{(-0.8)^nu[n]\} &= \mathcal{Z^+}\{(-0.8)^n\}\\
            &=\sum_{n=0}^{\infty}(-0.8)^n z^{-n}   \\
            &=\sum_{n=0}^{\infty}(-0.8z^{-1})^n
        \end{align}
        Aplicando la identidad $\sum_{n=0}^{\infty}a^n=\frac{1}{1-a}$, obtenemos:
        \begin{align}
            \sum_{n=0}^{\infty}(-0.8z^{-1})^n& = \frac{1}{1+0.8z^{-1}}
        \end{align}
        Para la región de convergencia, $|-0.8z^{-1}|=|0.8z^{-1}|<1$. Es decir $|z|>0.8$. La RC es el exterior de
        una circunferencia de radio $0.8$.

        Existe un único polo cuando $1+0.8z^{-1}=0$. Es decir cuando $z=-0.8$.
    \end{solution}

    \part $\sin(\frac{\pi}{4}n)u[n] + (0.5)^n u[n-3]$, utilizando todos sus conocimientos.
    \begin{solution}
        Para calcular la transformada Z en este caso aplicamos linealidad.
        Primero calculamos $\mathcal{Z}\{\sin(\frac{\pi}{4}n)u[n]\}$.
        Recordemos que en general:
        \begin{equation}
            \mathcal{Z}\{\sin(\omega n)u[n]\}=\frac{z\sin(\omega)}{z^2-2\cos(\omega) z+1}
        \end{equation}
        Con RC $|z|>1$.

        En nuestro caso $\omega = \frac{\pi}{4}$, y $\sin\frac{\pi}{4}=\cos\frac{\pi}{4}=\frac{\sqrt{2}}{2}$.
        Por lo que:
        \begin{equation}
            \mathcal{Z}\{\sin(\frac{\pi}{4}n)u[n]\}=\frac{z\frac{\sqrt{2}}{2}}{z^2-\sqrt{2} z+1}
        \end{equation}

        

        Ahora buscamos $\mathcal{Z}\{(0.5)^n u[n-3]\}$. Sabemos que:
        \begin{equation}
            \mathcal{Z}\{a^n u[n]\} = \frac{1}{1-az^{-1}}
        \end{equation}
        Con RC $|z|>|a|$.
        Podemos aplicar la propiedad de desplazamiento en el tiempo para obtener la transformada Z.
        Sin embargo, para esto debemos preocuparnos que todas las partes de la expresión tengan el desplazamiento:
        \begin{align}
            \mathcal{Z}\{(0.5)^n u[n-3]\} &= \mathcal{Z}\{(0.5)^3(0.5)^{n-3} u[n-3]\}\\
            &= (0.5)^3\mathcal{Z}\{(0.5)^{n-3} u[n-3]\}\\
            &=(0.5)^3z^{-3}\mathcal{Z}\{(0.5)^nu[n]\}\\
            &=\frac{1}{8z^3}\cdot\frac{1}{1-0.5z^{-1}}
        \end{align}

        Finalmente, sumando ambas partes tenemos la transformada Z completa:
        \begin{equation}
            \frac{z\frac{\sqrt{2}}{2}}{z^2-\sqrt{2} z+1} + \frac{1}{8z^3}\cdot\frac{1}{1-0.5z^{-1}}
        \end{equation}
        Con RC $|z|>0.5$
        Para la región de convergencia y polos, consideramos ambas partes. La RC será la intersección de ambas RC.
        Ambas son el exterior de una circunferencia. La RC será el exterior de la circunferencia más grande: $|z|>1$.

        Los polos vienen dados por $z^2-\sqrt{2} z+1=0$ y $1-0.5z^{-1}$. Los polos serían

    \end{solution}
\end{parts}

\question
Un sistema cumple:
\[
    y[n]-0.6y[n-1]=x[n],\quad y[-1]=1
\]
\begin{parts}
    \part Obtener $H(z)$.
    \part Calcular la respuesta $y[n]$ para entrada escalón usando transformada Z.
\end{parts}

\question
Dada la función de transferencia:
\[
    H(z) =\frac{z}{z-0.6}
\]

\begin{parts}
    \part Encontrar respuesta a impulso del sistema.
    \part Analizar causalidad y estabilidad
    \part ¿Que ocurriría si el polo estuviera en $z=1.2$?
\end{parts}
\end{questions}

%%%%%%%%%%%%%%%%%%%%%%%%%%%%%%%%
%% -- Fin del Documento -- %%
%%%%%%%%%%%%%%%%%%%%%%%%%%%%%%%%
\end{document}