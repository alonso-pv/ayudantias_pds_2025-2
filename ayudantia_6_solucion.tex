\documentclass[12pt]{exam}
\usepackage{preamble}
\usepackage{cancel}

% disable para esconder
\usepackage[enable]{easy-todo}

% Comentar para esconder soluciones
\printanswers

%%%%%%%%%%%%%%%%%%%%%%%%%%%%%%%%
%% -- Inicio del Documento -- %%
%%%%%%%%%%%%%%%%%%%%%%%%%%%%%%%%
\begin{document}
\begin{center}
    {\Large
        Ayudantía 6 - Procesamiento Digital de Señales
    }
\end{center}

\begin{questions}

\question Determinar la TFD de $x[n]=5(0.8)^n$, $0\leq n \leq 15$.
\begin{solution}
    Con muestras $n=0,\,1,\,2,\dots,\,15$, tenemos $N=16$ muestras.
    Por definición:
    \begin{align}
        X[k]&=\sum_{n=0}^{N-1}x[n]e^{-j\frac{2\pi}{N}kn}\\
        &=\sum_{n=0}^{15}5(0.8)^ne^{-j\frac{2\pi}{16}kn}\\
        &=5\sum_{n=0}^{15}\left(0.8e^{-j\frac{2\pi}{16}k}\right)^n\\
        &=5\frac{1-0.8e^{-j\frac{2\pi}{\cancel{16}}\cancel{16}k}}{1-0.8e^{-j\frac{2\pi}{16}k}}\\
        &=5\frac{1-0.8}{1-0.8e^{-j\frac{2\pi}{16}k}}\\
        &=5\frac{0.2}{1-0.8e^{-j\frac{2\pi}{16}k}}\\
        &=\frac{1}{1-0.8e^{-j\frac{2\pi}{16}k}},\quad 0\leq k \leq N-1
    \end{align}
\end{solution}

\question Sea $x[n]$ una señal discreta real, con $X[k]$ TFD de $N$ puntos de la
señal. Muestre que:
\begin{parts}
    \part $X[0]$ es real.
    \begin{solution}
        Por definición de la TFD:
        \begin{align}
            X[0]=\sum_{n=0}^{N-1}x[n]e^{-j\frac{2\pi}{N}(0)n}\\
            X[0]=\sum_{n=0}^{N-1}x[n]
        \end{align}
        Como $x[n]$ es real, la suma es real y $X[0]$ es real.
    \end{solution}
    \part $X[N/2]$ es real si $N$ es par.
    \begin{solution}
        Por definición:
        \begin{align}
            X[N/2]&=\sum_{n=0}^{N-1}x[n]e^{-j\frac{2\pi}{N}(\frac{N}{2})n}\\
            &=\sum_{n=0}^{N-1}x[n]e^{-j\pi n}\\
            &=\sum_{n=0}^{N-1}x[n]\Big(\cos(\pi n)-j\sin(\pi n)\Big)\\
            &=\sum_{n=0}^{N-1}x[n]\Big((-1)^n-0\Big)\\
            &=\sum_{n=0}^{N-1}x[n](-1)^n
        \end{align}
        $x[n](-1)^n$ es real. Por lo tanto la suma es real y $X[N/2]$ es real.
    \end{solution}
    \part $X[N-k]=X^*[k]$, con $1 \leq k \leq N-1$
    \begin{solution}
        \begin{align}
            X[N-k]&=\sum_{n=0}^{N-1}x[n]e^{-j\frac{2\pi}{N}(N-k)n}\\
            &=\sum_{n=0}^{N-1}x[n]e^{-j\frac{2\pi}{N}(N)n}e^{-j\frac{2\pi}{N}(-k)n}\\
            &=\sum_{n=0}^{N-1}x[n]\underbrace{e^{-j2\pi n}}_{1}e^{j\frac{2\pi}{N}kn}\\
            &=\sum_{n=0}^{N-1}x[n]e^{j\frac{2\pi}{N}kn}\\
            &=\sum_{n=0}^{N-1}x[n]\left(e^{-j\frac{2\pi}{N}kn}\right)^*\\
            &=\sum_{n=0}^{N-1}\left(x[n]e^{-j\frac{2\pi}{N}kn}\right)^*\quad(x[n] \text{ real})\\
            &=\left(\sum_{n=0}^{N-1}x[n]e^{-j\frac{2\pi}{N}kn}\right)^*\\
            &=X^*[k]
        \end{align}
    \end{solution}
\end{parts}

\question Sea $x_c(t)=5e^{-10t}\sin(20\pi t)u(t)$
\begin{parts}
    \part Determine la TFTC $X_c(j\Omega)$ de la señal.
    \begin{solution}
        \begin{align}
           x_c(t)&=5e^{-10t}\sin(20\pi t)u(t)\\
           &\Big\downarrow\mathcal{L}\nonumber\\
           X_c(s)&=5\left.\frac{20\pi}{s^2+(20\pi)^2}\right|_{s+10}\\
           X_c(s)&=5\frac{20\pi}{(s+10)^2+400\pi^2}\\
           &\Big\downarrow s=j\Omega\nonumber\\
           X_c(j\Omega)&=\frac{100\pi}{(j\Omega+10)^2+400\pi^2}
        \end{align}
    \end{solution}
    \part Muestre como se puede aproximar la TFTC usando la TFD, al muestrear la
    señal con un periodo de muestreo T.
    \begin{solution}
        Por definición:
        \begin{align}
            X_c(j\Omega)&=\int_{-\infty}^{\infty}x_c(t)e^{-j\Omega t}dt\\
            &\Big\downarrow x_c\text{ es de soporte positivo}\nonumber\\
            &=\int_{0}^{\infty}x_c(t)e^{-j\Omega t}dt
        \end{align}
        Muestreando la señal en $t=nT$ podemos aproximar la integral como una
        suma de Riemann. El diferencial $dt$ pasa a ser el periodo de muestreo $T$.
        \begin{equation}
            X_c(j\Omega)\approx\sum_{n=0}^{\infty}x_c(nT)e^{-j\Omega nT}\cdot T\\
        \end{equation}
        Despreciamos los $x_c(nT)$ para valores de $n\geq N$. Convertimos la serie
        en una suma finita.
        \begin{equation}
            X_c(j\Omega)\approx\sum_{n=0}^{N-1}x_c(nT)e^{-j\Omega nT}\cdot T\\
        \end{equation}
        Aún tenemos una suma de funciones continuas en $\Omega$. Muestreando esta
        variable se obtiene una suma discreta de señales discretas. Teniendo en cuenta
        que la frecuencia es periódica (dado que se muestreó en el tiempo), muestreamos
        la frecuencia tal que la exponencial recorra el intervalo $-\pi$ a $\pi$ y se
        obtenga la expresión de la TFD.
        Muestreando en $\Omega=\frac{2\pi}{NT}k$
        \begin{align}
            \left.X_c(j\Omega)\right|_{\Omega=\frac{2\pi}{NT}k}&\approx\sum_{n=0}^{N-1}x_c(nT)e^{-j\frac{2\pi}{NT}k nT}\cdot T\\
            &\approx T\sum_{n=0}^{N-1}x_c(nT)e^{-j\frac{2\pi}{N}k n}=TX[k]
        \end{align}
        Con $X[k]$ TFD de $N$ puntos de $x[n]=x_c(nT)$. La precisión de la aproximación
        depende de la frecuencia de muestreo $1/T$ (entra en juego el teorema de nyquist,
        aliasing), y la cantidad de muestras $N$.
    \end{solution}
    \part Use el comando \texttt{fft} Matlab para calcular la aproximación, y compare
    la aproximación con la expresión analítica en un gráfico.
    \begin{solution}
        Para los gráficos se toma en cuenta la el muestreo que se realiza en la frecuencia.
        Se utiliza la sustitución $\Omega=\frac{2\pi}{NT}k$.

        Tambien hay que notar que al muestrear la señal la frecuencia se hace
        periódica. Por esto, las muestras $k>N/2$ son más bien un reflejo de las
        frecuencias con $k<0$. Para solucionar esto se ocupa \texttt{fftshift} que
        reordena la TFD automáticamente y la centra en torno a $k=0$. Luego para
        graficar se debe ocupar $N/2<k<N/2-1$. 
        \begin{minted}{matlab}
%% 3
% Expresión analítica
X = @(Om) 100*pi./( (1j*Om+10).^2 + 400*pi^2 );

% Señal continua
xc = @(t) 5.*exp(-10*t).*sin(20*pi*t);

% Parámetros para el muestreo
Fs = 200;
T = 1/Fs;
N = 400;
n = 0:N-1;

X_dft = fft(xc(n*T));

X_approx = T * fftshift(X_dft);

% Muestreo de la frecuencia
k = -N/2:N/2-1;
Om_approx = 2*pi/(N*T)*k;

Om_cont = linspace(Om_approx(1), Om_approx(end), 2000);

figure;
subplot(2,1,1)
stem(Om_approx, abs(X_approx)); hold on
plot(Om_cont, abs(X(Om_cont)));
title("Magnitud de X(j\Omega)");
subplot(2,1,2)
stem(Om_approx, angle(X_approx)); hold on
plot(Om_cont, angle(X(Om_cont)));
title("Fase de X(j\Omega)"); 
        \end{minted}

        \includegraphics[width=14cm]{imagenes/6.3c.png}
    \end{solution}
\end{parts}

\question Sea una señal periódica $\tilde{x}_c(t)$, con periodo $T_0=5$ dada por
$\tilde{x}_c(t)=e^{-t}$, $0\leq t \leq 5$.

Los coeficientes de Fourier de la señal vienen dados por:
$$
c_k=\frac{1}{T_0}\int_{0}^{T_0}\tilde{x}_c(t)e^{-jk\frac{2\pi}{T_0}t}
    =\frac{1-e^{-5}}{5+j2\pi k}
$$
\begin{parts}
    \part Muestre como se pueden aproximar los coeficientes usando la TFD, al muestrear
    la señal con un periodo de muestreo $T$.
    \begin{solution}
        De la definición de los coeficientes de Fourier:
        \begin{align}
            c_k=\frac{1}{T_0}\int_{0}^{T_0}\tilde{x}_c(t)e^{-j\frac{2\pi}{T_0}kt}
        \end{align}
        Particionando el intervalo $[0,T_0]$ en $N$ subintervalos de longitud $T=T_0/N$
        (se muestrea la señal en $t=nT$),
        se aproxima la integral como una sumatoria. (Notar que $T_0=NT$)
        \begin{align}
           c_k&\approx\frac{1}{NT}\sum_{n=0}^{N-1}\tilde{x}_c(nT)e^{-j\frac{2\pi}{NT}knT}\cdot T\\
           c_k&\approx\frac{1}{N}\sum_{n=0}^{N-1}\tilde{x}_c(nT)e^{-j\frac{2\pi}{N}kn}=\frac{1}{N}X[k]
        \end{align}
        Con $X[k]$ TFD de $N$ muestras de $x[n]=\tilde{x}_c(nT)$. La aproximación
        mejora a medida que $N$ aumenta (y $T=T_0/N$ disminuye). Notar también, que con $N$
        muestras se obtienen los coeficientes $N$ ($0\leq k \leq N-1$).
    \end{solution}

    \part Aproxime los coeficientes usando el comando \texttt{fft} en Matlab. Compare
    la aproximación con la expresión analítica de los coeficientes en un gráfico.
    \begin{solution}

       \begin{minted}{matlab}
%% 4 Aproximación de coeificientes de fourier mediante TFD
% Expresión analítica
c = @(k) (1-exp(-5))./(5+2j*pi*k);

% Señal continua
xc = @(t) exp(-t); % 0<t<5
T0 = 5;

% Muestreo
N = 100;
T = T0/N;
n = 0:N-1;

% Approximacion de coeficientes
X_dft = fft(xc(n*T));
c_approx = fftshift(X_dft)/N;
k = -N/2:N/2-1;

figure;
subplot(2,1,1)
stem(k, abs(c_approx)); hold on
stem(k, abs(c(k)), "x");
legend("aproximación de c_k", "c_k analítico")
title("Magnitud de c_k")
%xlim([-N/4, N/4])
subplot(2,1,2)
stem(k, angle(c_approx)); hold on
stem(k, angle(c(k)), "x");
legend("aproximación de c_k", "c_k analítico")
title("Fase de c_k");
%xlim([-N/4, N/4]) 
       \end{minted}

       \includegraphics[width=14cm]{imagenes/6.4b.png}
    \end{solution}
\end{parts}

\question Considere la señal real:
$$
x[n]=\begin{cases}
   \cos(0.25\pi n),&0 \leq n \leq 99\\
   0. &\text{en otro caso} 
\end{cases}
$$
\begin{parts}
    \part Determine la TFTD $X(e^{j\omega})$.
    \begin{solution}
        Por definición:
        \begin{align}
            X(e^{j\omega})&=\sum_{n=-\infty}^{\infty}x[n]e^{-j\omega n}\\
            &=\sum_{n=0}^{99}\cos(0.25\pi n)\\
            &=\sum_{n=0}^{99}\frac{1}{2}\left(e^{j0.25\pi n}+e^{-j0.25\pi n}\right)e^{-j\omega n}\\
            &=\frac{1}{2}\left(\sum_{n=0}^{99}e^{j(0.25\pi -\omega)n}+\sum_{n=0}^{99}e^{-j(0.25\pi+\omega) n}\right)\\
            &=\frac{1}{2}\left(\frac{1-e^{j(0.25\pi-\omega)100}}{1-e^{j(0.25\pi-\omega)}}+\frac{1-e^{-j(0.25\pi+\omega)100}}{1-e^{-j(0.25\pi+\omega)}}\right)
        \end{align}
    \end{solution}

    \part Calcule la TFD de $N=100$ puntos de la señal.
    \begin{solution}
        Existe una propiedad para calcular la TFD a partir de la TFTD. Veremos
        como y cuando se puede utilizar.

        Por definición:
        \begin{align}
            X[k]&=\sum_{n=0}^{N-1}x[n]e^{-j\frac{2\pi}{N}kn}\\
            &=\sum_{n=0}^{99}x[n]e^{-j\frac{2\pi}{100}kn}
        \end{align}
        Notemos que la señal $x[n]$ es 0 para $n<0$ y $n>99$. Podemos añadir terminos
        a la suma sin cambiar su valor.
        \begin{align}
            X[k]&=\sum_{n=0}^{99}x[n]e^{-j\frac{2\pi}{100}kn}\\
            &=\sum_{n=-\infty}^{\infty}x[n]e^{-j\frac{2\pi}{100}kn}
        \end{align}
        La serie es sospechosamente similar a la expresión de la TFTD. En efecto,
        en lugar de $\omega$ está $2\pi k/100$. La expresión es un muestreo de
        $X(e^{j\omega})$ en $\omega=2\pi k/100$:
        \begin{align}
            X[k]&=\sum_{n=-\infty}^{\infty}x[n]e^{-j\frac{2\pi}{100}kn}=\left.X(e^{j\omega})\right|_{\omega=\frac{2\pi}{100}k}\\
            &=\frac{1}{2}\left(\frac{1-e^{j(0.25\pi-\omega)100}}{1-e^{j(0.25\pi-\omega)}}+\frac{1-e^{-j(0.25\pi+\omega)100}}{1-e^{-j(0.25\pi+\omega)}}\right)_{\omega=\frac{2\pi}{100}k}\\
            &=\frac{1}{2}\left(\frac{1-e^{j(0.25\pi-\frac{2\pi k}{100})100}}{1-e^{j(0.25\pi-\frac{2\pi k}{100})}}+\frac{1-e^{-j(0.25\pi+\frac{2\pi k}{100})100}}{1-e^{-j(0.25\pi+\frac{2\pi k}{100})}}\right)\\
            &=\frac{1}{2}\left(\frac{1-e^{j(25\pi-2\pi k)}}{1-e^{j(0.25\pi-\frac{2\pi k}{100})}}+\frac{1-e^{-j(25\pi+2\pi k)}}{1-e^{-j(0.25\pi+\frac{2\pi k}{100})}}\right)\\
            &=\frac{1}{2}\left(\frac{2}{1-e^{j(0.25\pi-\frac{2\pi k}{100})}}+\frac{2}{1-e^{-j(0.25\pi+\frac{2\pi k}{100})}}\right)\\
            &=\frac{1}{1-e^{j(0.25\pi-\frac{2\pi k}{100})}}+\frac{1}{1-e^{-j(0.25\pi+\frac{2\pi k}{100})}}\\
        \end{align}

        Esta es una propiedad de la TFD. En resumen:

        Sea $x[n]$, con $X(e^{j\omega})$ su TFTD. Si $x[n]=0$ para $n<0$, $n\geq N$
        entonces la TFD de $N$ puntos viene dado por:
        \begin{equation}
            X[k]=\left.X(e^{j\omega})\right|_{\omega=2\pi k/N}
        \end{equation}

        Esto también aplica en el sentido inverso. Se puede usar la TFD para obtener
        muestras de la TFTD, si se cumple $x[n]=0$ para $n<0$ y $n\geq N$.

        Por otro lado si $x[n]\approx 0$ para los $n$ mencionados, la TFD
        \textbf{aproxima} muestras de la TFTD.
    \end{solution}

    \part Repita con $N=200$.
    \begin{solution}
        Podemos usar la propiedad en el punto anterior.
        \begin{align}
            X[k]=\sum_{n=0}^{N-1}x[n]e^{-j\frac{2\pi}{N}kn}=\sum_{n=0}^{199}x[n]e^{-j\frac{2\pi}{200}kn}
        \end{align}
        Como $x[n]=0$ para $n<0$ y $n\geq200$, podemos concluir que:
        \begin{align}
            X[k]&=\left.X(e^{j\omega})\right|_{\omega=2\pi k/200}\\
            &=\frac{1}{2}\left(\frac{1-e^{j(0.25\pi-\omega)100}}{1-e^{j(0.25\pi-\omega)}}+\frac{1-e^{-j(0.25\pi+\omega)100}}{1-e^{-j(0.25\pi+\omega)}}\right)_{\omega=\frac{2\pi}{200}k}\\
            &=\frac{1}{2}\left(\frac{1-e^{j(0.25\pi-\frac{2\pi k}{200})100}}{1-e^{j(0.25\pi-\frac{2\pi k}{200})}}+\frac{1-e^{-j(0.25\pi+\frac{2\pi k}{200})100}}{1-e^{-j(0.25\pi+\frac{2\pi k}{100})}}\right)\\
            &=\frac{1}{2}\left(\frac{1-e^{j(25\pi-\pi k)}}{1-e^{j(0.25\pi-\frac{2\pi k}{200})}}+\frac{1-e^{-j(25\pi+\pi k)}}{1-e^{-j(0.25\pi+\frac{2\pi k}{100})}}\right)\\
        \end{align}
    \end{solution}

    \part Superponga ambos resultados con la TFTD en Matlab.
    \begin{solution}
        En esta ocasión al calcular la TFD (con $N=200$) con \texttt{fft} se obtienen algunos
        valores de la transformada que deberían ser cero, que por razones numéricas
        se obtienen valores muy cercanos, pero distintos de cero. Para solucionar
        ese problema se aplica una tolerancia sobre el resultado de \texttt{fft}
        para forzar a cero los valores que sean muy cercanos a cero.
        \begin{minted}{matlab}
%% 5 TFTD y TFD comparadas
% Expresión analítica
X_tftd = @(w) ...
 1/2*( 1 - exp(1j*(0.25*pi-w)*100) )./( 1 - exp(1j*(0.25*pi-w)) ) ...
 + 1/2*( 1 - exp(-1j*(0.25*pi+w)*100) )./( 1 - exp(-1j*(0.25*pi+w)) );

% Señal discreta
x = @(n) cos(0.25*pi*n).*(n>=0 & n<=99);

% TFD
N=100;
%N=200;
X_tfd = fft(x(0:N-1));

% Tolerancia
% Fases erróneas en valores que deberían ser cero.
tol = 10^-6;
X_tfd(X_tfd<tol) = 0;

w = linspace(0, 2*pi, 1000);

% Muestreo de la frecuencia
k = 0:N-1;
w_discr = 2*pi*k/N;

figure
subplot(2,1,1)
plot(w, abs(X_tftd(w))); hold on;
stem(w_discr, abs(X_tfd));
title("Magnitud de X(e^{j\omega})")
subplot(2,1,2)
plot(w, angle(X_tftd(w))); hold on;
stem(w_discr, angle(X_tfd));
title("Fase de X(e^{j\omega})")
        \end{minted}
        Con $N=100$:\\
        \includegraphics[width=14cm]{imagenes/6.5d.png}\pagebreak

        Con $N=200$:\\
        \includegraphics[width=14cm]{imagenes/6.5d2.png}
    \end{solution}
\end{parts}

\end{questions}
%\tableofcontents
%\listoffigures
%\listoftables
%\listoftodos

%%%%%%%%%%%%%%%%%%%%%%%%%%%%%%%%
%% -- Fin del Documento -- %%
%%%%%%%%%%%%%%%%%%%%%%%%%%%%%%%%
\end{document}