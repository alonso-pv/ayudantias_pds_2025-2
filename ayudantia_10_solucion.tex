\documentclass[12pt]{exam}
\usepackage{preamble}

% disable para esconder
\usepackage[enable]{easy-todo}

% Comentar para esconder soluciones
\printanswers

%%%%%%%%%%%%%%%%%%%%%%%%%%%%%%%%
%% -- Inicio del Documento -- %%
%%%%%%%%%%%%%%%%%%%%%%%%%%%%%%%%
\begin{document}
\begin{center}
    {\Large
        Ayudantía 10 - Procesamiento Digital de Señales
    }
\end{center}

\begin{questions}
\question Considere el filtro FIR con respuesta a impulso $h[n]=u[n]-u[n-4]$.
\begin{parts}
    \part Determine el tipo de filtro FIR (I, II, III o IV)
    \begin{solution}
        La respuesta a impulso puede ser reescrite como $h[n]=\{1,\;1,\;1,\;1\}$.
        Dado que es simétrica ($h[n]=h[M-n]$) y de orden par ($M=3$), se trata de
        un filtro FIR tipo II. 
    \end{solution}

    \part Determine y bosqueje la respuesta de magnitud $|H(e^{j\omega})|$
    \begin{solution}
        De la respuesta a impulso:
            \begin{align}
                h[n]&=u[n]-u[n-4]\\
                &\downarrow\mathcal{Z}\notag\\ 
                H(z)&=\frac{1}{1-z^{-1}}-z^{-4}\frac{1}{1-z^{-1}}\\
                &=\frac{1-z^{-4}}{1-z^{-1}}\\
                &\downarrow z=e^{j\omega}\notag\\ 
                H(e^{j\omega})&=\frac{1-e^{-4j\omega}}{1-e^{-j\omega}}
            \end{align}
            Bosquejar la magnitud a partir de esta expresión es complicado. Factorizando
            numerador y denominador por exponenciales complejas:
            \begin{align}
                H(e^{j\omega})&=\frac{1-e^{-4j\omega}}{1-e^{-j\omega}}\\
                &=\frac{e^{2j\omega}-e^{-2j\omega}}{e^{j\omega/2}-e^{-j\omega/2}}\left(\frac{e^{-2j\omega}}{e^{-j\omega/2}}\right)
            \end{align} 
            Usando $e^{j\theta}-e^{-j\theta}=2j\sin(\theta)$:
            \begin{align}
                H(e^{j\omega})&=\frac{e^{2j\omega}-e^{-2j\omega}}{e^{j\omega/2}-e^{-j\omega/2}}\left(\frac{e^{-2j\omega}}{e^{-j\omega/2}}\right)\\
                &=\frac{2j\sin(2\omega)}{2j\sin(\omega/2)}e^{-\frac{3}{2}j\omega}\\
                &=\frac{\sin(2\omega)}{\sin(\omega/2)}e^{-\frac{3}{2}j\omega}\\
                &\downarrow |\cdot|\notag\\
                |H(e^{j\omega})|&=\left|\frac{\sin(2\omega)}{\sin(\omega/2)}\right|\left|e^{-\frac{3}{2}j\omega}\right|\\
                &=\left|\frac{\sin(2\omega)}{\sin(\omega/2)}\right|
            \end{align}
            Evaluando la expresión se obtienen valores para el bosquejo (para el caso $\omega=0$ se
            usa el límite):
            \begin{flalign}
                \omega&=0 & \omega&=\pm\pi/2 & \omega&=\pm\pi
                \\
                \left|\frac{\sin(2\omega)}{\sin(\omega/2)}\right|&\approx\left|\frac{2\omega}{\omega/2}\right|
                & \left|\frac{\sin(2\omega)}{\sin(\omega/2)}\right|&=\left|\frac{\sin(\pm\pi)}{\sin(\pm\pi/4)}\right|
                & \left|\frac{\sin(2\omega)}{\sin(\omega/2)}\right|&=\left|\frac{\sin(\pm2\pi)}{\sin(\pm\pi/2)}\right|
                \\
                &=|4|
                & &=\left|\frac{0}{\sin(\pm\pi/4)}\right|
                & &=\left|\frac{0}{\sin(\pm\pi/2)}\right|
                \\
                &=4
                & &=0
                & &=0
            \end{flalign}
            La respuesta de magnitud tiene un lóbulo principal entre $[-\pi/2,\;\pi/2]$, y
            dos lóbulos más pequeños entre $[-\pi,\;-\pi/2]$, y $[\pi/2,\;\pi$]:\\
            \includegraphics[width=14cm]{imagenes/10.1b.png}
    \end{solution}

    \part Determine y bosqueje la respuesta de amplitud $A(e^{j\omega})$
    \begin{solution}
        A partir de la expresión de la respuesta en frecuencia hallada en el punto
        anterior, se deduce la respuesta de amplitud $A(e^{j\omega})$:
        \begin{gather}
                H(e^{j\omega})=\frac{\sin(2\omega)}{\sin(\omega/2)}e^{-\frac{3}{2}j\omega}=A(e^{j\omega})e^{j\Psi(\omega)}\\
                \downarrow\notag\\
                A(e^{j\omega})=\frac{\sin(2\omega)}{\sin(\omega/2)}
        \end{gather}
        Como $|A(e^j{\omega})|=|H(e^{j\omega})|$, los ceros que se vieron en el punto anterior
        se vuelven a utilizar para el bosquejo. Para dibujar los lóbulos se debe tener en cuenta
        el signo de la expresión:
        \begin{flalign}
            &-\pi<\omega<-\pi/2 & &-\pi/2<\omega<0 & &0<\omega<\pi/2 & &\pi/2<\omega<\pi\\
            %&\sin(2\omega)>0 & &\sin(2\omega)<0 & &\sin(2\omega)>0 & &\sin(2\omega)<0\\
            %&\sin(\omega/2)<0 & &\sin(\omega/2)<0 & &\sin(\omega/2)>0 & &\sin(\omega/2)>0\\
            &\frac{\sin(2\omega)}{\sin(\omega/2)}<0 & &\frac{\sin(2\omega)}{\sin(\omega/2)}>0 & &\frac{\sin(2\omega)}{\sin(\omega/2)}>0 & &\frac{\sin(2\omega)}{\sin(\omega/2)}<0 
        \end{flalign}
        \includegraphics[width=14cm]{imagenes/10.1c.png}
    \end{solution}

    \part Determine y bosqueje la respuesta de fase $\angle H(e^{j\omega})$
    \begin{solution}
        Tenemos que:
        \begin{align}
            \angle H(e^{j\omega})&=\angle\left(\frac{\sin(2\omega)}{\sin(\omega/2)}e^{-\frac{3}{2}j\omega}\right)\\
            &=\angle\left(\frac{\sin(2\omega)}{\sin(\omega/2)}\right) -\frac{3}{2}\omega
        \end{align}
        Cuando $\frac{\sin(2\omega)}{\sin(\omega/2)}<0$, su fase toma el valor $\pm\pi$:
        \begin{equation}
            \angle H(e^{j\omega})=\begin{cases}
                -\pi-\frac{3}{2}\omega & -\pi<\omega<-\frac{\pi}{2}\\
                -\frac{3}{2}\omega & -\frac{\pi}{2}<\omega<\frac{\pi}{2}\\
                \pi-\frac{3}{2}\omega & \frac{\pi}{2}<\omega<\pi\\
            \end{cases}
        \end{equation}
        \includegraphics[width=14cm]{imagenes/10.1d.png}
    \end{solution}

    \part Determine y bosqueje la respuesta de ángulo $\Psi(\omega)$
    \begin{solution}
        La respuesta de ángulo $\Psi(\omega)$ se deduce de $H(e^{j\omega})$:
        \begin{gather}
                H(e^{j\omega})=\frac{\sin(2\omega)}{\sin(\omega/2)}e^{-\frac{3}{2}j\omega}=A(e^{j\omega})e^{j\Psi(\omega)}\\
                \downarrow\notag\\
                \Psi(\omega)=-\frac{3}{2}\omega
        \end{gather}
        \includegraphics[width=14cm]{imagenes/10.1e.png}
    \end{solution}
\end{parts}

\question Se requiere un filtro digital pasa-alto para procesar una señal de sonido:
\begin{itemize}
    \item La señal se muestrea a 48 kHz.
    \item Frecuencia borde de la banda de rechazo: $F_s=10.8$ kHz
    \item Frecuencia borde de la banda de paso: $F_p=12$ kHz
    \item Atenuación de banda de rechazo $A_s=42$ dB
    \item Ripple en banda de paso $A_p=0.2$ dB
\end{itemize}
    \includegraphics[width=15cm]{imagenes/9.p3.png}\\

\begin{parts}
    \part Determine las especificaciones absolutas $\delta_s$ y $\delta_p$, y las
    frecuencias normalizadas $\omega_s$ y $\omega_p$ para el diseño del filtro.\label{preg:2a}
    \begin{solution}
        Para convertir de especificaciones relativas a absolutas usamos:
        \begin{align}
            \delta_p&=\frac{10^{A_p/20}-1}{10^{A_p/20}+1} &
            \delta_s&=\frac{1+\delta_p}{10^{A_s/20}}
        \end{align}

        Reemplazando por los datos se obtiene:
        \begin{align}
            \delta_p&=\frac{10^{0.2/20}-1}{10^{0.2/20}+1} &
            \delta_s&=\frac{1+0.0115}{10^{42/20}}\\
            &=0.0115 & &=0.0080
        \end{align}

        Se convierte de las frecuencias en Hz, a frecuencia normalizada usando $\omega=2\pi\frac{F}{F_m}$:
        \begin{align}
            \omega_s&=2\pi\frac{F_s}{F_m} & \omega_p&=2\pi\frac{F_p}{F_m}\\
            &=2\pi\frac{10.8}{48} & &=2\pi\frac{12}{48}\\
            &=0.45\pi & &=0.5\pi
        \end{align}
    \end{solution}

    \part Determine la respuesta a impulso ideal para el diseño de un filtro FIR
    por medio de ventana.
    \begin{solution}
        \paragraph{Filtro pasa-bajo}
        Recordando que
        \begin{align}
            &H_{LP}(e^{j\omega})=\begin{cases}
                e^{-j\alpha\omega} & |\omega| < \omega_c\\
                0 & \omega_c < |\omega| <\pi
            \end{cases} & &\longleftrightarrow &
            &h_{LP}[n]=\frac{\sin(\omega_c[n-\alpha])}{\pi[n-\alpha]}
        \end{align}
        Es la respuesta de un filtro pasa-bajo ideal y su respuesta a impulso. Se deducen las
        las respuestas a impulso de otros tipos de filtro.
        \paragraph{Filtro pasa-banda}
        Un filtro pasa banda ideal con frecuencias de corte $\omega_{c1}$ y $\omega_{c2}$, se
        obtiene restando dos pasabajos:
        \begin{align}
            &H_{BP}(e^{j\omega})=\begin{cases}
                0 & |\omega| <\omega_{c1}\\
                e^{-j\alpha\omega} & \omega_{c1} < |\omega| < \omega_{c2}\\
                0 & \omega_{c2} < |\omega| <\pi
            \end{cases} & &\longleftrightarrow &
            &\begin{split}
                h_{BP}[n]=\frac{\sin(\omega_{c2}[n-\alpha])}{\pi[n-\alpha]}\\
                -\frac{\sin(\omega_{c1}[n-\alpha])}{\pi[n-\alpha]}
            \end{split}
        \end{align}
        
        \paragraph{Filtro pasa-alto} Un filtro pasa-alto con frecuencia de corte $\omega_c$
        se obtiene restando un pasa-bajos de un impulso:
         \begin{align}
            &H_{HP}(e^{j\omega})=\begin{cases}
                0 & |\omega| < \omega_c\\
                e^{-j\alpha\omega} &  \omega_c < |\omega|
            \end{cases} & &\longleftrightarrow &
            &h_{HP}[n]=\delta[n-\alpha]-\frac{\sin(\omega_c[n-\alpha])}{\pi[n-\alpha]}
        \end{align}

        Para el diseño del filtro, escogemos la frecuencia de corte $\omega_c$
        en el punto medio de la banda de transición $[\omega_s,\;\omega_p]$:
        \begin{equation}
            \omega_c = \frac{\omega_s+\omega_p}{2}=\frac{0.45\pi+0.5\pi}{2}
            =0.475\pi
        \end{equation}
        La respuesta a impulso ideal a utilizar para el filtro FIR de orden $M$
        (por definir) es:
        \begin{align}
            h_{HP}[n]=\delta[n-M/2]-\frac{\sin(0.475\pi[n-M/2])}{\pi[n-M/2]}
        \end{align}
    \end{solution}

    \part Escoja una ventana apropiada de la tabla y utilice esta para determinar el filtro
    FIR en Matlab.
    \begin{solution}
        \paragraph{Selección de ventana}
        Considerando las especificaciones calculadas en (\ref{preg:2a}):
        \begin{align*}
            &\delta_s=0.0080 &
            &\delta_p=0.0115 &
            &\omega_s=0.45 &
            &\omega_p=0.5
        \end{align*}
        Obtenemos un ripple mínimo y lo convertimos a atenuación en dB para la
        elección de la ventana:
        \begin{align}
            \delta &=\min\{\delta_s,\; \delta_p\} &
            A&=-20\log(0.0080)\\
            &=0.0080 &
            &=42\text{ dB}
        \end{align}
        De la columna ``$A_s$ (\textbf{dB})'' de la tabla escogemos la ventana.
        Con $A=42 < 44$ dB, se escoge la ventana de Hann.

        \paragraph{Selección de orden $M$} Para hallar el orden $M=L-1$ se iguala la
        banda de transición requerida ($\Delta\omega=\omega_p-\omega_s=0.05\pi$)
        con la banda producida por la ventana escogida (columna \textbf{Exact}
        $\Delta\omega$ de la tabla). Para la ventana Hann.
        
        Así, tenemos:
        \begin{align}
            \Delta\omega=0.05\pi=\frac{6.2\pi}{L} \longrightarrow M=L-1=123 
        \end{align}

        \paragraph{Ajuste de $M$} Al tener $M=123$ \textbf{impar}, aplicar la
        ventana produce un filtro \textbf{FIR tipo II}. Este tipo de filtro tiene
        la restricción $\left.H(e^{j\omega})\right|_{\omega=\pi}=0$. Esto es malo
        considerando que diseñamos un pasa-alto.

        Por esta razón, aumentamos en 1 el orden para obtener $M=124$ \textbf{par},
        que produce un filtro \textbf{FIR tipo I}, que no tiene esta restricción.

        \paragraph{Respuesta a impulso final} Considerando lo anterior, la
        respuesta a impulso del filtro FIR queda como:
        \begin{align}
            h[n]&=h_{HP}[n]\cdot w_{\text{Hann}}[n]    \\
            &=\left(\delta[n-62]-\frac{\sin(0.475\pi[n-62])}{\pi[n-62]}\right)
            w_{\text{Hann}}[n]
        \end{align}
        Con\begin{equation}
            w_{\text{Hann}}[n]=\begin{cases}
                0.5-0.5\cos\left(2\pi n/M\right) & 0\leq n \leq M=124 \\
                0 & \text{otro valor}
            \end{cases}
        \end{equation}

        Esto se puede realizar de forma manual en Matlab, o bien, utilizando la
        función \texttt{fir1} de Matlab:
        \begin{minted}[fontsize=\footnotesize]{matlab}
M = 124;
wc = 0.475*pi;

% Forma manual
n=0:M;
h_ideal = -sin(wc*(n-M/2))./(pi* (n-M/2));
%h_ideal = sin(pi*(n-M/2))./(pi* (n-M/2))-sin(wc*(n-M/2))./(pi* (n-M/2));
h_ideal(isnan(h_ideal)) = -wc/pi;
h_ideal(M/2+1) = h_ideal(M/2+1) + 1; % M/2+1 porque matlab indexa desde uno.

h_HP = h_ideal.*hann(M+1)';

% Usando fir1
h_fir1 = fir1(M,wc/pi,"high",hann(M+1));
        \end{minted}

        Al graficar respuesta a impulso y magnitud:\\
        \includegraphics[width=14cm]{imagenes/10.2c.png}
    \end{solution}

    \part Compruebe el filtro en Matlab con una señal de prueba
    $\cos(2\pi F_1 t) + \sin(2\pi F_2 t)$. (Con $F_1=16 \text{kHz}$, $F_2=8$ kHz)
    \begin{solution}
        Al testear el filtro con la señal de prueba, vemos como la componente de $8$ kHz
        se reduce en amplitud, siendo imperceptible en la señal filtrada. La otra
        componente se mantiene prácticamente intacta.

        Para la señal original y filtrada, se grafica la forma de onda y transformada
        de Fourier (aproximada):\\
        \includegraphics[width=14cm]{imagenes/10.2d.png}
        \begin{minted}[fontsize=\footnotesize]{matlab}
%% 2 (d)

F1 = 16000; F2 = 8000;
% Señal continua
x_cont = @(t) cos(2*pi*F1*t) + sin(2*pi*F2*t);

% Muestreo
Fm = 48000; % Frec. muestreo
T = 1/Fm;
nT = 0:T:0.1;
x_muestreada = x_cont(nT);

x_filtrada = filter(h_HP, 1, x_muestreada);

% Aproximacion de TFTC
f_m = abs(fftshift(fft(x_muestreada)*T));
f_f = abs(fftshift(fft(x_filtrada)*T));

% % Ploteamos una porción de la señal
n = 0:1000;
x_muestreada = x_muestreada(n+1);
x_filtrada = x_filtrada(n+1);

x_m_dac = interp(x_muestreada, 10);
x_f_dac = interp(x_filtrada, 10);
n_dac = (0:length(x_m_dac)-1) / 10; 

% ploteo fourier
N = length(f_m);
F = (-N/2:N/2-1)*(1/(N*T));

figure;
% Ploteo de la señal filtrada
subplot(2,2,1)
stem(n*T, x_muestreada, 'filled', 'LineWidth', 2); hold on
plot(n_dac*T, x_m_dac, '--', Linewidth=1.5)
xlim([0.01 0.0103])
xlabel("Tiempo (s)")
ylabel("Amplitud")
title("Señal original")

subplot(2,2,2)
plot(F, f_m, LineWidth=2)
title("Transformada de Fourier (aprox.) de la señal original")
xlabel("F")
ylabel("X(2{\pi}F)")
xticks([-16 -8 8 16]*1000)
xticklabels(["-16 kHz" "-8 kHz" "8 kHz" "16 kHz"])

subplot(2,2,3)
stem(n*T, x_filtrada,'filled', LineWidth=2); hold on
plot(n_dac*T, x_f_dac, '--', Linewidth=1.5)
xlim([0.01 0.0103])
xlabel("Tiempo (s)")
ylabel("Amplitud")
title('Señal filtrada')

subplot(2,2,4)
plot(F, f_f, LineWidth=2)
title("Transformada de Fourier (aprox.) de la señal filtrada")
xlabel("F")
ylabel("X(2{\pi}F)")
xticks([-16 -8 8 16]*1000)
xticklabels(["-16 kHz" "-8 kHz" "8 kHz" "16 kHz"])
        \end{minted}
    \end{solution}
\end{parts}




\end{questions}
%\tableofcontents
%\listoffigures
%\listoftables
%\listoftodos

%%%%%%%%%%%%%%%%%%%%%%%%%%%%%%%%
%% -- Fin del Documento -- %%
%%%%%%%%%%%%%%%%%%%%%%%%%%%%%%%%
\end{document}