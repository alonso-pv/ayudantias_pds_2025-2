\documentclass[12pt]{exam}
\usepackage{preamble}

% disable para esconder
\usepackage[enable]{easy-todo}

% Comentar para esconder soluciones
\printanswers

%%%%%%%%%%%%%%%%%%%%%%%%%%%%%%%%
%% -- Inicio del Documento -- %%
%%%%%%%%%%%%%%%%%%%%%%%%%%%%%%%%
\begin{document}
\begin{center}
    {\Large
        Listado 0 - Procesamiento Digital de Señales
    }
\end{center}

\begin{questions}
\question
Esta es una pregunta. Calcular $\frac{d}{dx}e^{2x}$.

\begin{solution}
    Esta es la solución.
    \begin{align}
        \frac{d}{dx}e^{2x} &=e^{2x}\cdot\frac{d}{dx}2x\\
        &=2e^{2x}
    \end{align}
\end{solution}

\question
Esta es otra pregunta. Escriba un script en Matlab que sume los números del 1 al 100.
Entregue el resultado.

\begin{solution}
    \begin{minted}{matlab}
    suma = 0;
    for n = 1:100
        suma = suma + num;
    end
    print(suma)
    \end{minted}
    El resultado es 5050.
\end{solution}

\begin{solution}
    Esta es una solución alternativa.
    \begin{minted}{matlab}
    sum(1:100)
    \end{minted}
    El resultado es 5050.
\end{solution}

\question
Dibuje la función $x^3+2$ en el plano cartesiano.

\begin{solution} 
    La figura se dibuja a continuación.
    
    \begin{tikzpicture}
        \draw[<->] (-2,0) -- (2,0);
        \draw[<->] (0,-1) -- (0,3);
        \draw[black, thick, domain=-1.4:1] plot (\x, {\x*\x*\x+2});
    \end{tikzpicture}
\end{solution}


\end{questions}
%\tableofcontents
%\listoffigures
%\listoftables
%\listoftodos

%%%%%%%%%%%%%%%%%%%%%%%%%%%%%%%%
%% -- Fin del Documento -- %%
%%%%%%%%%%%%%%%%%%%%%%%%%%%%%%%%
\end{document}