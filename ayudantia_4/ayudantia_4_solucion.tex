\documentclass[12pt]{exam}
\usepackage{../preamble}

% disable para esconder
\usepackage[enable]{easy-todo}

% Comentar para esconder soluciones
\printanswers

%%%%%%%%%%%%%%%%%%%%%%%%%%%%%%%%
%% -- Inicio del Documento -- %%
%%%%%%%%%%%%%%%%%%%%%%%%%%%%%%%%
\begin{document}
\begin{center}
    {\Large
        Ayudantía 4 - Procesamiento Digital de Señales
    }
\end{center}

% Pregunta identificar filtros

% Diseñar un filtro simple IIR
%a) diseñar para cierta característica
%b) bosquejar fase y magnitud
%c) Encontrar respuesta para cierta señal periodica
%d) Encontrar ecuación de diferencias

% Diseñar un pasabanda simple.
%a) diseñar con ciertos requerimientos.

\begin{questions}

\question
Considere el sistema LTI causal descrito por $y[n]=0.8y[n-1]+0.2x[n]$.
\begin{parts}
    \part Determine $H(z)$ y $H(e^{j\omega})$
    \begin{solution}
        \begin{align}
            y[n]&=0.8y[n-1]+0.2x[n] \\
            &\Big\downarrow \mathcal{Z}\nonumber\\
            Y(z)&=0.8z^{-1}Y(z)+0.2X(z)\\
            \Rightarrow H(z)&=\frac{0.2}{1-0.8z^{-1}}\\
            &\Big\downarrow z=e^{j\omega}\nonumber\\
            H(e^{j\omega})&=\frac{0.2}{1-0.8e^{-j\omega}}=\frac{0.2}{1-0.8\cos\omega+j0.8\sin\omega}
        \end{align}
    \end{solution}

    \part Obtenga $|H(e^{j\omega})|$ y $\angle H(e^{j\omega})$
    \begin{solution}
        \begin{align}
            |H(e^{j\omega})| &= \frac{0.2}{|1-0.8\cos\omega+j0.8\sin\omega|}\\
            &= \frac{0.2}{\sqrt{(1-0.8\cos\omega)^2+(0.8\sin\omega)^2}}\\
            &= \frac{0.2}{\sqrt{1.64-1.6\cos\omega}}
        \end{align}

        \begin{align}
            \angle H(e^{j\omega})&=-\angle(1-0.8\cos\omega+j0.8\sin\omega)\\
            &=-\arctan\left(\frac{0.8\sin\omega}{1-0.8\cos\omega}\right)
        \end{align}
    \end{solution}

    \part Halle la ganancia DC y en $\omega=\pi$. Identificar el tipo de filtro
    y hallar la frecuencia de corte.
    \begin{solution}
        Para la ganancia DC, ocupamos la TFTD evaluada en $\omega = 0$:
        \begin{align}
            G_{DC}=H(e^{j(0)})&=\frac{0.2}{1-0.8}\\
            &=\frac{0.2}{0.2}=1
        \end{align}

        La magnitud en $\omega=\pi$:
        \begin{align}
            |H(e^{j(\pi)})|&=\frac{0.2}{\sqrt{1.64-1.6\cos\pi}}\\
            &=\frac{0.2}{\sqrt{3.24}}=\frac{0.2}{1.8}\\
            &=1/9
        \end{align}

        Para encontrar la frecuencia de corte $\omega_c$, consideramos la ecuación:
        \begin{align}
            |H(e^{j\omega_c})|=\frac{0.2}{\sqrt{1.64-1.6\cos\omega_c}}&=\frac{1}{\sqrt{2}}=-3_{\text{dB}}\\
            \frac{1}{\sqrt{1.64-1.6\cos\omega_c}}&=\frac{1}{\sqrt{0.08}}\\
            1.64-1.6\cos\omega_c&=0.08\\
            \omega_c&=\cos^{-1}\frac{1.56}{1.6}\\
            &=0.2241
        \end{align}
    \end{solution}

    \part Bosqueje la magnitud.
    \begin{solution}
        La magnitud para sistemas con respuesta a impulso reales es par. El bosquejo es simétrico.
        Se realiza el bosquejo entre $-\pi$ y $\pi$, pues la magnitud es periódica.
    
        \includegraphics[width=10cm]{imagenes/4.1d.png}
    \end{solution}
    \part Encuentre el retardo de fase y el retardo de grupo. ¿Que se puede concluir?
    \begin{solution}
        El retardo de fase es por definición:
        \begin{equation}
            \tau_{\text{pd}}(\omega)=-\frac{\angle H(e^{j\omega})}{\omega}=\frac{1}{\omega}\arctan\left(\frac{0.8\sin\omega}{1-0.8\cos\omega}\right)
        \end{equation}
        Para la respuesta continua de fase consideramos $\Psi(\omega)=\angle H(e^{j\omega})$
        pues la fase $\angle H(e^{j\omega})$ es en este caso continua. Por lo tanto,
        el retardo de grupo es por definición:
        \begin{align}
            \tau_{\text{gd}}(\omega)=-\frac{d\Psi(\omega)}{d\omega}&=\frac{d}{d\omega}\arctan\left(\frac{0.8\sin\omega}{1-0.8\cos\omega}\right)\\
            &=\frac{0.8\cos\omega-0.64}{1.64-1.6\cos\omega}
        \end{align}
        Es evidente que ni el retardo de fase $\tau_{\text{pd}}$ ni el retardo de grupo $\tau_{\text{gd}}$
        son constantes. Como no son constantes, se concluye que el filtro distorsiona
        la señal de entrada.
    \end{solution}

    \part Considere la entrada $x[n]=3+5\cos(\frac{\pi}{4}n)$. Determine la salida
    $y[n]$.
    \begin{solution}
        Para la salida, aplicamos la propiedad de sistemas LTI para señales periódicas.
        Para la constante utilizamos la ganancia DC ($G_{\text{DC}}=1$), y para el coseno ($\omega=\pi/4$)
        consideramos:
        \begin{align}
            |H(e^{j\pi/4})|&=\frac{0.2}{\sqrt{1.64-1.6\cos(\pi/4)}}&  \angle H(e^{j\pi/4})&=-\arctan\left(\frac{0.8\sin(\pi/4)}{1-0.8\cos(\pi/4)}\right)  \\
            &=0.2804&    &=-0.9160
        \end{align}
        Utilizando estos valores, se deduce la salida del sistema:
        \begin{align}
            x[n]&=3+5\cos\left(\frac{\pi}{4}n\right)\\
            &\Big\downarrow\mathcal{H}\nonumber\\
            y[n]&=3G_{\text{DC}} + 5|H(e^{j\omega})|\cos\left(\frac{\pi}{4}n+\angle H(e^{j\omega})\right)\\
            &=3(1) + 5(0.2804)\cos\left(\frac{\pi}{4}n-0.9160\right)\\
            &=3+1.402\cos\left(\frac{\pi}{4}n-0.9160\right)
        \end{align}
    \end{solution}

\end{parts}

\question Un filtro se define por la siguiente ecuación:
$$
    y[n] = \frac{1}{5}\sum_{k=0}^{4}x[n-k]
$$

\begin{parts}
    \part Encuentre la respuesta a impulso $h[n]$ e identifique el tipo de filtro.
    \begin{solution}
        Puesto que ya tenemos una expresión explícita para la salida, podemos encontrar
        la respuesta a impulso reemplazando $x[n]=\delta[n]$.
        \begin{align}
            y[n] &= \frac{1}{5}\sum_{k=0}^{4}x[n-k]\\
            &\Big\downarrow x[n]=\delta[n]\nonumber\\
            h[n] &= \frac{1}{5}\sum_{k=0}^{4}\delta [n-k]\\
            &=\frac{1}{5}(\delta[n]+\delta[n-1]+ \dots +\delta[n-4])\\
            &=\frac{1}{5}(u[n]-u[n-5])
        \end{align}
    \end{solution}

    \part Demuestre que la respuesta en frecuencia se puede escribir como:
    $H(e^{j\omega}) = \frac{1}{5}\frac{\sin (5\omega/2)}{\sin(\omega/2)}e^{-j2\omega}$
    \begin{solution}
        Aplicando transformada Z sobre la respuesta a impulso, y evaluando en $e^{j\omega}$
        se obtiene la respuesta en frecuencia. 
        
        Recordar que: $\mathcal{Z}\{\delta[n]\}=1$,
         $\sum_{k=0}^{N}a^k=\frac{1-a^{N+1}}{1-a}$,
         $2j\sin\omega=e^{j\omega}-e^{-j\omega}$
        \begin{align}
            h[n] &= \frac{1}{5}\sum_{k=0}^{4}\delta [n-k]\\
            &\Big\downarrow \mathcal{Z}\nonumber\\
            H(z) &= \frac{1}{5}\sum_{k=0}^{4}z^{-k}\\
            &=\frac{1}{5}\cdot\frac{1-z^{-5}}{1-z^{-1}}\\
            &\Big\downarrow z=e^{j\omega}\nonumber\\
            H(e^{j\omega})&=\frac{1}{5}\cdot\frac{1-e^{-j5\omega}}{1-e^{-j\omega}}\\
            &=\frac{1}{5}\cdot\frac{(1-e^{-j5\omega})e^{j5\omega/2}}{(1-e^{-j\omega})e^{j\omega/2}}\cdot\frac{e^{-j5\omega/2}}{e^{-j\omega/2}}\\
            &=\frac{1}{5}\cdot\frac{e^{j5\omega/2}-e^{-j5\omega/2}}{e^{j\omega/2}-e^{-j\omega/2}}\cdot e^{-2j\omega}\\
            &=\frac{1}{5}\cdot\frac{\sin (5\omega/2)}{\sin(\omega/2)}\cdot e^{-2j\omega}
        \end{align}
    \end{solution}

    \part Encuentre el retardo de grupo del sistema.
    \begin{solution}
        Para el retardo de grupo necesitamos la respuesta continua de fase $\Psi(\omega)$.
        Esta se puede encontrar obteniendo la fase y descartando discontinuidades,
        o bien mediante inspección, si la respuesta en frecuencia se escribe en
        coordenadas polares:
        \begin{gather}
            H(e^{j\omega})=G(\omega)e^{j\Psi(\omega)}=\left(\frac{1}{5}\cdot\frac{\sin (5\omega/2)}{\sin(\omega/2)}\right)\cdot e^{j(-2\omega)}\\
            \Psi(\omega)=-2\omega\\
            \tau_{\text{gd}}(\omega)=-\frac{d\Psi(\omega)}{d\omega}=2
        \end{gather}
        El retardo de grupo es constante. Este sistema no distorsiona señal.
    \end{solution}

    \part Encuentre los ceros de $H(z)$ y ubíquelos en el plano Z.
    \begin{solution}
        Con:
        $$
        H(z)=\frac{1}{5}\cdot\frac{1-z^{-5}}{1-z^{-1}}
        $$
        Los ceros vienen dados por:
        \begin{align}
            1-z^{-5}&=0\\
            z^5&=1\\
            z&=e^{j 2\pi k/5},\quad k=0,1,2,3, 4
        \end{align}
        Al analizar los polos, tenemos:
        \begin{align}
            1-z^{-1}&=0\\
            z&=1
        \end{align}
        Este cero se cancela con el polo $z=e^{j2\pi(0)/5}=1$. Por lo tanto, los
        ceros de la función de transferencia son:
        \begin{equation}
            z=e^{j 2\pi k/5},\quad k=1,2,3, 4
        \end{equation}
        \includegraphics[width=10cm]{imagenes/4.2d.png}

    \end{solution}

    \part Determine la salida $y[n]$ ante entrada $x[n]=(-1)^n+3\sin(\pi n/2)$
    \begin{solution}
        La señal es periódica por lo que podemos aplicar la
        propiedad de funciones propias para sistemas LTI. Para encontrar la respuesta sin embargo,
        conviene escribirla en términos de senos y cosenos:
        \begin{align}
            x[n]&=(-1)^n+3\sin(\pi n/2)\\
            &=\cos(\pi n)+3\sin(\pi n/2)
        \end{align}
        Debemos calcular magnitud y fase para $\omega=\pi,\,\pi/2$. Primero obtenemos
        la magnitud y fase del sistema:
        \begin{align}
            |H(e^{j\omega})|&=\left|\frac{1}{5}\cdot\frac{\sin(5\omega/2)}{\sin(\omega/2)}\cdot e^{-2j\omega}\right|& \angle H(e^{j\omega})&=\angle\left(\frac{1}{5}\cdot\frac{\sin(5\omega/2)}{\sin(\omega/2)}\cdot e^{-2j\omega}\right)\\
            &=\frac{1}{5}\left|\frac{\sin(5\omega/2)}{\sin(\omega/2)}\right|& &=\angle\left(\frac{\sin(5\omega/2)}{\sin(\omega/2)}\right)-2\omega
        \end{align}
        Evaluamos en los valores ya mencionados:
        \begin{align}
            &(\omega=\pi):&   &|H(e^{j\pi})|=\frac{1}{5} &\angle H(e^{j\pi})=0-2\pi=-2\pi\\
            &(\omega=\pi/2):& &|H(e^{j\pi/2})|=\frac{1}{5} &\angle H(e^{j\pi/2})=\pi-2\frac{\pi}{2}=0
        \end{align}
        Utilizando los valores calculados, se obtiene la respuesta del sistema:
        \begin{align}
            x[n]&=\cos(\pi n)+3\sin(\pi n/2)\\
            &\Big\downarrow\mathcal{H}\nonumber\\
            y[n]&=|H(e^{j\pi})|\cos\left[\pi n + \angle H(e^{j\pi})\right]
                + 3|H(e^{j\pi/2})|\sin\left[\pi n/2 + \angle H(e^{j\pi/2})\right]\\
            &=\frac{1}{5}\cos(\pi n -2\pi) + \frac{3}{5}\sin(\pi n/2+0)\\
            &=\frac{1}{5}\cos(\pi n) + \frac{3}{5}\sin(\pi n/2)
        \end{align}
    \end{solution}
\end{parts}

\question
Considere el filtro pasabajo $h_{lp}[n]=(0.9)^nu[n]$

\begin{parts}
\part Encuentre la respuesta en frecuencia $H_{lp}(e^{j\omega})$
\begin{solution}
    \begin{align}
        h_{lp}[n]&=(0.9)^nu[n]\\
        &\Big\downarrow\mathcal{Z}\nonumber\\
        H_{lp}(z)&=\frac{1}{1-0.9z^{-1}}\\
        &\Big\downarrow z=e^{j\omega}\nonumber\\
        H_{lp}(e^{j\omega})&=\frac{1}{1-0.9e^{j\omega}}
    \end{align}
\end{solution}

\part Diseñe un filtro pasabanda $H_{bp}(e^{j\omega})$ cuya magnitud sea máxima para las frecuencias
$\omega\approx\pm \pi/3$
\begin{solution}
    El filtro pasabajo de (a) tiene máxima magnitud para $\omega=0$. Si consideramos el
    filtro pasabajo con desplazamiento en la frecuencia $H(e^{j(\omega-\omega_c)})$,
    este es máximo en magnitud para $\omega=\omega_c$.
    
    Utilizando dos pasabajos desplazados por $\omega_c=\pm\pi/3$, deberíamos
    obtener un filtro que tenga dos picos de magnitud para las frecuencias deseadas:
    \begin{align}
        H_{bp}(e^{j\omega})&=H_{lp}(e^{j(\omega-\omega_c)})+H_{lp}(e^{j(\omega+\omega_c)})\\
        &=\frac{1}{1-0.9e^{j(\omega-\omega_c)}} + \frac{1}{1-0.9e^{j(\omega+\omega_c)}}
    \end{align}
\end{solution}

\part Bosqueje la magnitud. Compruebe y determine $|H(e^{j\omega})|_\text{max}$
y la frecuencia correspondiente en Matlab.
\begin{solution}
    La magnitud es $10.5421$, para $\omega=1.0498$.

    \includegraphics[width=14cm]{imagenes/4.3c.png}
    \begin{minted}{matlab}
%% 3 (c)
Hlp_ej = @(w) 1./(1-0.9.*exp(-1j.*w));

Hbp_ej = @(w) Hlp_ej(w-pi/3) + Hlp_ej(w+pi/3);
mag = @(w) abs(Hbp_ej(w));

w_max = fminbnd(@(w) -mag(w), 0, pi)
mag_max = mag(w_max)

w = linspace(-pi,pi,1000);
figure;
plot(w, mag(w),'LineWidth',2); hold on
plot(w_max, mag(w_max),'o','LineWidth',2)
plot(-w_max, mag(-w_max),'o','LineWidth',2)
xline([-pi/3 pi/3])
xlabel('Frequencia (rad/s)');
ylabel('Magnitud');
title('Magnitud de Hbp');
grid on;
    \end{minted}
\end{solution}

\end{parts}

\end{questions}
%\tableofcontents
%\listoffigures
%\listoftables
%\listoftodos

%%%%%%%%%%%%%%%%%%%%%%%%%%%%%%%%
%% -- Fin del Documento -- %%
%%%%%%%%%%%%%%%%%%%%%%%%%%%%%%%%
\end{document}