\documentclass[12pt]{exam}
\usepackage{preamble}

% disable para esconder
\usepackage[enable]{easy-todo}
\usepackage{bm}
\usepackage{amssymb,amsmath}

% Comentar para esconder soluciones
\printanswers

%%%%%%%%%%%%%%%%%%%%%%%%%%%%%%%%
%% -- Inicio del Documento -- %%
%%%%%%%%%%%%%%%%%%%%%%%%%%%%%%%%
\begin{document}
\begin{center}
    {\Large
        Ayudantía 7 - Procesamiento Digital de Señales
    }
\end{center}

\begin{questions}
\question
Sea la secuencia $x[n]=\{1\;1\;0\;3\}$
\begin{parts}
    \part Calcule la TFD $X[k]$ utilizando la matriz $\bm{W_4}$.
    \begin{solution}
        Recordemos que la TFD se define por:
        \begin{equation}
            X[k] = \sum_{n=0}^{N-1}x[n]e^{-j\frac{2\pi}{N}kn}
        \end{equation}

        Y definiendo $W_N=e^{-j\frac{2\pi}{N}}$:
        \begin{equation}
            X[k] = \sum_{n=0}^{N-1}x[n]W_N^{kn}
        \end{equation}
        Esta suma se puede reescribir como un producto entre una matriz $\bm{W_N}$
        y un vector $\bm x$. De forma general:
        \begin{gather}
            \bm{X}=\bm{W_Nx}\\
            \begin{bmatrix}
                X[0]\\X[1]\\X[2]\\\vdots\\X[N-1]
            \end{bmatrix}
            =\begin{bmatrix}
                1 & 1 & 1 & \dots & 1\\
                1 & W_N^1 & W_N^2 & \dots & W_N^{N-1}\\  
                1 & W_N^2 & W_N^4 & \dots & W_N^{2(N-1)}\\  
                \vdots & \vdots & \vdots & \ddots & \vdots\\  
                1 & W_N^{N-1} & W_N^{2(N-1)} & \dots & W_N^{(N-1)(N-1)}\\  
            \end{bmatrix}
            \begin{bmatrix}
                x[0]\\  x[1]\\  x[2]\\ \vdots\\x[N-1]
            \end{bmatrix}
        \end{gather}
        Para la matriz $\bm{W_N}$: 
        \begin{itemize}
            \item Tiene dimensiones $N\times N$.
            \item La primera fila y columna son solo unos.
            \item Para la fila $k$, los exponentes de $W_N$ aumentan de $k-1$ en $k-1$. Ej: La segunda fila los exponentes
        van de uno en uno, para la tercera fila van de dos en dos, etc...
        \item Es simétrica ($\bm{W_N=W_N^\text{T}}$)
        \end{itemize} 
        En el caso de la secuencia $x[n]=\{1\; 1\; 0\; 3\}$ ($N=4$) la transformada es:
        \begin{align}
            X[k]=\sum_{n=0}^{3}x[n]W_{4}^{kn}
        \end{align}
        y en forma matricial:
        \begin{gather}
            \bm{X}=\bm{W_4x}\\
            \begin{bmatrix}
                X[0]\\ X[1]\\ X[2]\\ X[3] 
            \end{bmatrix}
            =\begin{bmatrix}
                1 & 1 & 1 & 1\\
                1 & W_4^1 & W_4^2 & W_4^3 \\
                1 & W_4^2 & W_4^4 & W_4^6 \\
                1 & W_4^3 & W_4^6 & W_4^9 \\
            \end{bmatrix}
            \begin{bmatrix}
                x[0]\\  x[1]\\  x[2]\\  x[3]
            \end{bmatrix}
        \end{gather}
        Como $W_N^n=W_N^{n+N}$, entonces $W_4^4=W_4^0$, $W_4^6=W_4^2$ y $W_4^9=W_4^1$ se escribe:
         \begin{align}
            \begin{bmatrix}
                X[0]\\ X[1]\\ X[2]\\ X[3] 
            \end{bmatrix}
            =\begin{bmatrix}
                1 & 1 & 1 & 1\\
                1 & W_4^1 & W_4^2 & W_4^3 \\
                1 & W_4^2 & W_4^0 & W_4^2 \\
                1 & W_4^3 & W_4^2 & W_4^1 \\
            \end{bmatrix}
            \begin{bmatrix}
                x[0]\\  x[1]\\  x[2]\\  x[3]
            \end{bmatrix}
        \end{align}
        Notando que $W_4^0=1$, $W_4^1=e^{-j\frac{2\pi}{4}}=-j$, $W_4^2=-1$ y $W_4^3=j$, y reemplazando
        por los valores de $x[n]$:
        \begin{align}
            \begin{bmatrix}
                X[0]\\ X[1]\\ X[2]\\ X[3] 
            \end{bmatrix}
            =\begin{bmatrix}
                1 & 1 & 1 & 1\\
                1 & -j & -1 & j \\
                1 & -1 & 1 & -1 \\
                1 & j & -1 & -j \\
            \end{bmatrix}
            \begin{bmatrix}
                1\\  1\\  0\\  3
            \end{bmatrix}
            =\begin{bmatrix}
                1+1+0+3\\
                1-j+0+3j\\
                1-1+0-3\\
                1+j+0-3j
            \end{bmatrix}
            =
            \begin{bmatrix}
                5\\1+2j\\-3\\1-2j
            \end{bmatrix}
        \end{align}
        Es decir: $X[k]=\{5\quad {1+2j}\quad-3\quad1-2j\}$
    \end{solution}

    \part Determine la la TFD inversa $z[n]$ de $W^{2k}_4X[k]$. Grafique $z[n]$
    y $x[n]$ para su comparación.
    \begin{solution}
        Para el cálculo de la transformada inversa de
        $W_4^{2k}X[k]=e^{-j\frac{2\pi}{4}2k}X[k]$ debemos tener en cuenta la propiedad
        de desplazamiento circular. Sea una señal $x[n]$ y $X[k]$ su TFD de $N$ puntos:
        \begin{align}
            \boxed{x[\langle n-m \rangle_N]
            \stackrel{TFD}{\longleftrightarrow}
            W_N^{mk}X[k]=e^{-j\frac{2\pi}{N}mk}X[k]}
        \end{align} Con $\langle*\rangle_N$ la operación módulo $N$. Notar que la
        exponencial debe estar en forma $e^{-j\frac{2\pi}{N}mk}$.
        
        Al aplicar la propiedad en nuestro caso ($N=4$, $m=2$):
        \begin{equation}
            W_4^{2k}X[k]\xrightarrow[]{TFDI}z[n]=x[\langle n-2 \rangle_4]
        \end{equation}

        Para encontrar los valores de la señal para $n=0,\,1,\,2,\,3$ se retarda
        la señal en $m=2$ rotando los valores que se salen del rango $0\leq n\leq 3=N-1$:
        \begin{align}
            x[n]&=\{1\;1\;0\;3\}\\
            &\downarrow\nonumber\\
            z[n]=x[\langle n-2\rangle_4]&=\{0\;3\;1\;1\}
        \end{align}

        Al graficar $x[n]$, su extensión periódica y $z[n]$ vemos como la TFD no ``ve''
        la señal $x[n]$ sino su extensión periódica al aplicar la propiedad de desplazamiento:
        
        \includegraphics[width=14cm]{imagenes/7.1b.png}
        
    \end{solution}

    \part Realice la convolución circular $x[n]\circledast h[n]$, con $h[n]=\{1\;0\;1\;0\}$
    en el plano temporal.

    \part Repita (c) usando las TFD $X[k]$ y $H[k]$.
\end{parts}

\question Muestre brevemente como se realizaría el calculo de la TFD de $x[n]=\{0\;1\;2\;3\}$
usando el algoritmo de Transformada Rápida de Fourier visto en clases.

\question Implemente en Matlab el algoritmo de Transformada Rápida de Fourier en
Matlab. Puede asumir que el vector de entrada es de $N=2^{m}$ una potencia de dos.

 \begin{parts}
    \part Compruebe la validez calculando la TFD $x[n]$ de la pregunta 1.
    \part Considere la siguiente implementación directa (por definición) de la TFD:
\begin{minted}{matlab}
function Xdft=dftdirect(x)
% Direct computation of the DFT
N=length(x); Q=2*pi/N;
for k=1:N
S=0;
for n=1:N
W(k,n)=exp(-1j*Q*(k-1)*(n-1));
S=S+W(k,n)*x(n);
end
Xdft(k)=S;
end
\end{minted}

    Usando las funciones \texttt{tic} y \texttt{toc} de Matlab, compare
    el rendimiento de su implementación y la de \texttt{dftdirect}. Para la generación del vector de entrada de prueba puede utilizar la función \texttt{rand}.
 \end{parts}

\end{questions}
%\tableofcontents
%\listoffigures
%\listoftables
%\listoftodos

%%%%%%%%%%%%%%%%%%%%%%%%%%%%%%%%
%% -- Fin del Documento -- %%
%%%%%%%%%%%%%%%%%%%%%%%%%%%%%%%%
\end{document}