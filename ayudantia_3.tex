\documentclass[12pt]{exam}
\usepackage{preamble}

% disable para esconder
\usepackage[enable]{easy-todo}

% Comentar para esconder soluciones
\printanswers

%%%%%%%%%%%%%%%%%%%%%%%%%%%%%%%%
%% -- Inicio del Documento -- %%
%%%%%%%%%%%%%%%%%%%%%%%%%%%%%%%%
\begin{document}
\begin{center}
    {\Large
        Ayudantía 3 - Procesamiento Digital de Señales
    }
\end{center}

\begin{questions}
\question Para las siguientes señales tiempo continuo, encuentre la TFTC:
\begin{parts}
\part $x_1(t)=e^{-3|t|}\sin 2\pi t$
    \begin{solution}
        La señal puede ser reescrita como:
        \begin{align}
            e^{-3|t|}\sin 2\pi t &= \begin{cases}
                e^{-3t}\sin 2\pi t,\quad t\geq 0\\
                e^{3t}\sin 2\pi t < 0
            \end{cases}\\
            &= e^{-3t}\sin(2\pi t) u(t) + e^{3t}\sin(2\pi t) u(-t)
        \end{align}
        Aplicamos transformada de Laplace usando la propiedad de linealidad. Para el primer término:
        \begin{align}
            \mathcal{L}\{e^{-3t}\sin(2\pi t) u(t)\}
            &=\left.\frac{2\pi}{s^2+(2\pi)^2}\right|_{s+3}\\
            &=\frac{2\pi}{(s+3)^2+4\pi^2}
        \end{align}
        Para el segundo término:
        \begin{align}
            \mathcal{L}\{e^{3t}\sin(2\pi t) u(-t)\} &= \mathcal{L}\{e^{-3(-t)}\sin(-2\pi [-t])u(-t)\}\\
            &=-\mathcal{L}\{e^{-3(-t)}\sin(2\pi [-t])u(-t)\}\\
            &=-\left.\mathcal{L}\{e^{-3t}\sin(2\pi t) u(t)\}\right|_{-s}\\
            &=-\left.\frac{2\pi}{(s+3)^2+4\pi^2}\right|_{-s}\\
            &=-\frac{2\pi}{(3-s)^2+4\pi^2}
        \end{align}
        Finalmente escribimos la transformada de Laplace como la suma de las partes y
        realizamos la sustitución para encontrar la TFTC.
        \begin{align}
           \mathcal{L}\{e^{-3|t|}\sin 2\pi t\}&= \frac{2\pi}{(s+3)^2+4\pi^2}-\frac{2\pi}{(3-s)^2+4\pi^2}\\
           &\Big\downarrow s=j\Omega\nonumber\\
           X_1(j\Omega)&=\frac{2\pi}{(j\Omega+3)^2+4\pi^2}-\frac{2\pi}{(3-j\Omega)^2+4\pi^2}
        \end{align}
    \end{solution}


\part $x_2(t)=e^{-3|t-5|}\sin(2\pi [t-5])*e^{-3|t+1|}\sin(2\pi [t+1])$
\begin{solution}
    Aquí notamos que las señales que se convolucionan se pueden escribir en función de la
    señal $x_1$ de la parte (a) con un retardo (o adelanto):
    \begin{equation}
       x_2(t)=x_1(t-5)*x_1(t+1)
    \end{equation}
    Recordemos que la convolución es multiplicación en la frecuencia. Podemos
    aplicar la TFTC directamente, considerando la propiedad de retardo en el tiempo.
    \begin{align}
        x_2(t)&=x_1(t-5)*x_1(t+1)\\
        &\Big\downarrow \text{TFTC}\nonumber\\
        X_2(j\Omega)&=e^{-5j\Omega}X_1(j\Omega)\cdot e^{j\Omega}X_1(j\Omega)\\
        &=e^{-4j\Omega}X_1^2(j\Omega)\\
        &=e^{-4j\Omega}\left(\frac{2\pi}{(j\Omega+3)^2+4\pi^2}-\frac{2\pi}{(3-j\Omega)^2+4\pi^2}\right)^2
    \end{align}
\end{solution}
\end{parts}

\question Sea la señal:
$$
x[n] = \begin{cases}
        1 - \sin(\pi n /4),\quad 0\leq n \leq 3 \\
        0,\quad e.t.o.c.
    \end{cases}
$$
\begin{parts}
    \part Determine la TFTD de la señal.
    \begin{solution}
        La señal toma valores distinto de $0$ solo para $0\leq n \leq 3$.
        Podemos reescribir la señal como una sucesión de impulsos:
        \begin{align}
            x[n]&=x[0]\delta[n]+x[1]\delta[n-1]+x[2]\delta[n-2]+x[3]\delta[n-3]\\
            &=\delta[n]+(1-\sqrt{2}/2)\delta[n-1]+(1-\sqrt{2}/2)\delta[n-3]\\
            &=\delta[n]+(1-\sqrt{2}/2)(\delta[n-1]+\delta[n-3])
        \end{align}
        Aplicamos la transformada Z y posteriormente sustituimos para obtener la TFTD:
       \begin{align}
            x[n]&=\delta[n]+(1-\sqrt{2}/2)(\delta[n-1]+\delta[n-3])\\
            &\Big\downarrow \mathcal{Z}\{\;\}\nonumber\\
            X(z)&=1+(1-\sqrt{2}/2)(z^{-1}+z^{-3})\\
            &\Big\downarrow z=e^{j\omega}\nonumber\\
            X(e^{j\omega})&=1+(1-\sqrt{2}/2)(e^{-j\omega}+e^{-3j\omega})
       \end{align} 
    \end{solution}

    \part Se construye la señal periódica $\tilde{x}_N$ de periodo $N\geq4$ a partir de la señal $x$:
    $$
    \tilde{x}_N[n]=\sum_{k=-\infty}^{\infty}x[n-kN]
    $$
    Encuentre una expresión para los coeficientes de Fourier de la señal periódica en función de $N$.
    \begin{solution}
       Para encontrar los coeficientes de Fourier tenemos dos maneras:
       \begin{equation}
            c_k=\frac{1}{N}\sum_{k=0}^{1-N}\tilde{x}_N[n]e^{-j\frac{2\pi}{N}kn}\qquad c_k=\left.\frac{1}{N}X(e^{j\omega})\right|_{\omega=\frac{2\pi}{N}k}
       \end{equation}
       Con $X(e^{j\omega})$, TFTD de un periodo base. Esta transformada ya la tenemos es la que se
       determino en 2(a). Sustituyendo tenemos:
       \begin{align}
        c_k&=\left.\frac{1}{N}X(e^{j\omega})\right|_{\omega=\frac{2\pi}{N}k}\\
        &=\frac{1}{N}\left[1+(1-\sqrt{2}/2)(e^{-j\omega}+e^{-3j\omega})\right]_{\omega=\frac{2\pi}{N}k}\\
        &=\frac{1}{N}\left[1+(1-\sqrt{2}/2)(e^{-j\frac{2\pi}{N}k}+e^{-j\frac{6\pi}{N}k})\right]
       \end{align}
    \end{solution}

    \part Escriba los coeficientes de Fourier para $\tilde{x}_4[n]$ $(N=4)$.
    \begin{solution}
        Si consideramos la señal de periodo 4 ($N=4$), la expresión para los coeficientes queda:
        \begin{align}
        c_k&=\frac{1}{4}\left[1+(1-\sqrt{2}/2)(e^{-j\frac{2\pi}{4}k}+e^{-j\frac{6\pi}{4}k})\right]\\
        &=\frac{1}{4}\left[1+(1-\sqrt{2}/2)(e^{-j\frac{\pi}{2}k}+e^{-j\frac{3\pi}{2}k})\right]\\
        &=\frac{1}{4}\left[1+(1-\sqrt{2}/2)(e^{-j\frac{\pi}{2}k}+e^{j\frac{\pi}{2}k})\right]\\
        &=\frac{1}{4}\left[1+(2-\sqrt{2})\cos\frac{\pi}{2}k\right]
        \end{align}
        Obtenemos los coeficientes evaluando $k$ desde 0 hasta $N-1=3$:
        \begin{align}
            c_0&=\frac{1}{4}\left[1+(2-\sqrt{2})\cos(0)\right]=0.3964\\
            c_1&=\frac{1}{4}\left[1+(2-\sqrt{2})\cos\frac{\pi}{2}\right]=0.25\\
            c_2&=\ldots = 0.1036\\
            c_3&=\ldots = 0.25
        \end{align}
    \end{solution}
\end{parts}

\question
Considere el sistema \textbf{causal} descrito por:
$$
    y[n] = ay[n-1] + 2x[n]
$$
\begin{parts}
    \part Determine la respuesta en frecuencia $H(e^{j\omega})$ para $a=0.5$.
    \begin{solution}
        Aplicamos transformada Z a la ecuación, para encontrar la función de transferencia:
        \begin{align}
            y[n] &= ay[n-1] + 2x[n]\\
            &\Big\downarrow \mathcal{Z}\{\;\}\nonumber\\
            Y(z) &= az^{-1}Y(z) + 2X(z)\\
            \Rightarrow H(z)&=\frac{Y(z)}{X(z)}=\frac{2}{1-az^{-1}}\\
            &\Big\downarrow a=0.5\nonumber\\
            H(z)&=\frac{2}{1-0.5z^{-1}}
        \end{align}
        El sistema tiene un polo $p_1=0.5$. Al tratarse de un sistema causal, deducimos
        deducimos que la RC es el exterior de una circunferencia. RC: $|z|>0.5$. El sistema
        es estable podemos obtener la respuesta en frecuencia sustituyendo $z=e^{j\omega}$:
        \begin{align}
            H(e^{j\omega})=\frac{2}{1-0.5e^{-j\omega}}=\frac{2}{1-0.5\cos \omega +0.5j\sin \omega}
        \end{align}
        Y las respuestas de magnitud y fase son:
        \begin{align}
            |H(e^{j\omega})|&=\frac{2}{|1-0.5\cos \omega +0.5j\sin \omega|}& \angle H(e^{j\omega})&=- \angle(1-0.5\cos j\omega +0.5\sin j\omega)\\
            &=\frac{2}{\sqrt{1.25-\cos\omega}} & &=-\arctan\frac{0.5\sin\omega}{1-0.5\cos\omega}
        \end{align}
    \end{solution}

    \part Encuentre la salida $y[n]$ si la secuencia de entrada es $x[n] = 3\sin(\pi n/4)$.
    \begin{solution}
        Para encontrar la salida de un sistema estable, con una señal periódica como entrada,
        consideramos la propiedad de funciones propias de un sistema LTI. Para una señal exponencial
        compleja de frecuencia angular $\omega_0$, la salida viene dada por:
        \begin{equation}
            \boxed{x[n]=e^{j\omega_0n}\xrightarrow{\mathcal{H}}y[n]=H(e^{j\omega_0})e^{j\omega_0 n}=|H(e^{j\omega_0})|e^{j\omega_0 n+\angle H(e^{j\omega_0})}}
        \end{equation}
        Esta propiedad, se extiende para señales sinusoidales cuando la respuesta a impulso
        es real (una demostración aparece en la página 203 del libro guía). En nuestro caso, para 
        una entrada $x[n]=3\sin(\pi n/4)$:
        \begin{equation}
            x[n]=3\sin\Big(\underbrace{\frac{\pi}{4}}_{\omega_0}n\Big)\xrightarrow{\mathcal{H}}3|H(e^{j\pi/4})|\sin\left(\frac{\pi}{4}n+\angle H(e^{j\pi/4})\right)
        \end{equation}
        Sustituyendo por las expresiones de magnitud y fase encontradas anteriormente, tenemos:
        \begin{align}
            y[n]&=3\frac{2}{\sqrt{1.25-\cos\pi/4}}\sin\left(\frac{\pi}{4}n-\arctan\frac{0.5\sin\omega}{1-0.5\cos\omega}\right)\\
            &=8.1432\sin\left(\frac{\pi}{4}n-0.5005\right)
        \end{align}
    \end{solution}

    \part Repita (b) para una secuencia de entrada $x[n] = \tilde{x}_4[n]$ de la pregunta 2(c).
    \begin{solution}
        Para encontrar la respuesta al tomar la señal periódica $\tilde{x}_4$ consideramos
        su representación mediante los coeficientes de Fourier encontrados en 2(c):
        \begin{equation}
            \tilde{x}_4[n]=\sum_{k=0}^{3}c_ke^{j\frac{2\pi}{4}kn}=\sum_{k=0}^{3}c_ke^{j\frac{\pi}{2}kn}
        \end{equation}
        Para encontrar la respuesta aplicamos la propiedad de funciones propias para
        sistemas LTI término a término:
        \begin{align}
            \tilde{x}_4[n]&=\sum_{k=0}^{3}c_ke^{j[\frac{\pi}{2}k]n}\\
            &\Big\downarrow \mathcal{H}\nonumber\\
            y[n]&=\sum_{k=0}^{3}c_kH(e^{j\frac{\pi}{2}k})e^{j\frac{\pi}{2}kn}
        \end{align}
        Evaluando para $0\leq k \leq 3$ y expandiendo la suma:
        \begin{align}
            y[n]&=c_0H(e^{0}) +c_1H(e^{j\pi/2})e^{j\frac{\pi}{2}n} +c_2H(e^{j\pi})e^{j\pi n} +c_3H(e^{j3\pi/2})e^{j\frac{3\pi}{2}n}\\
            \;\nonumber\\
            &=(0.3964)(4) +(0.25)(1.6-0.8j)e^{j\frac{\pi}{2}n}\nonumber\\
            &\qquad +(0.1036)(1.33)e^{j\pi n} +(0.25)(1.6+0.8j)e^{j\frac{3\pi}{2}n}\\
            \;\nonumber\\
            &=1.5858 +(0.4-0.2j)e^{j\frac{\pi}{2}n}\nonumber\\
            &\qquad +0.1381e^{j\pi n} +(0.4+0.2j)e^{j\frac{3\pi}{2}n}
        \end{align}
    \end{solution}

    \part ¿Que ocurriría si $a=1.5$?
    \begin{solution}
        Si $a=1.5$, la función de transferecia queda como:
        \begin{equation}
            H(z)=\frac{2}{1-1.5z^{-1}}
        \end{equation}
        Con un polo $p_1=1.5$, y RC:$|z|>1.5$ (causal). Concluimos que es sistema no
        es estable. La respuesta en frecuencia $H(e^{j\omega})$ no existe pues involucra
        evaluar $z$ en la circunferencia unitaria que no esta en la RC.

        La respuesta con entrada una señal periódica no es posible calcularse.
    \end{solution}
\end{parts}
\end{questions}
%\tableofcontents
%\listoffigures
%\listoftables
%\listoftodos

%%%%%%%%%%%%%%%%%%%%%%%%%%%%%%%%
%% -- Fin del Documento -- %%
%%%%%%%%%%%%%%%%%%%%%%%%%%%%%%%%
\end{document}