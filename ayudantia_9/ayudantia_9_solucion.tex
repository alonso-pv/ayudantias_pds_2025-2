\documentclass[12pt]{exam}
\usepackage{../preamble}

% disable para esconder
\usepackage[enable]{easy-todo}

% Comentar para esconder soluciones
\printanswers

%%%%%%%%%%%%%%%%%%%%%%%%%%%%%%%%
%% -- Inicio del Documento -- %%
%%%%%%%%%%%%%%%%%%%%%%%%%%%%%%%%
\begin{document}
\begin{center}
    {\Large
        Ayudantía 9 - Procesamiento Digital de Señales
    }
\end{center}

\begin{questions}
\question Realice las siguientes conversiones de especificaciones de filtro:
\begin{parts}
    \part Dadas las especificaciones absoutas $\delta_p=0.01$ y $\delta_s=0.0001$,
    determine las especificaciones relativas $A_p$ y $A_s$, y las especificaciones
    análogas $\epsilon$ y $A$.
    \begin{solution}\\
        \includegraphics[width=14cm]{imagenes/9.1.png}

        \paragraph{Absolutas a relativas} Para pasar de especificaciones absolutas análogas, usamos las fórmulas
        \begin{align}
            &A_p=20\log\left(\frac{1+\delta_p}{1-\delta_p}\right)
            &A_s=20\log\left(\frac{1+\delta_p}{\delta_s}\right)\label{eq:2}
        \end{align}

        Reemplazando:
        \begin{align}
            A_p&=20\log\left(\frac{1+0.01}{1-0.01}\right)
            &A_s&=20\log\left(\frac{1+0.01}{0.0001}\right)\\
            &=0.17\text{ dB} & &=80.08\text{ dB}
        \end{align}

        \paragraph{Relativas a análogas} Para obtener las especificaciones análogas
        utilizamos la fórmulas
        \begin{align}
            &\epsilon=\sqrt{10^{A_p/10}-1} &A=10^{A_s/20}\label{eq:1}
        \end{align}

        Reemplazando por los valores calculados:        
        \begin{align}
            \epsilon&=\sqrt{10^{0.17/10}-1} & A&=10^{80.08/20}\\
            &=0.19 & &=10000
        \end{align}

    \end{solution}

    \part Dadas las especificaciones análogas $\epsilon=0.25$ y $A=200$, obtenga
    las especificaciones relativas $A_p$ y $A_s$, y las especificaciones absolutas
    $\delta_p$ y $\delta_s$.
    \begin{solution}
        \paragraph{Análogas a relativas} Despejando las ecuaciones del paso (\ref{eq:1})
        se obtiene
        \begin{align}
            A_p&=20\log(\sqrt{1+\epsilon^2}) & A_s&=20\log(A)
        \end{align}

        Reemplazando por los valores:
        \begin{align}
            A_p&=20\log(\sqrt{1+0.25^2}) & A_s&=20\log(200)\\
            &=0.26\text{ dB} & &=46.02\text{ dB}
        \end{align}

        \paragraph{Relativas a absolutas} Despejando las ecuaciones del paso (\ref{eq:2})
        se obtiene
        \begin{align}
            \delta_p&=\frac{10^{A_p/20}-1}{10^{A_p/20}+1} &
            \delta_s&=\frac{1+\delta_p}{10^{A_s/20}}
        \end{align}

        Reemplazando por los datos se obtiene:
        \begin{align}
            \delta_p&=\frac{10^{0.26/20}-1}{10^{0.26/20}+1} &
            \delta_s&=\frac{1+0.014}{10^{46.02/20}}\\
            &=0.014 & &=0.005
        \end{align}

    \end{solution}
\end{parts}

\question Para cada uno de los siguientes filtros FIR de fase lineal descritos por respuesta a
impulso, graficar en Matlab respuesta a impulso, respuesta de amplitud, respuesta
de ángulo, y ceros.
\begin{parts}
    \part Filtro tipo I: $h[n]=\{1,\;2,\;-1,\;5,\;-1,\;2,\;1\}$
    \begin{solution}
        $h[n]$ es de tipo I ya que tiene simetría par, con orden par:
        \begin{equation}
            h[n]=h[M-n]\text{ (simetría par)},\qquad M=6 \text{ (par)}
        \end{equation}
        Reescribiendo $h[n]$ como
        \begin{equation}
            h[n]=\delta[n]
                +2\delta[n-1]
                -1\delta[n-2]
                +5\delta[n-3]
                -1\delta[n-4]
                +2\delta[n-5]
                +\delta[n-6]
        \end{equation}
        Obtenemos la TFTD:
        \begin{gather}
            H(z)=1 +2z^{-1} -z^{-2} +5z^{-3} -z^{-4} +2z^{-5} +z^{-6}\\
            \downarrow z=e^{j\omega}\notag\\
            H(e^{j\omega})=1 +2e^{-j\omega} -e^{-2j\omega} +5e^{-3j\omega} -e^{-4j\omega} +2e^{-5j\omega} +e^{-6j\omega}
        \end{gather}

        Factorizamos la expresión por $e^{-3j\omega}=e^{-\frac{M}{2}j\omega}$,
        nos aprovechamos de la simetría par de los coeficientes:
        \begin{align}
            H(e^{j\omega})=\left(e^{3j\omega} +2e^{2j\omega} -e^{j\omega}
                +5 -e^{-j\omega} +2e^{-2j\omega} +e^{-3j\omega}\right)e^{-3j\omega}
        \end{align}
        Recordando que $e^{j\theta}+e^{-j\theta}=2\cos(\theta)$:
        \begin{align}
            &e^{3j\omega} +2e^{2j\omega} -e^{j\omega}
                +5 -e^{-j\omega} +2e^{-2j\omega} +e^{-3j\omega}\\
            =&(e^{3j\omega} +e^{-3j\omega})+2(e^{2j\omega} + e^{-2j\omega}) -(e^{j\omega} +e^{-j\omega})
                +5  \\
            =&2\cos(3\omega)+4\cos(2\omega)-2\cos(\omega)+5
        \end{align}
        \begin{align}
            H(e^{j\omega})&=(2\cos(3\omega)+4\cos(2\omega)-2\cos(\omega)+5)e^{-3j\omega}\\
            &=A(e^{j\omega})e^{j\Psi(\omega)},\qquad \Psi(\omega)=-3\omega=-\frac{M}{2}\omega
        \end{align}

        Para graficar se usa:
        \begin{minted}[fontsize=\footnotesize]{matlab}
% --- Tipo I ---
h = [1, 2, -1, 5, -1, 2, 1];    M = 6; % M par, simetría par
A = 2*cos(3*w) +4*cos(2*w) -2*cos(w) + 5;
psi = -M/2*w;
        \end{minted}

        \includegraphics[width=14cm]{imagenes/9.2a.png}
        

        
    \end{solution}
    
    \part Filtro tipo II: $h[n]=\{1,\;2,\;-1,\;-1,\;2,\;1\}$
    \begin{solution}
        $h[n]$ es de tipo II ya que tiene simetría par y orden impar:
        \begin{equation}
            h[n]=h[M-n]\text{ (simetría par)},\qquad M=5 \text{ (impar)}
        \end{equation}
        Siguiendo el mismo procedimiento que en el punto anterior se obtiene la
        TFTD:
        \begin{gather}
            h[n]=\{1,\;2,\;-1,\;-1,\;2,\;1\}\\
            \downarrow\text{TFTD}\notag\\
            H(e^{j\omega})=1 +2e^{-j\omega} -e^{-2j\omega} -e^{-3j\omega} +2e^{-4j\omega} +e^{-5j\omega}
        \end{gather}
        Con $M=5$, factorizamos por $e^{-\frac{5}{2}j\omega}=e^{-\frac{M}{2}j\omega}$ para
        obtener $A(e^{j\omega})$ y $\Psi(\omega)$ (similar al punto anterior):
        \begin{align}
            H(e^{j\omega})&=\left(e^{\frac{5}{2}j\omega} +2e^{\frac{3}{2}j\omega} -e^{\frac{1}{2}j\omega} -e^{-\frac{1}{2}j\omega} +2e^{-\frac{3}{2}j\omega} +e^{-\frac{5}{2}j\omega}\right)e^{-\frac{5}{2}j\omega}\\
            &=\left(e^{\frac{5}{2}j\omega} +e^{-\frac{5}{2}j\omega} +2e^{\frac{3}{2}j\omega}  +2e^{-\frac{3}{2}j\omega}-e^{\frac{1}{2}j\omega} -e^{-\frac{1}{2}j\omega}\right)e^{-\frac{5}{2}j\omega}\\
            &=\left[2\cos\left(\frac{5}{2}\omega\right) + 4\cos\left(\frac{3}{2}\omega\right) - 2\cos\left(\frac{1}{2}\omega\right)\right]e^{-\frac{5}{2}j\omega}\\
            &=A(e^{j\omega})e^{j\Psi(\omega)},\qquad \Psi(\omega)=-\frac{5}{2}\omega=-\frac{M}{2}\omega
        \end{align}

        Notemos que en este caso, $\cos(\pm\frac{n}{2}\pi)=0$, es decir $\left.A(e^{j\omega})\right|_{\omega=\pm\pi}=0$.
        \begin{minted}[fontsize=\footnotesize]{matlab}
% --- Tipo II ---
h = [1, 2, -1, -1, 2, 1];      M = 5; % M impar, simetría par
A = 2*cos(5/2*w) +4*cos(3/2*w) -2*cos(1/2*w);
psi = -M/2*w;
        \end{minted}

        \includegraphics[width=14cm]{imagenes/9.2b.png}
    \end{solution}

    \part Filtro tipo III: $h[n]=\{1,\;2,\;-1,\;0,\;1,\;-2,\;-1\}$
    \begin{solution}
        En este caso $h[n]$ es de tipo III ya que tiene simetría impar y orden par:
        \begin{equation}
            h[n]=-h[M-n]\text{ (simetría impar)},\qquad M=6 \text{ (par)}
        \end{equation}
        El procedimiento para obtener la respuesta de amplitud es similar a los puntos anteriores:
        \begin{gather}
            h[n]=\{1,\;2,\;-1,\;0,\;1,\;-2,\;-1\}\\
            \downarrow\text{TFTD}\notag\\
            H(e^{j\omega})=1 +2e^{-j\omega} -e^{-2j\omega} +(0)e^{-3j\omega} +e^{-4j\omega} -2e^{-5j\omega} -e^{-6j\omega}
        \end{gather}
        Factorizamos la expresión por $e^{-3j\omega}=e^{-\frac{M}{2}j\omega}$:
        \begin{align}
            H(e^{j\omega})&=\left(e^{3j\omega} +2e^{2j\omega} -e^{j\omega} +e^{-j\omega} -2e^{-2j\omega} -e^{-3j\omega}\right)e^{-3j\omega}\\
            &=\left[e^{3j\omega} -e^{-3j\omega} +2\left(e^{2j\omega} -e^{-2j\omega}\right) -\left(e^{j\omega} -e^{-j\omega}\right)\right]e^{-3j\omega}
        \end{align}
        Usando $e^{j\theta}-e^{-j\theta}=2j\sin(\theta)$, y $j=e^{j\pi/2}$:
        \begin{align}
            H(e^{j\omega})&=\left[2j\sin(3\omega) +4j\sin(2\omega) -2j\sin(\omega)\right]e^{-3j\omega}\\
            &=\left[2\sin(3\omega) +4\sin(2\omega) -2\sin(\omega)\right]e^{j(\frac{\pi}{2}-3\omega)}\\
            &=A(e^{j\omega})e^{j\Psi(\omega)},\qquad\Psi(\omega)=\frac{\pi}{2}-3\omega =\frac{\pi}{2}-\frac{M}{2}\omega
        \end{align}

        Como $\sin(n\pi)=0$, entonces $\left.A(e^{j\omega})\right|_{\omega=0,\;\pm\pi}=0$.
        \begin{minted}[fontsize=\footnotesize]{matlab}
% --- Tipo III ---
h = [1, 2 -1, 0, 1, -2, -1];   M = 6; % M par, simetría impar
A = 2*sin(3*w) +4*sin(2*w) -2*sin(w);
psi = pi/2-M/2*w;
        \end{minted}

        \includegraphics[width=14cm]{imagenes/9.2c.png}
    \end{solution}

    \part Filtro tipo IV: $h[n]=\{1,\;2,\;-1,\;1,\;-2,\;-1\}$
    \begin{solution}
       $h[n]$ es de tipo IV pues tiene simetría impar y orden impar:
        \begin{equation}
            h[n]=-h[M-n]\text{ (simetría impar)},\qquad M=5 \text{ (impar)}
        \end{equation}
        Se obtiene la respuesta de amplitud y ángulo como en los puntos anteriores:
        \begin{gather}
            h[n]=\{1,\;2,\;-1,\;1,\;-2,\;-1\}\\
            \downarrow\text{TFTD}\notag\\
            H(e^{j\omega})=1 +2e^{-j\omega} -e^{-2j\omega}  +e^{-3j\omega} -2e^{-4j\omega} -e^{-5j\omega}
        \end{gather}
        Factorizando por $e^{-\frac{5}{2}j\omega}=e^{-\frac{M}{2}j\omega}$:
        \begin{align}
            H(e^{j\omega})&=\left(e^{\frac{5}{2}j\omega} +2e^{\frac{3}{2}j\omega} -e^{\frac{1}{2}j\omega}  +e^{-\frac{1}{2}j\omega} -2e^{-\frac{3}{2}j\omega} -e^{-\frac{5}{2}j\omega}\right)e^{-\frac{5}{2}j\omega}\\
            &=\left[e^{\frac{5}{2}j\omega} -e^{-\frac{5}{2}j\omega} +2(e^{\frac{3}{2}j\omega} -e^{-\frac{3}{2}j\omega}) -(e^{\frac{1}{2}j\omega}  -e^{-\frac{1}{2}j\omega})\right]e^{-\frac{5}{2}j\omega}
        \end{align}
        Considerando $e^{j\theta}-e^{-j\theta}=2j\sin(\theta)$, y $j=e^{j\pi/2}$:
        \begin{align}
            H(e^{j\omega})&=\left[2j\sin\left(\frac{5}{2}\omega\right) +4j\sin\left(\frac{3}{2}\omega\right) -2j\sin\left(\frac{1}{2}\omega\right)\right]e^{-\frac{5}{2}j\omega}\\
            &=\left[2\sin\left(\frac{5}{2}\omega\right) +4\sin\left(\frac{3}{2}\omega\right) -2\sin\left(\frac{1}{2}\omega\right)\right]e^{j\left(\frac{\pi}{2}-\frac{5}{2}\omega\right)}\\
            &=A(e^{j\omega})e^{j\Psi(\omega)},\qquad\Psi(\omega)=\frac{\pi}{2}-\frac{5}{2}\omega=\frac{\pi}{2}-\frac{M}{2}\omega
        \end{align}

        \begin{minted}[fontsize=\footnotesize]{matlab}
% --- Tipo IV ---
h = [1, 2, -1, 1, -2, -1];     M = 5; % M impar, simetría impar
A = 2*sin(5/2*w) + 4*sin(3/2*w) - 2*sin(1/2*w);
psi = pi/2-M/2*w; 
        \end{minted}

        \includegraphics[width=14cm]{imagenes/9.2d.png}
        

    \end{solution}
\end{parts}

\question Diseñe un filtro FIR pasabajos que cumpla:
\begin{itemize}
    \item Frecuencia de borde de banda de paso $\omega_p=0.3\pi$
    \item Ripple de banda de paso $A_p=0.5$ dB
    \item Frecuencia de borde de rechaza banda $\omega_s=0.5\pi$
    \item Atenuación de rechaza banda $A_s=50$ dB
\end{itemize}
\begin{parts}
    \part Determinar la respuesta a impulso de un filtro ideal que cumpla con las
    especificaciones.
    \begin{solution}
        El filtro pasabajo ideal tiene la forma:
        \begin{align}
            &H(e^{j\omega})=\begin{cases}
                e^{-j\omega\alpha}, & |\omega|<\omega_c\\
                0, & \omega_c<|\omega|<\pi
            \end{cases}
            & &h[n]=\frac{\sin\left(\omega_c[n-\alpha]\right)}{\pi[n-\alpha]}
        \end{align}
        Este filtro tiene una banda de transición $\Delta\omega=\omega_s-\omega_p=0$, no tiene ripple ($A_p=0$ dB)
        y atenuación infinita ($A_s=\infty$ dB). $\alpha$ no tiene efectos en ninguno de estos elementos.

        Escogiendo la frecuencia de corte como el punto medio de la banda de transición requerida:
        $\omega_c=(\omega_p+\omega_s)/2=(0.3\pi+0.5\pi)/2=0.4\pi$. Y $\alpha=0$:
        \begin{align}
            &H(e^{j\omega})=\begin{cases}
                1, & |\omega|<0.4\pi\\
                0, & 0.4\pi<|\omega|<\pi
            \end{cases}
            & &h[n]=\frac{\sin\left(0.4\pi n\right)}{\pi n}
        \end{align}

        \begin{minted}[fontsize=\footnotesize]{matlab}
n = -30:30;

wc = 0.4*pi;
a = 0;
h_ideal = sin(wc*(n-a))./(pi* (n-a));
h_ideal(isnan(h_ideal)) = wc/pi;
        \end{minted}

        \includegraphics[width=14cm]{imagenes/9.3a.png}
    \end{solution}
    \part Utilizando una ventana rectangular, determine en Matlab un filtro FIR que cumpla con
    la especificación. Grafique la respuesta de magnitud.
    \begin{solution}
        El diseño del filtro se realiza a partir de un filtro ideal. Escogiendo la frecuencia de corte
        $\omega_c=(\omega_p+\omega_s)/2=(0.3\pi+0.5\pi)/2=0.4\pi$:
        \begin{align}
            h_i[n]=\frac{\sin\left(0.4\pi [n-\alpha]\right)}{\pi [n-\alpha]}
        \end{align}
        Para que el filtro sea FIR (respuesta a impulso finita), aplicamos una ventana rectangular de largo
        L:
        \begin{align}
            h_{LP}=h_i[n]w_R[n]=\begin{cases}
                \frac{\sin\left(0.4\pi [n-M/2]\right)}{\pi [n-M/2]}, & 0\leq n \leq M=L-1\\
                0, & \text{otro valor}
            \end{cases}
        \end{align}
        La decisión del largo L se realiza de acuerdo a la banda de transición.
        De la tabla del punto 3(c), la banda de transición
        con una ventana rectangular es $\Delta\omega=1.8\pi/L$ (columna \textbf{Exact} $\Delta\omega$).
        Con las especificaciones $\omega_p=0.3\pi$ y $\omega_s=0.5\pi$, la banda de transición
        deseada es $\Delta\omega=\omega_s-\omega_p=0.2\pi$. Por lo tanto:
        \begin{align}
            &\Delta\omega=1.8\pi/L=0.2\pi & &\longrightarrow & L=9, M=8
        \end{align}
        
        Con ese valor de M, se determina la respuesta a impulso del filtro:
        \begin{align}
            h_{LP}=h_i[n]w_R[n]=\begin{cases}
                \frac{\sin\left(0.4\pi [n-4]\right)}{\pi [n-4]}, & 0\leq n \leq 8=M=L-1\\
                0, & \text{otro valor}
            \end{cases}
        \end{align}
        Esto se realiza en Matlab y se comprueba si el filtro cumple con los requerimientos:
        \begin{minted}[fontsize=\footnotesize]{matlab}
%% 3 (b)

% Especificaciones
wp = 0.3*pi; ws = 0.5*pi;
delta_p = 0.028; delta_s = 0.0032;

M = 8;
wc = 0.4*pi;
n = 0:M;

h_rect = sin(wc*(n-M/2))./(pi * (n-M/2));
h_rect(isnan(h_rect)) = wc/pi;

w = linspace(0, pi, 1000);
H_rect = freqz(h_rect, 1, w);

subplot(1,2,1)
stem(n,h_rect,'filled', LineWidth=2);
xlabel('n');
ylabel('h_{LP}[n]');
title('Respuesta a impulso');
grid on;


subplot(1,2,2)
plot(w, abs(H_rect), LineWidth=2)
xline(wp, '--', '\omega_p', LabelOrientation="horizontal", LineWidth=2)
xline(ws, '--', '\omega_s', LabelOrientation="horizontal", LineWidth=2)
yline(1+delta_p, '--', '1+\delta_p', LineWidth=2)
yline(1-delta_p, '--', '1-\delta_p', LineWidth=2)
yline(delta_s, '--', '\delta_s', LineWidth=2)
xlabel('Frecuencia \omega');
ylabel('Magnitud |H(e^{j\omega})|');
title('Respuesta de magnitud');
grid on;
        \end{minted}

        \includegraphics[width=14cm]{imagenes/9.3b.png}

        Al aplicar la ventana sobre la respuesta al filtro ideal, nunca se obtiene
        un filtro perfecto. En particular la ventana rectangular produce un ripple
        alto en la banda de paso y rechazo, pero tiene una buena banda de transición
        con una ventana pequeña. Para cumplir con todos los requerimientos es necesario
        escoger otra ventana.
    \end{solution}
    \part Repita seleccionando una ventana más apropiada de la tabla.\\
    \includegraphics[width=15cm]{imagenes/9.p3.png}\\
    Compare la respuesta de magnitud respecto al punto anterior.
    \begin{solution}
        Aqui utilizamos el procedimiento de diseño completo.
        \paragraph{1. Conversion de especificaciones} Convertimos de especificaciones
        relativas\\$A_p=0.5$ dB, $A_s=50$ dB a absolutas $\delta_p$ y $\delta_s$:
        \begin{align}
            \delta_p&=\frac{10^{A_p/20}-1}{10^{A_p/20}+1} &
            \delta_s&=\frac{1+\delta_p}{10^{A_s/20}}\\
            &=\frac{10^{0.5/20}-1}{10^{0.5/20}+1} &
            &=\frac{1+0.028}{10^{50/20}}\\
            &=0.028 & &=0.032
        \end{align}

        \paragraph{2. Frecuencia de corte} Escogemos la frecuencia de corte en la mitad
        de la banda de transición $\omega_c=(\omega_p+\omega_s)/2=(0.3\pi+0.5\pi)/2=0.4\pi$

        \paragraph{3. Ripple mínimo y atenuación equivalente} El criterio para escoger una
        ventana u otra es la atenuación que la ventana logra.
        
        Reducimos la especificación a un valor de ripple: 
        \begin{align}
            \delta=\min\{\delta_p,\;\delta_s\}=0.028
        \end{align}
        
        Encontramos la atenuación equivalente:
        \begin{align}
            A&=20\log\left(\frac{1}{\delta}\right)=-20\log(\delta)\\
            &=49.89\text{ dB}
        \end{align}

        \paragraph{4. Selección de ventana} Escogemos de la tabla la ventana que tenga la menor
        atenuación $A_s$ que sea mayor a $A=49.89$ calculado. En este caso la ventana
        de Hamming cumple con esta condición $53>49.89$.

        \paragraph{5. Determinación de longitud de la ventana} Ubicamos en
        la columna \textbf{Exact} $\Delta\omega$ la banda de transición para la ventana
        escogida, e igualamos al $\Delta\omega$ de las especificaciones:
        \begin{align}
            \Delta\omega&=6.6\pi/L=\omega_s-\omega_s\\
            &=6.6\pi/L=0.2\pi\quad\longrightarrow\quad L=33=M+1
        \end{align}

        \paragraph{6. Filtro ideal} Se determina la respuesta a impulso del filtro ideal
        \begin{equation}
            h_i[n]=\frac{\sin(\omega_c[n-M/2])}{\pi[n-M/2]}=\frac{\sin(0.4\pi[n-16])}{\pi[n-16]}
        \end{equation}
        
        \paragraph{7. Respuesta a impulso aplicando ventana} La respuesta a impulso del
        filtro diseñado se determina aplicando la ventana escogida:
        \begin{align}
            h[n]=h_i[n]w[n]
        \end{align}
        En este caso $w[n]$ es la ventana Hamming que se escogió de orden $M=32$.
        \begin{align}
            w[n]=\begin{cases}
                0.54-0.46\cos(2\pi n/M), & 0\leq n \leq M\\
                0, & \text{otro valor}
            \end{cases}
        \end{align}

        \paragraph{8. Comprobación} Se comprueba en Matlab si el filtro cumple con las especificaciones.
        De no ser así, se escoge un $M$ mayor y se vuelve al paso 6 de diseño.

        \begin{minted}[fontsize=\footnotesize]{matlab}
%% 3 (c)

% Especificaciones
wp = 0.3*pi; ws = 0.5*pi;
delta_p = 0.028; delta_s = 0.0032;

M = 32; 
wc = 0.4*pi;
n_hamm = 0:M;

h_ideal = ( sin(wc*(n_hamm-M/2))./(pi * (n_hamm-M/2)) );
h_ideal(isnan(h_ideal)) = wc/pi;
h_hamming = h_ideal.*hamming(M+1)';


w = linspace(0, pi, 1000);
H_hamming = freqz(h_hamming, 1, w);

subplot(1,2,1)
stem(n_hamm,h_hamming,'filled', LineWidth=2);
xlabel('n');
ylabel('h_{LP}[n]');
title('Respuesta a impulso');
grid on;

subplot(1,2,2)
plot(w, abs(H_hamming), LineWidth=2)
xline(wp, '--', '\omega_p', LabelOrientation="horizontal")
xline(ws, '--', '\omega_s', LabelOrientation="horizontal")
yline(1+delta_p, '--', '1+\delta_p')
yline(1-delta_p, '--', '1-\delta_p')
yline(delta_s, '--', '\delta_s')
xlabel('Frecuencia \omega');
ylabel('Magnitud |H(e^{j\omega})|');
title('Respuesta de magnitud');
grid on;
        \end{minted}

        \includegraphics[width=14cm]{imagenes/9.3c.png}
        
        \includegraphics[width=10cm]{imagenes/9.3c.2.png}
        
        El filtro cumple con todas las especificaciones por lo que no es necesario
        aumentar $M$

        Al comparar los filtros:
        
        \includegraphics[width=14cm]{imagenes/9.3c.3.png}

        Al aplicar la ventana de Hamming se obtiene un ripple mucho menor. Por otro
        lado se debe ocupar una ventana más grande para lograr una buena banda de transición, por lo que es más costoso implementar
        este filtro. La ventana rectangular por otro lado logra una banda de transición
        relativamente buena con una ventana pequeña.

    \end{solution}
\end{parts}


\end{questions}

%%%%%%%%%%%%%%%%%%%%%%%%%%%%%%%%
%% -- Fin del Documento -- %%
%%%%%%%%%%%%%%%%%%%%%%%%%%%%%%%%
\end{document}