\documentclass[12pt]{exam}
\usepackage{../preamble}

% disable para esconder
\usepackage[enable]{easy-todo}

% Comentar para esconder soluciones
\printanswers

%%%%%%%%%%%%%%%%%%%%%%%%%%%%%%%%
%% -- Inicio del Documento -- %%
%%%%%%%%%%%%%%%%%%%%%%%%%%%%%%%%
\begin{document}
\begin{center}
    {\Large
        Ayudantía 11 - Procesamiento Digital de Señales
    }
\end{center}

\begin{questions}
\question Considere un filtro pasa-bajo análogo Butterworth $H_c(s)$ de cuarto
orden con una frecuencia de corte (3 dB) en 20 Hz. Determine y bosqueje los polos de $H_c(s)$.
\begin{solution}
    En general, el filtro de Butterworth de orden $N$ cumple
    \begin{equation}
        \left|H(e^{j\omega})\right|^2=\frac{1}{1+\left(\Omega/\Omega_c\right)^{2N}}
    \end{equation}
    Los polos del filtro cumplen con:
    \begin{align}
        1+\left(\Omega/\Omega_c\right)^{2N}&=0   \\
        \left(\Omega/\Omega_c\right)^{2N}&=-1   \\
        \left(\Omega/\Omega_c\right)^{2N}&=e^{j(2k-1)\pi} 
    \end{align}
\end{solution}

\question Con ayuda de Matlab, diseñe un pasa-bajo análogo Butterworth con las
especificaciones:
\begin{itemize}
    \item Frecuencia borde de banda de paso: $F_p=50$ Hz
    \item Ripple en banda de paso: $A_p=0.5$ dB
    \item Frecuencia borde de banda de rechazo $F_s=80$ Hz
    \item Atenuación en banda de rechazo: $A_s=45$ dB
\end{itemize}
\begin{parts}
    \part Obtenga el orden $N$ del filtro.
    \part Obtenga la frecuencia de corte $\Omega_c$ para el diseño del filtro.
    \part Obtener en Matlab los polos.
    \part Gragicar magnitud y fase.
    \part Obtener un filtro digital mediante transformación impulso invariante, 
    y graficar magnitud y fase
\end{parts}

\question Con ayuda de Matlab y usando transformación bilinear, diseñe un filtro
pasa-bajo de Chebyshev I que cumpla:
\begin{itemize}
    \item Frecuencia borde de banda de paso: $\omega_p=0.2\pi$
    \item Ripple en banda de paso: $A_p=1$ dB
    \item Frecuencia borde de banda de rechazo $\omega_s=0.3\pi$
    \item Atenuación en banda de rechazo: $A_s=60$ dB
\end{itemize}
\begin{parts}
    \part Obtener las frecuencias $\Omega_p$ y $\Omega_s$ equivalentes pare el
    diseño del filtro.
    \part Obtener $\epsilon$ y $A$ para el diseño.
    \part Obtener Orden del filtro $N$ y frecuencia de corte $\Omega_c$.
    \part Obtener en Matlab la función de transferencia y el filtro digital
    usando \texttt{bilinear}, y graficar magnitud y fase de este.
\end{parts}


\end{questions}
%\tableofcontents
%\listoffigures
%\listoftables
%\listoftodos

%%%%%%%%%%%%%%%%%%%%%%%%%%%%%%%%
%% -- Fin del Documento -- %%
%%%%%%%%%%%%%%%%%%%%%%%%%%%%%%%%
\end{document}