\documentclass[12pt]{exam}
\usepackage{../preamble}

% disable para esconder
\usepackage[enable]{easy-todo}

% Comentar para esconder soluciones
\printanswers

%%%%%%%%%%%%%%%%%%%%%%%%%%%%%%%%
%% -- Inicio del Documento -- %%
%%%%%%%%%%%%%%%%%%%%%%%%%%%%%%%%
\begin{document}
\begin{center}
    {\Large
        Ayudantía 11 - Procesamiento Digital de Señales
    }
\end{center}

\begin{questions}
\question Considere un filtro pasa-bajo análogo Butterworth $H_c(s)$ de cuarto
orden con una frecuencia de corte (3 dB) en 20 Hz. Determine y bosqueje los polos de $H_c(s)$.
\begin{solution}
    \paragraph{Derivación}
    En general, el filtro de Butterworth de orden $N$ cumple
    \begin{equation}
        \left|H(\Omega)\right|^2=\frac{1}{1+\left(\Omega/\Omega_c\right)^{2N}}
    \end{equation}
    Considerando $s=j\Omega$, $-1=e^{j(2k-1)\pi}$, y $j=e^{j\pi/2}$ los $N$ polos del filtro cumplen con:
    \begin{align}
        1+\left(s/j\Omega_c\right)^{2N}&=0   \\
        \left(s/j\Omega_c\right)^{2N}&=-1   \\
        \left(s/j\Omega_c\right)^{2N}&=e^{j(2k-1)\pi}   \\
        s/j\Omega_c&=e^{j\frac{2k-1}{2N}\pi}   \\
        s&=j\Omega_ce^{j\frac{2k-1}{2N}\pi}   \\
        s&=\Omega_ce^{j\left(\frac{\pi}{2}+\frac{2k-1}{2N}\pi\right)}
    \end{align}
    Esta ecuación tiene $2N$ soluciones $k=1,\;2,\dots,\;2N$. Sin embargo el
    filtro tiene $N$ polos. Como se desea un filtro con respuesta impulso estable,
    se escogen los polos con $k=1,\;2,\dots,\;N$, que se encuentran en el SP
    izquierdo.


    \paragraph{Polos del filtro Butterworth y F. de T.} Así, dados $\Omega_c$ (frecuencia
    de corte) y $N$ (orden del filtro), los polos y función de transferencia son:
    \begin{align}
        &\boxed{p_k=\Omega_ce^{j\left(\frac{\pi}{2}+\frac{2k-1}{2N}\pi\right)}}
        & &\boxed{H(s)=\frac{\Omega_c^N}{(s-p_1)(s-p_2)\dots(s-p_N)}}
        & &\boxed{k=1,\;2,\dots,\;N}
    \end{align}
    Los polos cubren el lado izquierdo de una circunferencia de radio $\Omega_c$.
    
    \paragraph{Ahora sí, el problema}
    Con orden $N=4$, y frecuencia de corte $F_c=20$ Hz, tenemos
    $\Omega_c=2\pi F_c=40\pi$ y $2N=8$. Por lo tanto:
    \begin{gather}
            p_k=40\pi e^{j\left(\frac{\pi}{2}+\frac{2k-1}{8}\pi\right)}
            =\begin{cases}
            p_1=40\pi e^{j5\pi/8}   \\
            p_2=40\pi e^{j7\pi/8}   \\
            p_3=40\pi e^{j9\pi/8}   \\
            p_4=40\pi e^{j11\pi/8}
            \end{cases}
    \end{gather}
    Y la función de transferencia es:
    \begin{equation}
        H(s)=\frac{(40\pi)^4}{(s-p_1)(s-p_2)(s-p_3)(s-p_4)}
    \end{equation}
    
    Para bosquejar los polos, se dibuja una circunferencia de radio
    $\Omega_c=40\pi$ como guía, y se ubican los polos en los ángulos que indican
    las exponenciales complejas. Los polos siempre ``cubren'' la mitad izquierda
    de la circunferencia

    \includegraphics[width=14cm]{imagenes/11.1.png}
\end{solution}
    
\question Con ayuda de Matlab, diseñe un pasa-bajo análogo Butterworth con las
especificaciones:
\begin{itemize}
    \item Frecuencia borde de banda de paso: $F_p=50$ Hz
    \item Ripple en banda de paso: $A_p=0.5$ dB
    \item Frecuencia borde de banda de rechazo $F_s=80$ Hz
    \item Atenuación en banda de rechazo: $A_s=45$ dB
\end{itemize}
\begin{parts}
    \part Obtenga el orden $N$ del filtro.

    \begin{solution}
        Se tiene
        \begin{gather}
            F_p=50\text{ Hz} \longrightarrow \Omega_p=2\pi F_p =100\pi  \\
            F_s=80\text{ Hz} \longrightarrow \Omega_s=2\pi F_s =160\pi
        \end{gather}
        La fórmula para hallar el orden $N$ del filtro de Butterworth es:
        \begin{align}
            &\boxed{N\geq \frac{\ln\beta}{\ln\alpha}}   
            & &\boxed{\alpha=\frac{\Omega_s}{\Omega_p}} 
            & &\boxed{\beta=\frac{\sqrt{10^{A_s/10}-1}}{\sqrt{10^{A_p/10}-1}}}
        \end{align}
        Reemplazando:
            \begin{align}
                \alpha&=\frac{160\pi}{100\pi}
                & \beta&=\frac{\sqrt{10^{45/10}-1}}{\sqrt{10^{0.5/10}-1}}
                & N&\geq\frac{\ln1.6}{\ln509.7} \\
                %
                &=1.6
                & &=509.7
                & &\geq13.26   \\
                %
                &
                & &
                & N&=14
            \end{align}
    \end{solution}

    \part Obtenga la frecuencia de corte $\Omega_c$ para el diseño del filtro.
    
    \begin{solution}
        Una vez obtenido el orden $N$, se utiliza
        \begin{equation}
            \boxed{
            \Omega_p\left(10^{A_p/10}-1\right)^{-1/2N}
            \leq \Omega_c \leq
            \Omega_s\left(10^{A_s/10}-1\right)^{-1/2N}
            }
        \end{equation}
        Para escoger la frecuencia de corte. Cualquier valor dentro del
        intervalo sirve. Al reemplazar:
        \begin{align}
            100\pi\left(10^{45/10}-1\right)^{-1/28}
            &\leq \Omega_c \leq
            160\pi\left(10^{0.5/10}-1\right)^{-1/28}    \\
            %
            338.67 &\leq \Omega_c \leq 347.18
        \end{align}
        Escogiendo el límite superior $\Omega_c=347.18$, se obtiene mejor ripple
        en la banda de paso.
    \end{solution}

    \part Obtener en Matlab los polos.

    \begin{solution}
        Matlab realiza el procedimiento de diseño (encontrar $\Omega_c$ y $N$),
        mediante el comando \texttt{buttord}:
        \begin{minted}[fontsize=\footnotesize]{matlab}
Omegap = 100*pi;
Omegas = 160*pi;
Ap = 0.5;
As = 45;

[N, Omegac] = buttord(Omegap, Omegas, Ap, As, 's')

N =

    14

Omegac =

  347.1811
        \end{minted}
        
        Y se obtienen los polos con el comando \texttt{butter}:
        \begin{minted}[fontsize=\footnotesize]{matlab}
[z, p, k] = butter(N, Omegac, 's')
        \end{minted}
        Equivalente a:
        \begin{align}
            p_k&=\Omega_ce^{j\left(\frac\pi2 + \frac{2k-1}{2N}\pi\right)}
            =347.18e^{j\left(\frac\pi2 + \frac{2k-1}{28}\pi\right)},
            & &k=1,2,\dots,14
        \end{align}
        Al graficar los polos:
        
        \includegraphics[width=14cm]{imagenes/11.2c.png}
    \end{solution}

    \part Gragicar magnitud y fase.
    \begin{solution}
        Se obtiene magnitud y fase con
        \begin{minted}[fontsize=\footnotesize]{matlab}
[b, a] = zp2tf(z,p,k);
h = freqs(b,a,Omega);
h_mag = abs(h);
h_fase = angle(h);
        \end{minted}
        Al graficar:\\
        \includegraphics[width=14cm]{imagenes/11.2d.png}

        La magnitud tiene una pendiente de $-20N=-280$ dB por octava, y fase
        disminuye en $-\frac{\pi}{2}N=-7\pi$.
    \end{solution}

    \part Obtener un filtro digital mediante transformación impulso invariante, 
    y graficar magnitud y fase

    \begin{solution}
        \paragraph{Transformación impulso invariante}
        La transformación impulso invariante se obtiene realizando:
        \begin{equation}
            \boxed{h[n]=T_dh_c(nT_d),\quad T_d\text{ Periodo de muestreo}}
        \end{equation}
        Y la respuesta en frecuencia pasa a ser
        \begin{equation}
            \boxed{H(e^{j\omega})=
            \sum_{k=-\infty}^{\infty}
            H_c\left(j\frac{\omega}{T_d}+j\frac{2\pi}{T_d}k\right)}
        \end{equation}
        Es decir, hay aliasing debido al muestreo. A medida que $T_d$ decrece,
        el aliasing se vuelve despreciable.

        Las frecuencias y los polos se mapean:
        \begin{align}
            &\boxed{\omega=\Omega T_d} & &\boxed{p_k=e^{s_kT_d}}
        \end{align}
        Asumiendo que se filtran frecuencias
        hasta $500$ Hz, se escoge $F_d=1000$ Hz por el teorema de muestreo. Es
        decir $T_d=1/1000$:
        \begin{equation}
            h[n]=\frac{1}{1000}h_c\left(\frac{n}{1000}\right)
        \end{equation}
        
        En Matlab:
        \begin{minted}[fontsize=\footnotesize]{matlab}
Td = 1/1000;
[bz, az] = impinvar(b,a, 1/Td);

t = linspace(0,0.12,1000);
[ht, tout] = impulse(tf(b, a), t); 
[hn, nT]= impz(bz, az, 120, 1/Td);
        \end{minted}

        Al graficar magnitud y fase (mapeando las especificaciones según
        $\omega=\Omega/1000$):\\
        \includegraphics[width=14cm]{imagenes/11.2e.2.png}\\
        En la banda de paso y transición, la diferencia con el filtro original
        es despreciable
    \end{solution}

\end{parts}

\question Con ayuda de Matlab y usando transformación bilinear, diseñe un filtro
pasa-bajo de Chebyshev I que cumpla:
\begin{itemize}
    \item Frecuencia borde de banda de paso: $\omega_p=0.2\pi$
    \item Ripple en banda de paso: $A_p=1$ dB
    \item Frecuencia borde de banda de rechazo $\omega_s=0.3\pi$
    \item Atenuación en banda de rechazo: $A_s=60$ dB
\end{itemize}
\begin{parts}
    
    \pagebreak
    \part Obtener las frecuencias $\Omega_p$ y $\Omega_s$ equivalentes pare el
    diseño del filtro.

    \begin{solution}
        Aplicar la transformación bilinear mapea las frecuencias de acuerdo a:
        \begin{equation}
            \boxed{\omega=2\tan^{-1}\left(\frac{T_d}{2}\Omega\right)}
            \longleftrightarrow
            \boxed{\Omega=\frac{2}{T_d}\tan\left(\frac{\omega}{2}\right)}
            \quad T_d\text{ arbitrario.}
        \end{equation}
        Escogiendo $T_d=2$ a conveniencia:
        \begin{align}
            \Omega_p&=\frac{2}{T_d}\tan\left(\frac{\omega_p}{2}\right)
            & \Omega_s&=\frac{2}{T_d}\tan\left(\frac{\omega_s}{2}\right)    \\
            %
            &=\tan\left(\frac{0.2\pi}{2}\right)
            & &=\tan\left(\frac{0.3\pi}{2}\right) \\
            %
            &=0.325
            & &=0.509
        \end{align}
    \end{solution}

    \part Obtener $\epsilon$ y $A$ para el diseño.

    \begin{solution}
        Se utilizan las ecuaciones de conversion de especificaciones relativas
        a análogas:
        \begin{align}
            &\boxed{\epsilon=\sqrt{10^{A_p/10}-1}}
            & &\boxed{A=10^{A_s/20}}
        \end{align}
        Sustituyendo:
        \begin{align}
            \epsilon&=\sqrt{10^{1/10}-1}
            & A&=10^{60/20} \\
            %
            &=0.509
            & &=1000
        \end{align}
    \end{solution}

    \part Obtener orden del filtro $N$ y frecuencia de corte $\Omega_c$.

    \begin{solution}
        Se selecciona $\Omega_c=\Omega_p=0.325$

        El orden $N$ se obtiene de acuerdo a las ecuaciones de diseño
        \begin{align}
            &\boxed{
                N\geq\frac
                {\ln(\beta+\sqrt{\beta^2-1})}
                {\ln(\alpha+\sqrt{\alpha^2-1})}
            }
            & &\boxed{\alpha=\frac{\Omega_s}{\Omega_p}}
            & &\boxed{\beta=\frac{1}{\epsilon}\sqrt{A^2-1}}
        \end{align}
        Reemplazando:
        \begin{align}
            \alpha&=\frac{0.509}{0.325}
            & \beta&=\frac{1}{0.509}\sqrt{1000^2-1}
            & N&\geq\frac
                {\ln(\beta+\sqrt{\beta^2-1})}
                {\ln(\alpha+\sqrt{\alpha^2-1})} \\
            %
            &=1.5661
            & &=1968.5
            & &\geq8.12 \\
            %
            &
            & &
            & N&=9
        \end{align}
    \end{solution}

    \part Obtener en Matlab la función de transferencia y el filtro digital
    usando \texttt{bilinear}, y graficar magnitud y fase de este.

    \begin{solution}
        La selección de orden $N=9$ y frecuencia de corte $\Omega_c=0.325$ se
        comprueba en Matlab mediante
        \begin{minted}[fontsize=\footnotesize]{matlab}
Omegap = 0.325;
Ap = 1;
Omegas = 0.509;
As = 60;
[N, Omegac] = cheb1ord(Omegap, Omegas, Ap, As, 's')

N =

     9

Omegac =

    0.3250
        \end{minted}
        La función de transerencia y respuesta en frecuencia del filtro digital
        (aplicando transformación bilinear) se obtiene mediante
        \begin{minted}[fontsize=\footnotesize]{matlab}
[b, a] = cheby1(N, Ap, Omegac, 's');
[bz, az] = bilinear(b, a, 1/Td)
[hz, w]= freqz(bz, az, 1024); 

bz =

   1.0e-04 *

  Columns 1 through 7

    0.0019    0.0167    0.0667    0.1556    0.2333    0.2333    0.1556

  Columns 8 through 10

    0.0667    0.0167    0.0019

az =

  Columns 1 through 7

    1.0000   -7.5802   26.2987  -54.7173   75.1516  -70.6047   45.3539

  Columns 8 through 10

  -19.2051    4.8651   -0.5620
        \end{minted}
        Donde \texttt{bz} y \texttt{az} son numerador y denominador de la
        función de transferencia:
        \begin{equation}
            H_d(z)=\frac{b_0+b_1z^{-1}+b_2z^{-2}+...+b_9z^{-9}}
            {a_0+a_1z^{-1}+a_2z^{-2}+...+a_9z^{-9}}
        \end{equation}
        Esto es equivalente a sustituir en la función de transferencia continua
        $H_c(s)$
        por $s=\frac{2}{T_d}\frac{1-z^{-1}}{1+z^{-1}}$:
        \begin{align}
            H_c(s)\longrightarrow
            H_d(z)&=H_c\left(\frac{2}{T_d}\frac{1-z^{-1}}{1+z^{-1}}\right)  \\
            &=H_c\left(\frac{1-z^{-1}}{1+z^{-1}}\right)
        \end{align}

        Al graficar magnitud y fase:
        
        \includegraphics[width=14cm]{imagenes/12.3d.1.png}

        La magnitud cumple con las especificaciones. Al tratarse de un filtro de
        Chebyshev tipo I, el filtro exhibe ripple en la banda de paso:

        \includegraphics[width=14cm]{imagenes/12.3d.2.png}

    \end{solution}
\end{parts}


\end{questions}
%\tableofcontents
%\listoffigures
%\listoftables
%\listoftodos

%%%%%%%%%%%%%%%%%%%%%%%%%%%%%%%%
%% -- Fin del Documento -- %%
%%%%%%%%%%%%%%%%%%%%%%%%%%%%%%%%
\end{document}