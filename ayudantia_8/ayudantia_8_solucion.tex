\documentclass[12pt]{exam}
\usepackage{../preamble}

% disable para esconder
\usepackage[enable]{easy-todo}
\usepackage{fancyvrb}
\usepackage{framed}

% Comentar para esconder soluciones
\printanswers

%%%%%%%%%%%%%%%%%%%%%%%%%%%%%%%%
%% -- Inicio del Documento -- %%
%%%%%%%%%%%%%%%%%%%%%%%%%%%%%%%%
\begin{document}
\begin{center}
    {\Large
        Ayudantía 8 - Procesamiento Digital de Señales
    }
\end{center}

\begin{questions}

\question Un sistema tiempo discreto es descrito por el diagrama de flujo de
señales a continuación:\\
\includegraphics[width=15cm]{imagenes/8.1.png}

\begin{parts}
\part Determine la ecuación de diferencias que relaciona la salida $y[n]$ con la
entrada $x[n]$.

\begin{solution}\\
    \includegraphics[width=14cm]{imagenes/ayudantia8.1.drawio.png}\\
    Asignando la señal $w[n]$ en el punto indicado, podemos escribir una ecuación
    de diferencias para $y[n]$:
    \begin{align}
        y[n]&= 3(2x[n] + w[n]) + 6w[n-1]\\
            &= 6x[n] + 3w[n] + 6w[n-1]
    \end{align}

    Mirando desde $w[n]$ hacia atrás escribimos una ecuación de diferencias para
    $w[n]$:
    \begin{equation}
        w[n]=x[n]+\frac{1}{3}w[n-1]
    \end{equation}

    Así formamos un sistema de ecuaciones de diferencias que describe el sistema:
    \begin{equation}
        \begin{cases}
            y[n] = 6x[n] + 3w[n] + 6w[n-1] \\
            w[n] = x[n] + \frac{1}{3}w[n-1]
        \end{cases}
    \end{equation}

    Sin embargo, nosotros queremos una ecuación de diferencias que relacione
    $y[n]$ con $x[n]$ sin la variable intermedia $w[n]$. En este caso no podemos
    reemplazar en la primera ecuación el $w[n]$, pues aparece el término $w[n-1]$
    y el problema no se resuelve.
    
    Aplicando transformada Z, desarrollamos para reemplazar $W(z)$:
    \begin{gather}
        \begin{cases}
            Y(z) = 6X(z) + W(z)(3+6z^{-1})  \\
            W(z) = X(z) + \frac{1}{3}z^{-1}W(z)
            \longrightarrow W(z) = \frac{X(z)} {1-\frac{1}{3}z^{-1}}
        \end{cases}\\
        \qquad\downarrow\nonumber\\
        \begin{align}
        Y(z)&=6X(z) + X(z)\frac{3+6z^{-1}}{1-\frac{1}{3}z^{-1}}\\
        Y(z)\left(1-\frac{1}{3}z^{-1}\right)&=X(z)\left(6-2z^{-1}\right) + X(z)\left(3+6z^{-1}\right)\\
        Y(z)&=X(z)(9+4z^{-1}) +\frac{1}{3}Y(z)z^{-1}\\
        &\downarrow \mathcal{Z}^{-1}\notag\\
        y[n]&=9x[n]+4x[n-1] + \frac{1}{3}y[n-1]
        \end{align}
    \end{gather}
\end{solution}

\part Determine la respuesta a impulso del sistema.
\begin{solution}
    A partir del paso (6) de la pregunta anterior:
    \begin{align}
        Y(z)&=6X(z) + X(z)\frac{3+6z^{-1}}{1-\frac{1}{3}z^{-1}}\\
        H(z)&=6 +\frac{3+6z^{-1}}{1-\frac{1}{3}z^{-1}}\\
        H(z)&=6 -18\frac{-\frac{1}{6}-\frac{1}{3}z^{-1}}{1-\frac{1}{3}z^{-1}}\\
        H(z)&=6  -18\frac{(1-\frac{1}{3}z^{-1})-\frac{7}{6}}{1-\frac{1}{3}z^{-1}}\\
        H(z)&=6 - 18 - 18\frac{-\frac{7}{6}}{1-\frac{1}{3}z^{-1}}\\ 
        H(z)&=-12 +\frac{21}{1-\frac{1}{3}z^{-1}}\\ 
        &\downarrow \mathcal{Z}^{-1}\notag\\
        h[n]&= -12\delta[n] + 21\left(\frac{1}{3}\right)^nu[n]
    \end{align}
\end{solution}

\end{parts}

\question Considere el sistema tiempo discreto dado por
$$
    y[n] = 3\sum_{m=0}^{2}\left(\frac{1}{3}\right)^mx[n-m]
    + \sum_{m=1}^{3}\left(\frac{1}{2}\right)^my[n-m]
$$
Determine y dibuje las siguientes estructuras:
\begin{parts}
    \part Forma directa I (normal)
    \begin{solution}
        La forma directa I consiste en concatenar la estructura que da los
        ceros del sistema (términos $x$) y luego la estructura que otorga los polos del sistema (términos $y$).
        
        Desarrollando la ecuación de diferencias:
        \begin{align}
            \begin{split}
            y[n]&=3\left(x[n]+\frac{1}{3}x[n-1]+\frac{1}{9}x[n-1]\right)\\
            &+ \frac{1}{2}y[n-1] +\frac{1}{4}y[n-2] +\frac{1}{8}y[n-3]
            \end{split}\\\notag\\
            \begin{split}
            y[n]&=3x[n]+x[n-1]+\frac{1}{3}x[n-1]\\
            &+ \frac{1}{2}y[n-1] +\frac{1}{4}y[n-2] +\frac{1}{8}y[n-3]
            \end{split}
        \end{align}

        Primero nos ocupamos de la expresión $3x[n]+x[n-1]+\frac{1}{3}x[n-1]$. Dibujando
        una rama de la cual se extrae la señal $x[n]$ con los retardos que necesitamos: 

        \includegraphics[width=10cm]{imagenes/8.2a.1.png}
        
        \pagebreak
        Se multiplican por los coeficientes de la expresión y se suman (el coeficiente $1$ puede
        no escribirse explícitamente):\\
        \includegraphics[width=14cm]{imagenes/8.2a.2.png}

        Luego nos ocupamos de la expresión $\frac{1}{2}y[n-1] +\frac{1}{4}y[n-2] +\frac{1}{8}y[n-3]$.
        
        Se extiende la estrucura y se dibuja una rama de
        donde se extrae la señal $y[n]$ con los retardos necesarios:

        \includegraphics[width=14cm]{imagenes/8.2a.3.png}

        \pagebreak
        Estos se multiplican por los coeficientes de la expresión y se suman
        \textbf{hacia atrás}, dando como resultado la estructura que describe
        el sistema:

        \includegraphics[width=14cm]{imagenes/8.2a.4.png}
        
        Observemos que si obtenemos la función de transferencia:
        \begin{equation}
            H(z)=\frac{3+z^{-1}+\frac{1}{3}z^{-2}}{1-\frac{1}{2}z^{-1}-\frac{1}{4}z^{-2}-\frac{1}{8}z^{-3}}
        \end{equation}
        Se nota que igualmente se puede obtener el diagrama de flujos de señales a
        partir de ella, fijándonos en el numerador y denominador:
        \begin{equation}
            H(z)=\frac{\overbrace{3+z^{-1}+\frac{1}{3}z^{-2}}^{\text{coeficientes para }x[n-d]}}{1\underbrace{-\frac{1}{2}z^{-1}-\frac{1}{4}z^{-2}-\frac{1}{8}z^{-3}}_{\text{coeficientes para y[n-d]}}}
        \end{equation}

        El exponente de $z$ indica el retardo y el coeficiente que lo acompaña por 
        cuanto se debe multiplicar. El signo de los coeficientes del denominador se invierte.
    \end{solution}

        \pagebreak
    \part Forma directa II (normal)
    \begin{solution}
        La forma directa II consiste en invertir el orden en que se realizan
        las operaciones de la forma directa I.

        Notemos que el sistema se puede escribir como:
        \begin{align}
            H(z)&=\frac{3+z^{-1}+\frac{1}{3}z^{-2}}{1-\frac{1}{2}z^{-1}-\frac{1}{4}z^{-2}-\frac{1}{8}z^{-3}}\\
            &=\left(3+z^{-1}+\frac{1}{3}z^{-2}\right)\cdot\frac{1}{1-\frac{1}{2}z^{-1}-\frac{1}{4}z^{-2}-\frac{1}{8}z^{-3}}\\
            &=H_1(z)\cdot H_2(z)
        \end{align}

        Esto se observa en el diagrama de flujos (forma directa I), como dos sistemas
        en cascada:

        \includegraphics[width=14cm]{imagenes/8.2b.1}

        Como $H_1(z)H_2(z)=H_2(z)H_1(z)$, podemos invertir el orden en que se aplican
        los sistemas en el diagrama:

        \includegraphics[width=13cm]{imagenes/8.2b.2}\\
        Al realizar aquello, vemos que esta señal intermedia $w[n]$, se retarda
        en dos ramas distintas a pesar de que es idéntica. Se pueden combinar estas
        ramas en una:

        \includegraphics[width=13cm]{imagenes/8.2b.3}
    \end{solution}
    \pagebreak
    \part Forma directa I (transpuesta)
    \begin{solution}
        La forma directa I transpuesta, como dice el nombre corresponde a transponer
        el diagrama de la forma directa I. Recordemos que la forma directa I es:\\
        \includegraphics[width=14cm]{imagenes/8.2c.1.png}\\
        Y procedemos a transponer el diagrama. Transponer el diagrama consiste en invertir todas las flechas e intercambiar
        la entrada y la salida ($x[n]$ e $y[n]$):\\
        \includegraphics[width=14cm]{imagenes/8.2c.2.png}\\
        Dando como resultado la forma directa I transpuesta. Si se quiere, se puede
        reflejar el diagrama como un espejo para que quede la entrada al lado izquierdo
        y la salida al lado derecho:\\
        \includegraphics[width=14cm]{imagenes/8.2c.3.png}\\
    \end{solution}
    \part Forma directa II (transpuesta)
    \begin{solution}
        La forma directa II transpuesta se puede obtener:
        \begin{itemize}
            \item Transponiendo la forma directa II normal
            \item Invirtiendo el orden de aplicación de la forma directa I transpuesta.
        \end{itemize}
        
        \pagebreak
        Partiendo de la forma directa I transpuesta:\\
        \includegraphics[width=14cm]{imagenes/8.2d.1.png}\\

        Si invertimos el orden en que se aplican los sistemas:\\
        \includegraphics[width=14cm]{imagenes/8.2d.2.png}\\

        Esta estructura representa la expresión
        \begin{align}
            Y(z)=3X(z) + X(z)z^{-1} + \frac{1}{3}X(z)z^{-1} + \frac{1}{2}Y(z)z^{-1}+ \frac{1}{4}Y(z)z^{-2}+ \frac{1}{8}Y(z)z^{-3}
        \end{align}
        Que es equivalente a:
        \begin{align}
            Y(z)=3X(z) + \left(X(z)+\frac{1}{2}Y(z)\right)z^{-1} + \left(\frac{1}{3}X(z)+\frac{1}{4}Y(z)\right)z^{-2} + \frac{1}{8}Y(z)z^{-3}
        \end{align}
        Es decir, podemos unir la estructura por la rama del medio:\\
        \includegraphics[width=14cm]{imagenes/8.2d.3.png}\\

        Es relativamente sencillo observar que se puede llegar a este diagrama
        directamente de la función de transferencia o ecuación de diferencias
        \begin{equation}
            H(z)=\frac{\overbrace{3+z^{-1}+\frac{1}{3}z^{-2}}^{\text{coeficientes para }x[n-d]}}{1\underbrace{-\frac{1}{2}z^{-1}-\frac{1}{4}z^{-2}-\frac{1}{8}z^{-3}}_{\text{coeficientes para y[n-d]}}}
        \end{equation}

        De hecho este diagrama es muy similar a la forma directa I, con la diferencia
        de que aquí se toman las señales $x[n]$ e $y[n]$, se multiplican, se suman,
        y luego se aplican los retardos correspondientes.

    \end{solution}
\end{parts}

\pagebreak
\question Un sistema FIR esta dado por
$$
    H(z) = 1 + 1.61z^{-1} + 1.74z^{-2} + 1.61z^{-3} + z^{-4}
$$
Determine y dibuje las siguientes estructuras:
\begin{parts}
    \part Forma cascada (utilice el comando \texttt{tf2sos} para hallar los
    coeficientes de segundo orden).
    \begin{solution}\\
        Ingresando el sistema a Matlab como numerador y denominador
        \begin{equation}
            H(z) = \frac{1 + 1.61z^{-1} + 1.74z^{-2} + 1.61z^{-3} + z^{-4}}{1}
        \end{equation}
        y aplicando \texttt{tf2sos}:

        \begin{minted}[fontsize=\footnotesize]{matlab}
%% 3 (a)

b = [1, 1.61, 1.74, 1.61, 1];
a = [1];

sos = tf2sos(b, a)
        \end{minted}
        Se obtiene:
        \begin{minted}[fontsize=\footnotesize]{matlab}
sos =

    1.0000   -0.1479    1.0000    1.0000         0         0
    1.0000    1.7579    1.0000    1.0000         0         0 
        \end{minted}
        La matriz \texttt{sos} representa una factorización del sistema original
        en dos sistemas de segundo orden. Las primeras tres columnas representan los numeradores,
        y las últimas tres los denominadores:

        \begin{align}    
            H(z) &= 1 + 1.61z^{-1} + 1.74z^{-2} + 1.61z^{-3} + z^{-4}\\
            &=\frac{1-0.14z^{-1}+z^{-2}}{1+0z^{-1}+0z^{-2}}\cdot\frac{1+1.75z^{-1}+z^{-2}}{1+0z^{-1}+0z^{-2}}\\
            &=\left(1-0.14z^{-1}+z^{-2}\right)\left(1+1.75z^{-1}+z^{-2}\right)
        \end{align}
        Dibujando estos dos sistemas obtenemos:\\
        \includegraphics[width=14cm]{imagenes/8.3a.1.png}

        Concatenándolos uno tras otro obtenemos el diagrama de flujos en forma
        cascada que implementa el sistema completo:\\
        \includegraphics[width=14cm]{imagenes/8.3a.2.png}
        
    \end{solution}
    \part Forma fase lineal
    \begin{solution}
        Dado que tenemos un sistema de fase lineal con
        \begin{align}
            H(z) &= 1 + 1.61z^{-1} + 1.74z^{-2} + 1.61z^{-3} + z^{-4}\\
            &\downarrow\mathcal{Z}^{-1}\notag\\
            h[n] &= \delta[n] + 1.61\delta[n-1] + 1.74\delta[n-2] + 1.61\delta[n-3] + \delta[n-4]
        \end{align}
        Es decir, $h[n]=h[4-n]$ ($0\leq n \leq 4$), podemos dibujar una estructura en forma fase
        lineal.

        Observemos que:
        \begin{align}
            H(z) &= 1 + 1.61z^{-1} + 1.74z^{-2} + 1.61z^{-3} + z^{-4}\\
               \frac{Y(z)}{X(z)} &= (1+z^{-4}) + 1.61(z^{-1} +z^{-3})+ 1.74z^{-2}
        \end{align}
        Pasándolo a ecuación de diferencias:
        \begin{equation}
            y[n]=(1)\left(x[n]+x[n-4]\right) + 1.61\left(x[n-1]+x[n-3]\right) + 1.74x[n-2]
        \end{equation}
        Esta es nuestra guía para dibujar el diagrama. Comenzamos sumando lo indicado
        entre paréntesis:\\
        \includegraphics[width=14cm]{imagenes/8.3b.1.png}

        Y posteriormente multiplicamos cada suma por su coeficiente correspondiente, y los
        sumamos:\\
        \includegraphics[width=14cm]{imagenes/8.3b.2.png}
    \end{solution}
\end{parts}

\question Un sistema IIR viene dado por:
$$
    H(z) = \frac{3.96 + 6.36z^{-1} + 8.3z^{-2} + 4.38z^{-3} + 2.07z^{-4}}
        {1 +0.39z^{-1} -0.93z^{-2} -0.33z^{-3} +0.34z^{-4}}
$$
Utilizando Matlab determine:
\begin{parts}
\part Una estructura en forma paralela
\begin{solution}
    Se utiliza el comando \texttt{residuez}. Al aplicarlo sobre el sistema:

    \begin{minted}[fontsize=\footnotesize]{matlab}
b = [3.96, 6.36, 8.3, 4.38, 2.07];
a = [1, 0.39, -0.93, -0.33, 0.34];

[R, p, C] = residuez(b, a)
    \end{minted}

    Se obtiene:
    \begin{minted}[fontsize=\footnotesize]{matlab}
R =

  -0.0252 + 1.6371i
  -0.0252 - 1.6371i
  -1.0389 -40.5443i
  -1.0389 +40.5443i


p =

  -0.8384 + 0.3211i
  -0.8384 - 0.3211i
   0.6434 + 0.0883i
   0.6434 - 0.0883i


C =

    6.0882
    \end{minted}
    Donde:
    \begin{align}
    H(z) &= \frac{3.96 + 6.36z^{-1} + 8.3z^{-2} + 4.38z^{-3} + 2.07z^{-4}}
        {1 +0.39z^{-1} -0.93z^{-2} -0.33z^{-3} +0.34z^{-4}}\\
        &\downarrow\texttt{residuez()}\notag\\
        &=\frac{R_1}{1-p_1z^{-1}}+\frac{R_2}{1-p_2z^{-1}}+\frac{R_3}{1-p_3z^{-1}}+\frac{R_4}{1-p_4z^{-1}}+C\\
        &=\frac{R_1}{1-p_1z^{-1}}+\frac{\overline{R_1}}{1-\overline{p_1}z^{-1}}+\frac{R_3}{1-p_3z^{-1}}+\frac{\overline{R_3}}{1-\overline{p_3}z^{-1}}+6.08
    \end{align}

    Si bien a partir de esta expresión podríamos dibujar el diagrama en forma paralela,
    es conveniente expresarlo en téminos de segundo orden para evitar matemáticas
    complejas en el diagrama. Sumando los pares conjugados obtenemos los sistemas
    de segundo orden con coeficientes reales.
    
    Esto se puede realizar utilizando el mismo comando \texttt{residuez}. Emparejando
    los términos correspondientes:
    \begin{equation}
        H(z)=\underbrace{\frac{R_1}{1-p_1z^{-1}}+\frac{\overline{R_1}}{1-\overline{p_1}z^{-1}}}_{\texttt{R(1:2), p(1:2)}}
        +\underbrace{\frac{R_3}{1-p_3z^{-1}}+\frac{\overline{R_3}}{1-\overline{p_3}z^{-1}}}_{\texttt{R(3:4), p(3:4)}}+6.08
    \end{equation}

    Se ingresan a matlab:
    \begin{minted}[fontsize=\footnotesize]{matlab}
[B1, A1] = residuez(R(1:2), p(1:2), [])
[B2, A2] = residuez(R(3:4), p(3:4), [])

B1 =

  -0.0504 + 0.0000i  -1.0937 + 0.0000i


A1 =

    1.0000    1.6769    0.8061


B2 =

   -2.0778    8.4985


A2 =

    1.0000   -1.2869    0.4218
    \end{minted}

    Es decir:
    \begin{equation}
        H(z)=\frac{-0.05-1.09z^{-1}}{1+1.67z^{-1}+0.80z^{-2}}
        +\frac{-2.07+8.49z^{-1}}{1-1.28z^{-1}+0.42z^{-2}} 
        +6.08
    \end{equation}

    Y dibujamos el diagrama de cada uno de los términos.

    Para $6.08$:\\
    \includegraphics[width=10cm]{imagenes/8.4a.1.png}

    Para $\frac{-0.05-1.09z^{-1}}{1+1.67z^{-1}+0.80z^{-2}}$:\\
    \includegraphics[width=10cm]{imagenes/8.4a.2.png}

    Para $\frac{-2.07+8.49z^{-1}}{1-1.28z^{-1}+0.42z^{-2}}$:\\
    \includegraphics[width=10cm]{imagenes/8.4a.3.png}

    Finalmente se unen todos en paralelo y se suman:\\
    \includegraphics[width=10cm]{imagenes/8.4a.4.png}
    
    
\end{solution}

\part Una estructura en forma cascada
\begin{solution}
    Se ingresa la función de transferencia a Matlab y se usa el comando \texttt{tf2sos}:
    \begin{minted}[fontsize=\footnotesize]{matlab}
%% 4 (b) 

b = [3.96, 6.36, 8.3, 4.38, 2.07];
a = [1, 0.39, -0.93, -0.33, 0.34];

sos = tf2sos(b, a)

sos =

    3.9600    2.2107    2.1236    1.0000   -1.2869    0.4218
    1.0000    1.0478    0.9748    1.0000    1.6769    0.8061
    \end{minted}
    
    La matriz respresenta una factorización de la función de transferencia en
    sistemas de segundo orden. Cada fila es un factor. Las tres primeras columnas
    corresponden a los numeradores y las últimas tres a los denominadores.
    Así:
    \begin{align}
    H(z) &= \frac{3.96 + 6.36z^{-1} + 8.3z^{-2} + 4.38z^{-3} + 2.07z^{-4}}
        {1 +0.39z^{-1} -0.93z^{-2} -0.33z^{-3} +0.34z^{-4}}\\
        &=\frac{3.96+2.21z^{-1}+2.12z^{-2}}{1-1.29z^{-1}+0.42z^{-2}}
        \cdot\frac{1+1.05z^{-1}+0.97z^{-2}}{1+1.68z^{-1}+0.81z^{-2}}
    \end{align}

    Dibujando el diagrama de cada factor (forma directa II transpuesta):\\
    Para $\frac{3.96+2.21z^{-1}+2.12z^{-2}}{1-1.29z^{-1}+0.42z^{-2}}$:\\
    \includegraphics[width=14cm]{imagenes/8.4b.1.png}\\
    
    \pagebreak
    Y para $\frac{3.96+2.21z^{-1}+2.12z^{-2}}{1-1.29z^{-1}+0.42z^{-2}}$:\\
    \includegraphics[width=14cm]{imagenes/8.4b.2.png}\\

    Al concatenar ambos se obtiene el diagrama de la estructura completa:\\
    \includegraphics[width=14cm]{imagenes/8.4b.3.png}\\

    
\end{solution}

\end{parts}
\end{questions}
%\tableofcontents
%\listoffigures
%\listoftables
%\listoftodos

%%%%%%%%%%%%%%%%%%%%%%%%%%%%%%%%
%% -- Fin del Documento -- %%
%%%%%%%%%%%%%%%%%%%%%%%%%%%%%%%%
\end{document}