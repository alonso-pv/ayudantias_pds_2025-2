\documentclass[12pt]{exam}
\usepackage{preamble}

% disable para esconder
\usepackage[enable]{easy-todo}

% Comentar para esconder soluciones
\printanswers

%%%%%%%%%%%%%%%%%%%%%%%%%%%%%%%%
%% -- Inicio del Documento -- %%
%%%%%%%%%%%%%%%%%%%%%%%%%%%%%%%%
\begin{document}
\begin{center}
    {\Large
        Ayudantía 5 - Procesamiento Digital de Señales
    }
\end{center}

\begin{questions}

\question
Una señal $x_c(t)$ tiene el espectro de magnitud $X_c(j\Omega)$ que se muestra
en la figura:

\includegraphics[width=15cm]{imagenes/5.2.png}

La señal se muestrea con un periodo de muestreo $T=1/8$ segundos.

\begin{parts}
    \part Encuentre la frecuencia de Nyquist $\Omega_H$.
    \begin{solution}
        La frecuencia de Nyquist corresponde a la máxima frecuencia presente
        en la señal. Por inspección: $\Omega_H=10\pi=5\cdot 2\pi$ (en Hz: $F_H=5$ Hz).
    \end{solution}

    \part ¿Se cumple el teorema de muestreo?
    \begin{solution}
        El teorema de muestreo (o teorema de Nyquist) dice que si $F_s\geq 2F_H$
        entonces la señal se puede recuperar completamente.
        
        Con $T=1/8$, $F_s=1/T=8$ Hz, y $F_H=5$ Hz, la desigualdad $F_s\geq2F_H$
        no se cumple y no se cumple el teorema de muestreo.
    \end{solution}

    \part Dibuje el espectro $X(e^{j\Omega T})$ de la señal muestreada $x[n]$ en
    el rango $-30\pi\leq\Omega\leq30\pi$.
    \begin{solution}
        El espectro corresponde a la sumatoria de copias desplazadas en $2\pi/T=2\pi F_s=16\pi$
        del espectro original.
\begin{align}
    X(e^{j\Omega T})&=\sum_{k=-\infty}^{\infty}X\left(j\Omega-j\frac{2\pi}{T}k\right)\\
    &=\sum_{k=-\infty}^{\infty}X\left(j\Omega-j16\pi k\right)
\end{align}

        \par\includegraphics[width=14cm]{imagenes/5.1.b.png}
    \end{solution}
\end{parts}

\question Se tiene una señal continua $x_c(t)=2\cos(150\pi t)$.

\begin{parts}
    \part Escoja una frecuencia de muestreo adecuada para la señal.
    \begin{solution}
        La frecuencia de Nyquist (máxima frecuencia) es $150\pi=75\cdot2\pi$ (es decir 75 Hz).
        Al escoger una frecuencia de muestreo, utilizamos como criterio el teorema
        de muestreo (o de Nyquist): $F_s\geq2F_H$.

        Por lo tanto:
        \begin{align}
            F_s&\geq2F_H=2\cdot 75\\
            &\geq150\text{ Hz}
        \end{align}

        Escogiendo una tasa de muestreo sobre $150$ Hz, seremos capaces de recuperar
        la señal sin aliasing. Podemos escoger por ejemplo, $F_s=300$ Hz.

        
    \end{solution}

    \part La señal se muestrea, y posteriormente se reconstruye utilizando un 
    retentor de orden cero y un filtro ideal que bloquea frecuencias por encima
    de la frecuencia de Nyquist. Bosqueje la señal en cada paso y determine la señal reconstruida.
\end{parts}

\question
Considere la siguiente señal tiempo continuo:
$$
x_c(t) = 5\cos\left(200\pi t + \frac{\pi}{6}\right) + 4\sin\left(400\pi t\right)
$$
\begin{parts}
    \part Determine las frecuencias (en Hz) presentes en la señal.
    \begin{solution}
        
    \end{solution}
    
    \part Determine la frecuencia de Nyquist, y tasa de muestreo para un muestreo
    sin aliasing.

    \part La señal se muestrea con $F_s=500$ Hz (ADC ideal). Determine el espectro $X(e^{j\Omega T})$
    de la señal muestreada $x[n]$ y grafique su magnitud como función de la frecuencia $F$ en Hz.

    \part Determine la señal $x_r(t)$ reconstruida utilizando un DAC ideal.

    \part Repita (c) y (d), muestreando la señal con $F_s=350$ Hz.
\end{parts}

\end{questions}
%\tableofcontents
%\listoffigures
%\listoftables
%\listoftodos

%%%%%%%%%%%%%%%%%%%%%%%%%%%%%%%%
%% -- Fin del Documento -- %%
%%%%%%%%%%%%%%%%%%%%%%%%%%%%%%%%
\end{document}