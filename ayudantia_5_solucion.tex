\documentclass[12pt]{exam}
\usepackage{preamble}

% disable para esconder
\usepackage[enable]{easy-todo}

% Comentar para esconder soluciones
\printanswers

%%%%%%%%%%%%%%%%%%%%%%%%%%%%%%%%
%% -- Inicio del Documento -- %%
%%%%%%%%%%%%%%%%%%%%%%%%%%%%%%%%
\begin{document}
\begin{center}
    {\Large
        Ayudantía 5 - Procesamiento Digital de Señales
    }
\end{center}

\begin{questions}

\question
Una señal $x_c(t)$ tiene el espectro de magnitud $X_c(j\Omega)$ que se muestra
en la figura:

\includegraphics[width=15cm]{imagenes/5.2.png}

La señal se muestrea con un periodo de muestreo $T=1/8$ segundos.

\begin{parts}
    \part Encuentre la frecuencia de Nyquist $\Omega_H$.
    \begin{solution}
        La frecuencia de Nyquist corresponde a la máxima frecuencia presente
        en la señal. Por inspección: $\Omega_H=10\pi=5\cdot 2\pi$ (en Hz: $F_H=5$ Hz).
    \end{solution}

    \part ¿Se cumple el teorema de muestreo?
    \begin{solution}
        El teorema de muestreo (o teorema de Nyquist) dice que si $F_s\geq 2F_H$
        entonces la señal se puede recuperar completamente.
        
        Con $T=1/8$, $F_s=1/T=8$ Hz, y $F_H=5$ Hz, la desigualdad $F_s\geq2F_H$
        no se cumple y no se cumple el teorema de muestreo.
    \end{solution}

    \part Dibuje el espectro $X(e^{j\Omega T})$ de la señal muestreada $x[n]$ en
    el rango $-30\pi\leq\Omega\leq30\pi$.
    \begin{solution}
        El espectro corresponde a la sumatoria de copias desplazadas en $2\pi/T=2\pi F_s=16\pi$
        del espectro original.
\begin{align}
    X(e^{j\Omega T})&=\sum_{k=-\infty}^{\infty}X\left(j\Omega-j\frac{2\pi}{T}k\right)\\
    &=\sum_{k=-\infty}^{\infty}X\left(j\Omega-j16\pi k\right)
\end{align}

        \par\includegraphics[width=14cm]{imagenes/5.1.b.png}
    \end{solution}
\end{parts}

\question Se tiene una señal continua $x_c(t)=2\cos(150\pi t)$.

\begin{parts}
    \part Escoja una frecuencia de muestreo adecuada para la señal.
    \begin{solution}
        La frecuencia de Nyquist (máxima frecuencia) es $150\pi=75\cdot2\pi$ (es decir 75 Hz).
        Al escoger una frecuencia de muestreo, utilizamos como criterio el teorema
        de muestreo (o de Nyquist): $F_s\geq2F_H$.

        Por lo tanto:
        \begin{align}
            F_s&\geq2F_H=2\cdot 75\\
            &\geq150\text{ Hz}
        \end{align}

        Escogiendo una tasa de muestreo sobre $150$ Hz, seremos capaces de recuperar
        la señal sin aliasing. Podemos escoger por ejemplo, $F_s=300$ Hz.
        
    \end{solution}

    \part La señal se muestrea, y posteriormente se reconstruye utilizando un 
    retentor de orden cero y un filtro ideal que bloquea frecuencias por encima
    de la frecuencia de Nyquist. Bosqueje la señal en cada paso y determine la señal reconstruida.
    \begin{solution}
        Se considera $F_s=300$ Hz ($T=1/300$).

        La señal original $x_c(t)$ se muestrea cada $t=nT$.
        \begin{align}
            x[n] &= x_c(nT)=2\cos(150\pi nT)\\
            &=2\cos\left(\frac{150}{300}\pi n\right)\\
            &=2\cos\left(\frac{\pi}{2} n\right)
        \end{align}
        Tras el retentor de orden cero, la señal $x_{SH}(t)$ mantiene el valor de
        cada muestra por $T=1/300$ tiempo.

        El filtro elimina las frecuencias $F>\frac{F_s}{2}$ introducidas por el retentor
        y el muestreo. Como
        se cumple el teorema de muestreo, la señal recuperada es la original con
        un retardo de $T/2=1/600$ (debido al retentor de orden cero):
        
        $x_r(t)=x_c(t-T/2)=2\cos(150\pi [t-T/2])=2\cos(150\pi t- \pi/4)$

        \includegraphics[width=14cm]{imagenes/5.2b.png}

        \underline{Nota}: Se asume que el filtro compensa la atenuación de la señal causada
        por este.
    \end{solution}
\end{parts}

\question
Considere la siguiente señal tiempo continuo:
$$
x_c(t) = 5\cos\left(200\pi t + \frac{\pi}{6}\right) + 4\sin\left(400\pi t\right)
$$
\begin{parts}
    \part Determine las frecuencias (en Hz) presentes en la señal.
    \begin{solution}
        Las frecuencias presentes en la señal se hallan mediante inspección:
        \begin{align}
            x_c(t) &= 5\cos\left(200\pi t + \frac{\pi}{6}\right) + 4\sin\left(400\pi t\right)\\
            &= 5\cos\Big(\underbrace{100}_{F_1}\cdot2\pi t + \frac{\pi}{6}\Big) + 4\sin\Big(\underbrace{200}_{F_2}\cdot2\pi t\Big)
        \end{align}

        Las frecuencias son $F_1=100$ Hz, y $F_2=200$ Hz.
    \end{solution}
    
    \part Determine la frecuencia de Nyquist, y tasa de muestreo para un muestreo
    sin aliasing.
    \begin{solution}
        La frecuencia de Nyquist $F_H$ corresponde a la mayor frecuencia en la señal.
        En este caso $F_2=200$ Hz es la mayor frecuencia, por lo que $F_H=200$ Hz.

        Para una tasa de muestreo sin aliasing, consideramos el teorema de Nyquist.
        \begin{align}
            F_s&\geq2F_H\\
            &\geq 400\text{ Hz}
        \end{align}

        Para una tasa de muestreo sin aliasing, esta debe ser mayor o igual a $400$ Hz.
    \end{solution}

    \part La señal se muestrea con $F_s=500$ Hz (ADC ideal). Determine el espectro $|X(e^{j\Omega T})|$
    de la señal muestreada $x[n]$ y grafique su magnitud como función de la frecuencia $F$ en Hz.
    \begin{solution}
        \underline{Nota}: $\mathcal{F}\{e^{j\Omega_0t}\}=\pi\delta(\Omega-\Omega_0)$

        Notemos que la señal se puede descomponer en exponenciales complejas:
        \begin{align}
            x_c(t) &= 5\cos\left(200\pi t + \frac{\pi}{6}\right) + 4\sin\left(400\pi t\right)\\
            &=\frac{5e^{j\pi/6}}{2}e^{j200\pi t} + \frac{5e^{-j\pi/6}}{2}e^{-j200\pi t}
            + \frac{4}{2}e^{j400\pi t}-\frac{4}{2}e^{-j400\pi t}
        \end{align}

        El espectro original tiene picos en las frecuencias $\pm 200\pi$ y $\pm400\pi$.
        En Hz, estas son $\pm100$ Hz, y $\pm 200$ Hz.

        Al muestrear la señal, el espectro se hace periódico. Aparecen copias del
        espectro original desplazadas en $2\pi/T=2\pi F_s=1000\pi$ hacia la izquierda
        y hacia la derecha. En Hz, las copias se desplazan $F_s=500$ Hz hacia los lados.

        En el gráfico, las copias debido al muestreo se muestran en rojo.
        
        \includegraphics[width=14cm]{imagenes/5.3c.png}

        Al cumplirse el teorema del muestreo, no hay aliasing (las copias no se
        cruzan con el espectro original).
    \end{solution}

    \part Determine la señal $x_r(t)$ reconstruida utilizando un DAC ideal.
    \begin{solution}
        Dado que se cumple el teorema del muestreo (con $F_s=500$ Hz $> 400$ Hz),
        se recupera la señal original (DAC ideal).
        
        DAC ideal reconstruye la señal tomando las frecuencias $|F|\leq F_s/2=250$ Hz.

        \includegraphics[width=14cm]{imagenes/5.3d.png}

        Como se puede ver en el gráfico esto nos deja con el espectro original.

        La señal reconstruida es igual a la original: $x_r(t)=x_c(t)$.
    \end{solution}

    \part Repita (c) y (d), muestreando la señal con $F_s=350$ Hz.
    \begin{solution}
        Al muestrear con $F_s=350$ Hz, no se cumple el teorema del muestreo
        
        (350 Hz <\, 400 Hz).

        Las copias del espectro original se desplazan $F_s=350$ Hz hacia la izquierda
        y hacia la derecha.

        \includegraphics[width=14cm]{imagenes/5.3e.png}

        Como se puede ver, el espectro de las copias se cruza con el espectro original.
        Hay aliasing.

        Al reconstruir la señal, el DAC ideal toma las frecuencias $|F|<F_s/2=175$ Hz.

        \includegraphics[width=14cm]{imagenes/5.3e.2.png}

        El espectro dentro de las lineas punteadas ya no es el espectro original.

        Para hallar la respuesta reconstruida, consideramos la formula de interpolación
        del DAC ideal (banda limitada):
        \begin{align}
            x_r(t)&=\sum_{n=-\infty}^{\infty}x[n]g_{BL}(t-nT)\\
            g_{BL}(t)&=\frac{\sin(\pi t/T)}{\pi t/T}
        \end{align}

        En este caso al tratarse de señales sinusoidales, se puede encontrar de
        una manera un poco más cualitativa:

        1. Aplicamos el muestreo.
        \begin{align}
            x[n]&=x_c(nT)=5\cos\left(200\pi nT + \frac{\pi}{6}\right) + 4\sin\left(400\pi nT\right)\\
            &=5\cos\left(\frac{200}{350}\pi n + \frac{\pi}{6}\right) + 4\sin\left(\frac{400}{350}\pi n\right)
        \end{align}

        2. En el dominio discreto la frecuencia es periódica. Llevamos todas las
        frecuencias al intervalo $[-\pi,\pi]$. Vemos que $400\pi/350>\pi$:
        \begin{align}
            x[n]&=5\cos\left(\frac{200}{350}\pi n + \frac{\pi}{6}\right) + 4\sin\left(\frac{400}{350}\pi n\right)\\
            &=5\cos\left(\frac{200}{350}\pi n + \frac{\pi}{6}\right) + 4\sin\left(\frac{400}{350}\pi n-2\pi n\right)\\
            &=5\cos\left(\frac{200}{350}\pi n + \frac{\pi}{6}\right) + 4\sin\left(\frac{400}{350}\pi n-\frac{700}{350}\pi n\right)\\
            &=5\cos\left(\frac{200}{350}\pi n + \frac{\pi}{6}\right) + 4\sin\left(-\frac{300}{350}\pi n\right)\\
            &=5\cos\left(\frac{200}{350}\pi n - \frac{\pi}{6}\right) - 4\sin\left(\frac{300}{350}\pi n\right)\\
        \end{align}

        3. El DAC ideal ``deshace'' la sustitución $t=nT=n/350$:
        \begin{align}
            x[n]&=5\cos\left(\frac{200}{350}\pi n - \frac{\pi}{6}\right) - 4\sin\left(\frac{300}{350}\pi n\right)\\
            &\Big\downarrow t=n/350\nonumber\\
            x_r(t)&=5\cos\left(200\pi t - \frac{\pi}{6}\right) - 4\sin(300\pi t)
        \end{align}

        En la señal reconstruida, desaparece la frecuencia 200 Hz ($400\pi$), y
        aparece otra frecuencia a 150 Hz ($300\pi$) producto del aliasing. Esto
        concuerda con el gráfico mostrado anteriormente.

    \end{solution}
\end{parts}

\end{questions}
%\tableofcontents
%\listoffigures
%\listoftables
%\listoftodos

%%%%%%%%%%%%%%%%%%%%%%%%%%%%%%%%
%% -- Fin del Documento -- %%
%%%%%%%%%%%%%%%%%%%%%%%%%%%%%%%%
\end{document}