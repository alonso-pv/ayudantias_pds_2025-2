\documentclass[12pt]{exam}
\usepackage{../preamble}

% disable para esconder
\usepackage[enable]{easy-todo}

% Comentar para esconder soluciones
\printanswers

%%%%%%%%%%%%%%%%%%%%%%%%%%%%%%%%
%% -- Inicio del Documento -- %%
%%%%%%%%%%%%%%%%%%%%%%%%%%%%%%%%
\begin{document}
\begin{center}
    {\Large
        Ayudantía 12 - Procesamiento Digital de Señales
    }
\end{center}

\begin{questions}

\question Dado el siguiente diagrama de flujo de señales\\
\includegraphics[width=15cm]{imagenes/12.1.png}
\begin{parts}
    
    \part Encontrar la función de transferencia del sistema.

    \part Encontrar una ecuación de diferencias.

\end{parts}

\question Para el sistema dado por la función de transferencia
\[
    H(z)=\frac{1+0.5z^{-1}}{1-0.2z^{-1}+0.8z^{-2}}
\]
\begin{parts}
    
    \part Dibujar estructura en forma directa I.

    \part Dibujar estructura en forma transpuesta II.

\end{parts}

\question Para los siguientes sistemas, identifique el tipo de filtro (IIR o
FIR) y determine si tiene fase lineal o no:
\begin{flalign*}
    H_0:\quad H_0(z)&=\frac{1+0.5z^{-1}}{1-0.2z^{-1}+0.8z^{-2}}
    & H_3:\quad h_3[n]&=\{4,1,0,-1,-4\}   \\
    \\
    H_1:\quad H_1(z)&=3 -2z^{-1} -2z^{-3} +3z^{-4}
    & H_4:\quad h_4[n]&=\{-5,1,3,-3,-1,5\}\\
    \\
    H_2:\quad h_2[n]&=\{5,6,3,3,6,5\}
    & H_5:\quad h_5[n]&=u[n]-u[n-4]+\delta[n-3]
\end{flalign*}

\question Hallar magnitud $|H(e^{j\omega})|$, amplitud $A(e^{j\omega})$,  y retardo de
grupo de:
\begin{parts}

    \part $H_6(z)= 3 -2z^{-1} -2z^{-3} +3z^{-4}$

    \part $h_7[n]= \{2,2,2,2,2\}$

\end{parts}

\pagebreak
\question Diseñe un pasabajos digital FIR para una señal con $F_p=500$ Hz y
$F_s=550$ Hz.

\includegraphics[width=15cm]{imagenes/9.p3.png}

\begin{parts}
    \part Determine $\omega_p$ y $\omega_s$, si la señal se muestrea con una
    frecuencia de muestreo $F_m=10$ kHz.

    \part Determine la frecuencia de corte $\omega_c$ y un filtro ideal para el
    diseño del filtro.

    \part Si los ripples aceptables son $\delta_s=0.073$ y
    $\delta_p=0.028$, determine una ventana apropiada para el filtro.

    \part Determine el orden y expresión de la respuesta a impulso del filtro.
\end{parts}

\question Las especificaciones para un filtro IIR son $\omega_p=0.4\pi$,
$\omega_s=0.5\pi$, $A_p=1$ dB y $A_s = 55$ dB

\begin{parts}
    \part Determine las frecuencias de diseño si se utiliza transformación
    bilinear con $T_d=1$.

    \part Determine las frecuencias de diseño si se utiliza impulso invariante
    con $T=0.005$
\end{parts}

\question Dada la función de transferencia
\[
H_c(s)=\frac{1}{s+1}
\]
\begin{parts}
    \part Aplique transformación bilinear para encontrar el filtro digital
    $H_d(z)$.

    \part Hallar la frecuencia de corte $\omega_c$ en el plano discreto.
\end{parts}

\end{questions}
%\tableofcontents
%\listoffigures
%\listoftables
%\listoftodos

%%%%%%%%%%%%%%%%%%%%%%%%%%%%%%%%
%% -- Fin del Documento -- %%
%%%%%%%%%%%%%%%%%%%%%%%%%%%%%%%%
\end{document}