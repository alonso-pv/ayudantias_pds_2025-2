\documentclass[12pt]{exam}
\usepackage{../preamble}

% disable para esconder
\usepackage[enable]{easy-todo}

% Comentar para esconder soluciones
\printanswers

%%%%%%%%%%%%%%%%%%%%%%%%%%%%%%%%
%% -- Inicio del Documento -- %%
%%%%%%%%%%%%%%%%%%%%%%%%%%%%%%%%
\begin{document}
\begin{center}
    {\Large
        Ayudantía 12 - Procesamiento Digital de Señales
    }
\end{center}

\begin{questions}

\question Dado el siguiente diagrama de flujo de señales\\
\includegraphics[width=15cm]{imagenes/12.1.png}
\begin{parts}
    
    \part Encontrar la función de transferencia del sistema.
    \begin{solution}
        Extendiendo un poco el diagrama y definiendo variables intermedias
        $v[n]$ y $w[n]$:\\
        \includegraphics[width=14cm]{imagenes/12.1a.png}
        Ecuaciones y F.de Ts. para $v[n]$, $w[n]$ y $y[n]$:
        \begin{flalign}
            v[n]&=x[n]+\frac{1}{2}v[n-1]
            & w[n]&=v[n]-v[n-1]
            & y[n]&=w[n]-\frac{1}{4}w[n-1] \\
            %
            &\downarrow\notag
            & &\downarrow\notag
            & &\downarrow\notag \\
            %
            \frac{V(z)}{X(z)}&=\frac{1}{1-\frac{1}{2}z^{-1}}
            & \frac{W(z)}{V(z)}&=1-z^{-1}
            & \frac{Y(z)}{W(z)}&=\frac{1}{1+\frac{1}{4}z^{-1}}
        \end{flalign}
        Finalmente:
        \begin{gather}
            H(z)=\frac{Y(z)}{X(z)}=
            \frac{Y(z)}{W(z)}\cdot
            \frac{W(z)}{V(z)}\cdot
            \frac{V(z)}{X(z)}   \\
            =\frac{1-z^{-1}}{(1+\frac{1}{4}z^{-1})(1-\frac{1}{2}z^{-1})}
        \end{gather}
    \end{solution}

    \part Encontrar una ecuación de diferencias.
    \begin{solution}
        \begin{gather}
            H(z)=\frac{1-z^{-1}}{(1+\frac{1}{4}z^{-1})(1-\frac{1}{2}z^{-1})}\\
            \frac{Y(z)}{X(z)}=\frac{1-z^{-1}}{1-\frac{1}{2}z^{-1}-\frac{1}{8}z^{-2}}\\
            \downarrow\notag\\
            y[n]=x[n]-x[n-1]+\frac{1}{2}y[n-1]+\frac{1}{8}y[n-2]
        \end{gather}
    \end{solution}

\end{parts}

\question Para el sistema dado por la función de transferencia
\[
    H(z)=\frac{1+0.5z^{-1}}{1-0.2z^{-1}+0.8z^{-2}}
\]
\begin{parts}
    
    \part Dibujar estructura en forma directa I.
    \begin{solution}
        \begin{equation}
            y[n]=x[n]+0.5x[n-1]\underbrace{+0.2y[n]-0.8y[n-2]}_{\text{Coeficientes se invierten}}
        \end{equation}
        Primero ceros de la F. de T., luego polos. Señales se retardan primero y
        después se multiplican por los coeficientes
        \\\includegraphics[width=14cm]{imagenes/12.2a.png}
    \end{solution}

    \pagebreak
    \part Dibujar estructura en forma transpuesta II.
    \begin{solution}
        Similar al anterior, pero señales se multiplican por los coeficientes,
        luego se retardan. Lineas de retardo se unen en una sola.
        \\\includegraphics[width=10cm]{imagenes/12.2b.png}
    \end{solution}

\end{parts}

\question Para los siguientes sistemas, identifique el tipo de filtro (IIR o
FIR) y determine si tiene fase lineal o no:
\begin{flalign*}
    H_0:\quad H_0(z)&=\frac{1+0.5z^{-1}}{1-0.2z^{-1}+0.8z^{-2}}
    & H_3:\quad h_3[n]&=\{4,1,0,-1,-4\}   \\
    \\
    H_1:\quad H_1(z)&=3 -2z^{-1} -2z^{-3} +3z^{-4}
    & H_4:\quad h_4[n]&=\{-5,1,3,-3,-1,5\}\\
    \\
    H_2:\quad h_2[n]&=\{5,6,3,3,6,5\}
    & H_5:\quad h_5[n]&=u[n]-u[n-4]+\delta[n-3]
\end{flalign*}

\begin{solution} 
    $\displaystyle H_0:\quad H_0(z)=\frac{1+0.5z^{-1}}{1-0.2z^{-1}+0.8z^{-2}}$

    Filtro tiene retroalimentación, por lo que es IIR.
    Al ser IIR, causal o anticausal, no tiene fase lineal.
\end{solution}

\begin{solution}
    $\displaystyle H_1:\quad H_1(z)=3 -2z^{-1} -2z^{-3} +3z^{-4}$

    \begin{align}
        h_1[n]=\{3,-2,0,-2,3\}
    \end{align}
    Es FIR, simétrica $h_1[n]=h_1[M-n]$, de orden $M=4$ par.
    
    Por lo tanto, $H_1$ es FIR tipo I y tiene fase lineal.
\end{solution}

\begin{solution}
    $\displaystyle H_2:\quad h_2[n]=\{5,6,3,3,6,5\}$
    
    Es FIR, simétrica $h_2[n]=h_2[M-n]$, de orden $M=5$ impar.

    Por lo tanto, $H_2$ es FIR tipo II y tiene fase lineal.
\end{solution}

\begin{solution}
    $\displaystyle H_3:\quad h_3[n]=\{4,1,0,-1,-4\}$
    
    Es FIR, antisimétrica $h_3[n]=-h_3[M-n]$, de orden $M=4$ par.

    Por lo tanto, $H_3$ es FIR tipo III y tiene fase lineal.
\end{solution}

\begin{solution}
    $\displaystyle H_4:\quad h_4[n]=\{-5,1,3,-3,-1,5\}$
    
    Es FIR, antisimétrica $h_4[n]=-h_4[M-n]$, de orden $M=5$ impar.

    Por lo tanto, $H_4$ es FIR tipo IV y tiene fase lineal.
\end{solution}

\begin{solution}
    $\displaystyle H_5:\quad h_5[n]=u[n]-u[n-4]+\delta[n-3]$

    \includegraphics[width=14cm]{imagenes/12.3.png}

    \begin{equation}
        h_5[n]=\{1,1,1,2\}
    \end{equation}
    Es FIR, pero no tiene respuesta impulso simétrica ni antisimétrica.
    Por lo tanto no tiene fase lineal.
\end{solution}

\question Hallar magnitud $|H(e^{j\omega})|$, amplitud $A(e^{j\omega})$,  y retardo de
grupo de:
\begin{parts}

    \part $H_6(z)= 3 -2z^{-1} -2z^{-3} +3z^{-4}$
    \begin{solution}
        \begin{align}
            H_6(e^{j\omega})&=3 -2e^{-j\omega} -2e^{-3j\omega} +3e^{-4j\omega} \\
            &=\left(3e^{2j\omega} -2e^{j\omega}
            -2e^{-j\omega} +3e^{-2j\omega}\right)e^{-2j\omega} \\
            %
            &\downarrow\boxed{e^{j\theta}+e^{-j\theta}=2\cos(\theta)}\notag\\
            %
            &=\left[6\cos(2\omega)-4\cos(\omega)\right]e^{-2j\omega}
            =A(e^{j\omega})e^{j\Psi(\omega)}
        \end{align}

        Por lo tanto:
        \begin{equation}
            |H(e^{j\omega})|=\left|6\cos(2\omega)-4\cos(\omega)\right|
        \end{equation}

        La amplitud:
        \begin{equation}
            A(e^{j\omega})=6\cos(2\omega)-4\cos(\omega)
        \end{equation}

        La respuesta de ángulo y retardo de grupo:
        \begin{align}
            &\Psi(\omega)=-2\omega & \tau_{gd}(\omega)&=-\frac{d\Psi}{d\omega} \\
            %
            & & &=2
        \end{align}
    \end{solution}

    \part $h_7[n]= \{2,2,2,2,2\}$
    \begin{solution}
        \begin{align}
            h_7[n]&=2(u[n]-u[n-5]) \\
            %
            &\downarrow \mathcal{Z}\notag\\
            %
            H_7(z)&=2\frac{1-z^{-5}}{1-z^{-1}}\\
            %
            &\downarrow z=e^{j\omega}\notag\\
            %
            H_7(e^{j\omega})&=2\frac{1-e^{-5j\omega}}{1-e^{-j\omega}}\cdot
            \left(\frac{e^{\frac{5}{2}j\omega}e^{-\frac{5}{2}j\omega}}
            {e^{\frac{1}{2}j\omega}e^{-\frac{1}{2}j\omega}}\right)   \\
            %
            &=2\frac{e^{\frac{5}{2}j\omega}-e^{-\frac{5}{2}j\omega}}
            {e^{\frac{1}{2}j\omega}-e^{-\frac{1}{2}j\omega}}
            \cdot e^{-2j\omega} \\
            %
            &=2\frac{\sin(5\omega/2)}{\sin(\omega/2)}\cdot e^{-2j\omega}
            =A(e^{j\omega})e^{j\Psi(\omega)}
        \end{align}

        Por lo tanto:
        \begin{equation}
            |H(e^j\omega)|=2\left|\frac{\sin(5\omega/2)}{\sin(\omega/2)}\right|
        \end{equation}
        
        La amplitud:
        \begin{equation}
            A(e^{j\omega})=2\frac{\sin(5\omega/2)}{\sin(\omega/2)}
        \end{equation}

        La respuesta de ángulo y retardo de grupo:
        \begin{align}
            &\Psi(\omega)=-2\omega & \tau_{gd}(\omega)&=-\frac{d\Psi}{d\omega} \\
            %
            & & &=2
        \end{align}
        
    \end{solution}

\end{parts}

\pagebreak
\question Diseñe un pasabajos digital FIR para una señal con $F_p=500$ Hz y
$F_s=550$ Hz.

\includegraphics[width=15cm]{imagenes/9.p3.png}

\begin{parts}
    \part Determine $\omega_p$ y $\omega_s$, si la señal se muestrea con una
    frecuencia de muestreo $F_m=10$ kHz.
    \begin{solution}
        \begin{align}
            &\omega_p=2\pi\frac{F_p}{F_m}=0.1\pi
            & &\omega_s=2\pi\frac{F_s}{F_m}=0.11\pi
        \end{align}
    \end{solution}

    \part Determine la frecuencia de corte $\omega_c$ y un filtro ideal para el
    diseño del filtro.
    \begin{solution}
        \begin{align}
            \omega_c &= \frac{\omega_p+\omega_s}{2}
            & h_{\text{ideal}}[n]&=\frac{\sin(\omega_c[n-M/2])}{\pi[n-M/2]}\\
            %
            &=0.105\pi
            & &=\frac{\sin(0.105\pi[n-M/2])}{\pi[n-M/2]}
        \end{align}
    \end{solution}

    \part Si los ripples aceptables son $\delta_s=0.073$ y
    $\delta_p=0.028$, determine una ventana apropiada para el filtro.
    \begin{solution}
        \begin{align}
            \delta&=\min\{\delta_s,\delta_p\}
            & A&=-20\log(\delta)\\
            %
            &=0.028
            & &=31.05 \text{ dB}
        \end{align}

        De la tabla (columna $A_s$ \textbf{dB}): $44>31.05\longrightarrow$ Hann
    \end{solution}

    \part Determine el orden y expresión de la respuesta a impulso del filtro.
    \begin{solution}
        \begin{equation}
            \Delta\omega=\omega_s-\omega_p=0.01\pi
        \end{equation}
        De la tabla (columna \textbf{Exact} $\Delta\omega$, ventana Hann):
        $\Delta\omega=6.2\pi/L$
        \begin{align}
            0.01\pi=6.2\pi/L\longrightarrow L=620\longrightarrow M=619=L-1
        \end{align}

        Considerando esto, la respuesta a impulso $h[n]$ del filtro FIR queda:
        \begin{equation} 
            h[n]=
            h_{\text{ideal}}[n]\cdot w_\text{Hann}[n]=
            \frac{\sin(0.105\pi[n-619/2])}{\pi[n-619/2]}\cdot w_\text{Hann}[n]\\
        \end{equation}
        Con $w_\text{Hann}[n]$ ventana de Hann de largo $L=620$
    \end{solution}

\end{parts}

\question Las especificaciones para un filtro IIR son $\omega_p=0.4\pi$,
$\omega_s=0.5\pi$, $A_p=1$ dB y $A_s = 55$ dB

\begin{parts}
    \part Determine las frecuencias de diseño si se utiliza transformación
    bilinear con $T_d=1$.
    \begin{solution}
    \begin{align}
        \Omega_p&=\frac{2}{T_d}\tan\left(\frac{\omega_p}{2}\right)
        & \Omega_s&=\frac{2}{T_d}\tan\left(\frac{\omega_s}{2}\right)
        \\
        &=2\tan\left(\frac{0.4\pi}{2}\right)
        & &=2\tan\left(\frac{0.5\pi}{2}\right)
        \\
        &=1.4531
        & &=2
    \end{align}
    \end{solution}

    \part Determine las frecuencias de diseño si se utiliza impulso invariante
    con $T=0.005$
    \begin{solution}
        \begin{equation}
            \omega=2\pi\frac{F}{F_m}=\frac{\Omega}{F_m}=\Omega T
            \longrightarrow\Omega=\frac{\omega}{T}
        \end{equation}

        \begin{align}
            \Omega_p&=\frac{\omega_p}{T}
            & \Omega_s&=\frac{\omega_s}{T}
            \\
            &=\frac{0.4\pi}{0.005}
            & &=\frac{0.5\pi}{0.005}
            \\
            &=80\pi
            & &=100\pi
        \end{align}
    \end{solution}
\end{parts}

\question Dada la función de transferencia
\[
H_c(s)=\frac{1}{s+1}
\]
\begin{parts}
    \part Aplique transformación bilinear para encontrar el filtro digital
    $H_d(z)$.
    \begin{solution}
        Transformación bilinear: $\displaystyle s=\frac{2}{T_d}\frac{1-z^{-1}}{1+z^{-1}}$

        Escogiendo $T_d$ por conveniencia:
        \begin{align}
            \left.H_c(s)\right|_{s=\frac{1-z^{-1}}{1+z^{-1}}}
            &=\frac{1}{\frac{1-z^{-1}}{1+z^{-1}}+1}
            \cdot\left(\frac{1+z^{-1}}{1+z^{-1}}\right)
            \\
            &=\frac{1+z^{-1}}{2}
            \\
            &=0.5+0.5z^{-1}
        \end{align}
    \end{solution}

    \part Hallar la frecuencia de corte $\omega_c$ en el plano discreto.
    \begin{solution}
        El filtro $H_c(s)$ es un pasabajo de primer orden con frecuencia de corte
        $\Omega_c=1$

        Tomando en cuenta que se escogió $T_d=2$:
        \begin{align}
            \omega_c&=2\tan^{-1}\left(\Omega_c\frac{T_d}{2}\right)\\
            &=2\tan^{-1}\left(1\frac{2}{2}\right)\\
            &=\pi/2=1.57
        \end{align}

    \end{solution}
\end{parts}

\end{questions}
%\tableofcontents
%\listoffigures
%\listoftables
%\listoftodos

%%%%%%%%%%%%%%%%%%%%%%%%%%%%%%%%
%% -- Fin del Documento -- %%
%%%%%%%%%%%%%%%%%%%%%%%%%%%%%%%%
\end{document}